\section{Обобщенная схема редукция размерности}
Одним из подходов к решению многомерных задач глобальной оптимизации является сведение их к одномерным и использование эффективных одномерных алгоритмов глобального поиска к редуцированной задаче. В предыдущем разделе была изложена идея редукции размерности с использованием кривых Пеано. Ниже излагается обобщенный способ редукции размерности, комбинирующий использование разверток и схему вложенной (рекурсивной) оптимизации.
\subsection{Рекурсивная схема редукции размерности}
Схема рекурсивной оптимизации основана на известном [10] соотношении
 ,				(9)
которое позволяет заменить решение многомерной задачи (1) решением семейства одно-мерных подзадач, рекурсивно связанных между собой.
Введем в рассмотрение множество функций 
 ,				(10)
 , .		(11)
Тогда, в соответствии с соотношением (9), решение исходной задачи  сводится к решению одномерной задачи
 .					(12)
Однако при этом каждое вычисление значения одномерной функции   в некоторой фиксированной точке предполагает решение одномерной задачи минимизации  ,и так далее до вычисления   согласно (10).
\subsection{Блочная рекурсивная схема редукции размерности}
Для изложенной выше рекурсивной схемы предложено обобщение (блочная рекурсивная схема), которое комбинирует использование разверток и рекурсивной схемы с целью эффективного распараллеливания вычислений.
Рассмотрим вектор y как вектор блочных переменных
 ,
где i-я блочная переменная ui представляет собой вектор размерности   из последовательно взятых компонент вектора y, т.е.  ,  ,…, , причем  .
С использованием новых переменных основное соотношение многошаговой схемы (9) может быть переписано в виде
	 ,				(13)
где подобласти  , являются проекциями исходной области поиска D на подпространства, соответствующие переменным  .
Формулы, определяющие способ решения задачи на основе соотношений (13) в целом совпадают с рекурсивной схемой (10)(12). Требуется лишь заменить исходные перемен-ные  , на блочные переменные  . 
При этом принципиальным отличием от исходной схемы является тот факт, что в блочной схеме вложенные подзадачи
	 , ,			(14)
являются многомерными, и для их решения может быть применен способ редукции раз-мерности на основе кривых Пеано.
Число векторов и количество компонент в каждом векторе являются параметрами блочной многошаговой схемы и могут быть использованы для формирования подзадач с нужными свойствами. Например, если  , т.е.  , то блочная схема идентична исходной; каждая из вложенных подзадач является одномерной. А если  , т.е.  , то решение задачи эквивалентно ее решению с использованием единствен-ной развертки, отображающей [0,1] в D; вложенные подзадачи отсутствуют.
