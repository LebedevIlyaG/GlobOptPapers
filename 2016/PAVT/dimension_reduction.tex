\section{Обобщенная схема редукция размерности}
Одним из подходов к решению многомерных задач глобальной оптимизации является сведение их к одномерным и использование эффективных одномерных алгоритмов глобального поиска к редуцированной задаче. В предыдущем разделе была изложена идея редукции размерности с использованием кривых Пеано. Ниже излагается обобщенный способ редукции размерности, комбинирующий использование разверток и схему вложенной (рекурсивной) оптимизации.
\subsection{Рекурсивная схема редукции размерности}
Схема рекурсивной оптимизации основана на известном \cite{gorodGrishOptBook} соотношении
\begin{equation}
\label{nestedScheme}
\min{\phi(y):y\in D}=\min_{a_1\leqslant y_1\leqslant b_1}\min_{a_2\leqslant y_2\leqslant b_2}\dots\min_{a_1\leqslant y_N\leqslant b_N}\phi(y)
\end{equation}
которое позволяет заменить решение многомерной задачи (\ref{task}) решением семейства одно-мерных подзадач, рекурсивно связанных между собой.
\par
Введем в рассмотрение множество функций 
\begin{equation}
\label{phiN}
\phi_N(y_1,\dots,y_N)=\phi(y_1,\dots,y_N)
\end{equation}
\begin{equation}
\phi_i(y_1,\dots,y_i)=\min_{a_{i+1}\leqslant y_{i+1} \leqslant b_{i+1}}\phi_{i+1}(y_1,\dots,y_i,y_{i+1}),1\leqslant i\leqslant N-1
\end{equation}
\par
Тогда, в соответствии с соотношением (\ref{nestedScheme}), решение исходной задачи  сводится к решению одномерной задачи
\begin{equation}
\label{phiFirst}
\phi_1(y_1^*)=\min\{\phi_1(y_1),y_1\in [a,b]\}
\end{equation}
\par
Однако при этом каждое вычисление значения одномерной функции \(\phi_1(y_1)\) в некоторой фиксированной точке предполагает решение одномерной задачи минимизации \(\phi_2(y_1,y_2^*)=\min\{\phi(y_1,y_2):y_2\in [a_2,b_2]\}\), и так далее до вычисления \(\phi_N\) согласно (\ref{phiN}).
\subsection{Блочная рекурсивная схема редукции размерности}
Для изложенной выше рекурсивной схемы предложено обобщение (блочная рекурсивная схема), которое комбинирует использование разверток и рекурсивной схемы с целью эффективного распараллеливания вычислений.
\par
Рассмотрим вектор \(y\) как вектор блочных переменных
\begin{displaymath}
y=(y_1,\dots,y_N)=(u_1,u_2,\dots,u_M)
\end{displaymath}
где \(i\)-я блочная переменная \(u_i\) представляет собой вектор размерности \(N\) из последовательно взятых компонент вектора \(y\), т.е. \(u_1=(y_1,y_2,\dots,y_{N_1}),u_2=(y_{N_1+1},y_{N_1+2},\dots,y_{N_1+N_2}),\dots,u_M=(y_{N-N_M+1},y_{N-N_M+2},\dots,y_N)\), причем \(N_1+N_2+\dots+N_M=N\).
\par
С использованием новых переменных основное соотношение многошаговой схемы (\ref{nestedScheme}) может быть переписано в виде
\begin{equation}
\label{blockNested}
\min\phi(y)_{y\in D}=\min_{u_1\in D_1}\min_{u_2\in D_2}\dots\min_{u_M\in D_M}\phi(y)
\end{equation}
где подобласти \(D_i,1\leqslant i\leqslant M\), являются проекциями исходной области поиска \(D\) на подпространства, соответствующие переменным \(u_1,1\leqslant i\leqslant M\).
\par
Формулы, определяющие способ решения задачи на основе соотношений (\ref{blockNested}) в целом совпадают с рекурсивной схемой (\ref{phiN})-(\ref{phiFirst}). Требуется лишь заменить исходные переменные \(y_i,1\leqslant i\leqslant N\), на блочные переменные \(u_1,1\leqslant i\leqslant M\).
\par
При этом принципиальным отличием от исходной схемы является тот факт, что в блочной схеме вложенные подзадачи
\begin{equation}
\label{subTasks}
\phi_i(u_1,\dots,u_i)=\min_{u_{i+1}\in D_{i+1}}\phi_{i+1}(u_1,\dots,u_i,u_{i+1}),1\leqslant i\leqslant M-1
\end{equation}
являются многомерными, и для их решения может быть применен способ редукции размерности на основе кривых Пеано.
\par
Число векторов и количество компонент в каждом векторе являются параметрами блочной многошаговой схемы и могут быть использованы для формирования подзадач с нужными свойствами. Например, если \(M=N\), т.е. \(y_i=u_i,1\leqslant i\leqslant N\), то блочная схема идентична исходной; каждая из вложенных подзадач является одномерной. А если \(M=1\), т.е. \(u=u_1=y\), то решение задачи эквивалентно ее решению с использованием единственной развертки, отображающей \([0,1]\) в \(D\); вложенные подзадачи отсутствуют.