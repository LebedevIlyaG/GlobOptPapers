\section{Введение}
Рассматриваются задачи многоэкстремальной оптимизации и параллельные методы их решения. Важной особенностью указанных задач является тот факт, что глобальный экстремум есть интегральная характеристика задачи, таким образом, его отыскание связано с построением покрытия области поиска и вычислением значений оптимизируемой функции во всех точках этого покрытия. На сложность решения задач рассматриваемого класса решающее влияние оказывает размерность: вычислительные затраты растут экспоненциально при ее увеличении. Использование простейших способов решения (таких, как пере-бор на равномерной сетке) является неприемлемым. Требуется применение более экономных методов, которые порождают в области поиска существенно неравномерную сетку, более плотную в окрестности глобального минимума и разреженную вдали от него [13]. Данная статья продолжает развитие информационно-статистического подхода к построению параллельных алгоритмов глобальной оптимизации, предложенного в ННГУ им. Н.И. Лобачевского, который описан в монографиях [4, 5].
В рамках обсуждаемого подхода решение многомерных задач сводится к решению набора связанных подзадач меньшей размерности. Соответствующая редукция основана на использовании разверток единичного отрезка вещественной оси на гиперкуб. Роль таких разверток играют непрерывные однозначные отображения типа кривой Пеано, называемые также кривыми, заполняющими пространство. Еще одним используемым механизмом снижения размерности решаемой задачи является схема вложенной (рекурсивной) оптимизации. Численные методы, позволяющие эффективно использовать аппарат таких отображений, детально разработаны и обоснованы в [4, 5].
Алгоритмы, развиваемые в рамках информационно-статистического подхода, основаны на предположении липшицевости оптимизируемого критерия, что является типичными для других методов (см., например, [2, 3). Предположение такого рода выполняется для многих прикладных задач, поскольку относительные вариации функций, характеризую-щих моделируемую систему, обычно не превышают некоторый порог, определяемый ограниченной энергией изменений в системе.
Использование современных параллельных вычислительных систем расширяет сферу применения методов глобальной оптимизации и, в то же время, ставит задачу эффективно-го распараллеливания процесса поиска. Именно поэтому разработка эффективных параллельных методов для численного решения задач многоэкстремальной оптимизации и со-здание на их основе программных средств для современных вычислительных систем является актуальной задачей. Особый интерес представляет разработка схем распараллеливания, позволяющих эффективно использовать ускорители вычислений, такие, как сопроцессор Intel Xeon Phi.
В рамках проводимого исследования мы будем предполагать, что время проведения одного испытания (вычисления значения функции в области поиска) может значительно отличаться в различных решаемых задачах. Это определяет разработку разных подходов к распараллеливанию в задачах с «легкими» и «сложными» критериями. В статье приведено описание универсального подхода к распараллеливанию алгоритма глобального по-иска, который охватывает оба этих случая. Указанный подход реализован в разработанной в ННГУ им. Н.И. Лобачевского параллельной программной системе решения задач глобальной оптимизации.
