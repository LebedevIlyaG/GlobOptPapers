\documentclass[a4paper]{article}
\usepackage{fontspec}
\usepackage{amsmath}
\usepackage{amssymb}

\usepackage[a4paper]{geometry}
\usepackage{indentfirst}
\usepackage{graphicx}
\usepackage{caption}
\usepackage{subcaption}

\setmainfont{CMU Serif}
\setsansfont{CMU Sans Serif}

\bibliographystyle{unsrt}
\fontsize{10}{12pt}\selectfont

\begin{document}

\title{Implementation of the parallel asynchronous global search algorithm on Intel Xeon Phi\
\footnote{
The paper was supported by the Russian Science Foundation, project No 15-11-30022 “Global optimization, supercomputing  computations, and applications”}}
\author{K.A. Barkalov, I.G. Lebedev, V.V. Sovrasov, A.V. Sysoyev}
\date{}
\maketitle

\begin{abstract}
The paper considers parallel algorithm for solving multiextremal optimization problems. The issues of implementation of the algorithm on state-of-the-art computing systems using Intel Xeon Phi coprocessor are examined. Two approaches for algorithm parallelization, which take into account information about laboriousness of the objective function computing, are considered. Speed up of the algorithm using Xeon Phi compared to the algorithm using CPU only is experimentally confirmed. Computational experiments are carried out on Lobachevsky supercomputer.
\par
\textit{Keywords:}global optimization, multiextremal functions, dimension reduction, parallel computing, Intel Xeon Phi.
\end{abstract}
\nocite{*}
\bibliography{refs_eng}
\end{document}