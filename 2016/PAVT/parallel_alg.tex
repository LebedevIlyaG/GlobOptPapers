\section{Параллельный алгоритм глобального поиска}
Рассмотрим задачу поиска глобального минимума N-мерной функции (y) в гиперинтервале   Будем предполагать, что функция удовлетворяет условию Липшица с априори неизвестной константой L.
	 ,					(1)
	 .			(2)
Существует ряд способов адаптации эффективных одномерных алгоритмов для решения многомерных задач, см., например, методы диагонального [6] или симплексного [7] разбиения области поиска. В данной работе мы будем использовать подход, основанный на идее редукции размерности с помощью кривой Пеано y(x), непрерывно и однозначно отображающей отрезок вещественной оси [0,1] на n-мерный куб 
 .
Вопросы численного построения отображений типа кривой Пеано и соответствующая теория подробно рассмотрены в [4]. Здесь же отметим, что численно построенная раз-вертка является приближением к теоретической кривой Пеано с точностью порядка , где m – параметр построения развертки. Использование подобного рода отображений позволяет свести многомерную задачу к одномерной задаче 
 
Важным свойством является сохранение ограниченности относительных разностей функции: если функция (y) в области D удовлетворяла условию Липшица, то функция (y(x)) на интервале [0,1] будет удовлетворять равномерному условию Гельдера 
 ,  ,
где константа Гельдера H связана с константой Липшица L соотношением 
 ,  .
Поэтому, не ограничивая общности, можно рассматривать минимизацию одномерной функции  ,  , удовлетворяющей условию Гельдера.
Рассматриваемый алгоритм решения данной задачи предполагает построение последовательности точек xk, в которых вычисляются значения минимизируемой функции zk = f(xk). Процесс вычисления значения функции (включающий в себя построение образа yk=y(xk)) будем называть испытанием, а пару (xk,zk) – результатом испытания. Множество пар {(xk,zk)}, 1kn составляют поисковую информацию, накопленную методом после про-ведения n шагов. В нашем распоряжении имеется   вычислительных элементов и в рамках одной итерации метода мы будем проводить p испытаний одновременно. Обозначим   общее число испытаний, выполненных после n параллельных итераций.
На первой итерации метода испытание проводится в произвольной внутренней точке x1 интервала [0,1]. Пусть выполнено   итераций метода, в процессе которых были про-ведены испытания в  k = k(n) точках xi, 1ik. Тогда точки   поисковых испытаний следующей  -ой итерации определяются в соответствии с правилами:
Шаг 1. Перенумеровать точки множества  , которое включает в себя граничные точки интервала [0,1], а также точки предшествующих испытаний, ниж-ними индексами в порядке увеличения значений координаты, т.е. 
 
Шаг 2. Полагая  , вычислить величины 
 ,  ,	(3)
где   является заданным параметром метода, а  .
Шаг 3. Для каждого интервала  , вычислить характеристику в соответствии с формулами
 ,  ,				(4)
 ,  .			(5)
Шаг 4. Характеристики  , упорядочить в порядке убывания 
	 				(6)
и выбрать   наибольших характеристик с номерами интервалов  .
Шаг 5. Провести новые испытания в точках  , вычисленных по формулам
 ,  
 ,  	(7)
Алгоритм прекращает работу, если выполняется условие   хотя бы для одного номера  ; здесь   есть заданная точность. В качестве оценки глобально-оптимального решения задачи  выбираются значения 
 ,  				(8)
Теоретическое обоснование данного способа организации параллельных вычислений изложено в [5].
