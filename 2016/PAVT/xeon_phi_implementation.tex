\section{Реализация на Xeon Phi}
В2012 году компания Intel представила первый сопроцессор с архитектурой Intel MIC (Intel® Many Integrated Core Architecture). Архитектура MIC позволяет использовать боль-шое количество вычислительных ядер архитектуры x86 в одном процессоре. В результате для параллельного программирования могут быть использованы стандартные технологии, такие как OpenMP и MPI. 
Архитектура Intel Xeon Phi поддерживает несколько режимов использования сопроцессора, которые можно комбинировать для достижения максимальной производительно-сти в зависимости от характеристик решаемой задачи. 
В режиме Offload процессы MPI выполняются только на CPU, а на сопроцессоре про-исходит запуск отдельных функций, аналогично использованию графических ускорителей.
В режиме MPI базовая система и каждый сопроцессор Intel Xeon Phi рассматриваются как отдельные равноправные узлы, и процессы MPI могут выполняться на центральных процессорах и сопроцессорах Xeon Phi в произвольных сочетаниях.
\subsection{Режим Offload}
Вначале рассмотрим ситуацию, когда проведение одного испытания является трудоемкой операцией. В этом случае ускоритель Xeon Phi может быть использован в режиме Offload для параллельного проведения сразу многих испытаний на одной итерации метода. Пересылки данных от CPU к Xeon Phi будут минимальны – требуется лишь передать на сопроцессор координаты точек испытаний, и получить обратно значения функции в этих точках. Функции, определяющие обработку результатов испытаний в соответствии с алгоритмом и требующие работы с большим объемом накопленной поисковой информацией, могут быть эффективно реализованы на CPU. Общая схема организации вычислений с использованием Xeon Phi будет следующий.
На CPU выполняются шаги 1 – 4 параллельного алгоритма глобального поиска из п. 2. При этом на каждой итерации происходит накопление координат точек испытания в буфере, и этот буфер передается на сопроцессор. На Xeon Phi выполняется параллельное вы-числение значений функции в этих точках (шаг 5 алгоритма). Используется OpenMP распараллеливание цикла, на каждой итерации которого происходит вычисление значений функции. По завершению испытаний происходит передача вычисленных значений функ-ции на CPU. 
\subsection{Режим MPI}
В случае, если проведение одного поискового испытания является относительно про-стой операцией, параллельное проведение многих итераций в режиме Offload не дает большого ускорения (сказывается влияние накладных расходов на передачу данных). Од-нако здесь можно увеличить вычислительную нагрузку на Xeon Phi, если применить блочную схему редукции размерности из п. 3.2, а для решения возникающих подзадач использовать сопроцессор в режиме MPI.
Для организации параллельных вычислений будем использовать небольшое (2-3) число уровней вложенности в блочной схеме, при котором исходная задача большой размерности разбивается на 2-3 вложенные подзадачи меньшей размерности. Тогда, применяя в блочной рекурсивной схеме (13) для решения вложенных подзадач (14) параллельный алгоритм глобальной оптимизации, мы получим схему параллельных вычислений с широ-кой степенью вариативности (например, можно варьировать количество процессоров на различных уровнях оптимизации, т.е. при решении подзадач по различным переменным  ).
Общая схема организации вычислений с использованием нескольких узлов кластера и нескольких сопроцессоров состоит в следующем. Процессы параллельной программы об-разуют дерево, соответствующее уровням вложенных подзадач, при этом вложенные под-задачи
 
при i1,…,M–2 решаются только с использованием CPU. Непосредственно в данных под-задачах вычислений значений оптимизируемой функции не происходит: вычисление значения функции   это решение задачи минимизации следующего уровня. Каждая подзадача решается в отдельном процессе; обмен данными осуществляется лишь между процессами-предками и процессами-потомками.
Подзадача последнего (M–1)-го уровня 
 
отличается от всех предыдущих подзадач – в ней происходит вычисление значений оптимизируемой функции, т.к.  . Подзадачи этого уровня решаются на сопроцессоре, и каждое ядро сопроцессора будет решать свою подзадачу (M–1)-го уровня в отдельном MPI-процессе.
Самый простой вариант использования данной схемы будет соответствовать двухкомпонентному вектору распараллеливания  . Здесь   будет соответствовать числу MPI-процессов на CPU, а  числу MPI-процессов на Xeon Phi; тем самым общее количество процессов будет определяться как  .
Отметим, что синхронный параллельный алгоритм глобальной оптимизации, описанный в п.3, в сочетании с блочной схемой редукции размерности обладает существенным недостатком, связанным с возможными простоями всех процессов, кроме корневого. Простои могут возникать в том случае, если часть потомков некоторого процесса закончили решение своих подзадач и отправили данные родителю раньше остальных. Для преодоления данной проблемы был применен предложенный в [12] асинхронный вариант параллельного алгоритма. Конкретные детали реализация блочной схемы в сочетании с асинхронным алгоритмом описаны в [12].
