\documentclass[11pt, oneside, a4paper]{article}
\usepackage[utf8]{inputenc}
%\usepackage[cp1251]{inputenc} % кодировка
\usepackage[english, russian]{babel} % Русские и английские переносы
\usepackage{graphicx}          % для включения графических изображений
\usepackage{cite}              % для корректного оформления литературы
\usepackage{pavt-ru}
\usepackage{amssymb}

\begin{document}

% \title - название статьи
% \authors - список авторов

\title{Использование параллельной системы глобальной оптимизации Globalizer для
решения задач оптимального управления\footnote{Исследование выполнено при финансовой
поддержке РФФИ в рамках научного проекта № 16-31-00244 мол\_а <<Параллельные методы
решения вычислительно трудоемких задач глобальной оптимизации на гибридных кластерных системах>>}}

\authors{И.Г.~Лебедев, В.В.~Соврасов}
\organizations{Нижегородский государственный университет им. Н.И. Лобачевского}
\setlength{\abovedisplayskip}{3pt}
\setlength{\belowdisplayskip}{3pt}

Задача многомерной многоэкстремальной оптимизации может быть поставлена следующим образом:
необходимо найти наименьшее значение вещественной функкции \(\varphi(y)\) в некоторой области \(D\),
задаваемой функциональными ограничениями:
\begin{displaymath}
  \label{task}
    \varphi(y^*)=\min\{\varphi(y):y\in D\}, D=\{x\in \mathbf{R}^n: g_j(x) \leqslant 0, j=\overline{1,m}\}
\end{displaymath}

В ННГУ им. Н.И. Лобачевского под руководством проф. Р.Г. Стронгина разработан
эффективный подход к решению задач глобальной оптимизации \cite{strGergrParOptBook}.
В рамках данного подхода решение многомерной задачи сводится к решению одномерной.
Для редукции размерности используются кривые Пеано, однозначно отображающие отрезок
вещественной оси \([0,1]\) на \(n\)-мерный куб. Для организации параллельных вычислений
используется параллельный алгоритм глобального поиска, эффективность которого была
показана ранее в \cite{parallelMethod}.

Задача глобальной оптимизации возникает при синтезе оптимальных с точки зрения некоторых
критериев управлений в линейных системах ОДУ. Если управление является линейной обратной
связью по состоянию, то система с управлением имеет вид:
\begin{displaymath}
    \dot x = (A+B_u\Theta)x + B_v v, x(0)=0,
\end{displaymath}
где  \(v(t)\in L_2\) --- некоторое возмущение.
Выходы системы описываются формулами \(z_k=(C_k+B_u\Theta),k=\overline{1,N}\).
Вляиние возмущения на \(k\)-й выход системы описывается критерием
\(J_k(\Theta)=\sup_{v\in L_2} \frac{\max_{1\leqslant i \leqslant n_k} \sup_{t\geqslant 0}|z_k^{(i)}(\Theta,t)|}{||v||_2}\).

Нужно найти компоненты вектора \(\Theta\), минимизирующие один из критериев при
заданных ограничениях на другие: \(J_1(\Theta^*)=\min\{J_1(\Theta):J_k(\Theta)\leqslant S_k,k=\overline{2,N}\}\).

В \cite{optControl} указан способ вычисления критериев, состоящий в решении СЛАУ,
а также даны решения рассматриваемой задачи в некоторых частных случаях. На
данных момент результаты, полученные в \cite{optControl}, повторены с помощью системы Globalizer,
ведётся подготовка к решению задач большей размерности из рассматриваемого класса, в которых
вычисление критериев --- трудоемкий процесс, требующий привлечения ресурсов вычислительного кластера.

\begin{biblio}

\bibitem{strGergrParOptBook}
Стронгин Р.Г. Гергель В.П. Гришагин
  В.А.~Баркалов К.А.
\newblock {Параллельные вычисления в задачах
  глобальной оптимизации}.
\newblock М.: Издательство Московского
  университета, 2013, 280с.

 \bibitem{parallelMethod}
Баркалов К.А. Лебедев И.Г. Соврасов В.В. Сысоев А.В.
\newblock {Реализация параллельного
алгоритма поиска глобального экстремума функции на Intel XEON PHI //
Параллельные вычислительные технологии (ПаВТ’2016) труды международной
научной конференции. 2016. С. 68-80.}

\bibitem{optControl}
 Д.В. Баландин М.М. Коган
 \newblock {Оптимальное по Парето обобщенное
 \(H_2\)-управление и задачи виброзащиты // Автоматика и телемеханика. Принято к печати.}

\end{biblio}
\end{document}
