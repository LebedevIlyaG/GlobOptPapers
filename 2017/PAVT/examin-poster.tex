\documentclass[11pt, oneside, a4paper]{article}
\usepackage[utf8]{inputenc}
%\usepackage[cp1251]{inputenc} % кодировка
\usepackage[english, russian]{babel} % Русские и английские переносы
\usepackage{graphicx}          % для включения графических изображений
\usepackage{cite}              % для корректного оформления литературы
\usepackage{pavt-ru}

\begin{document}

% \title - название статьи
% \authors - список авторов

\title{Использование параллельной системы глобальной оптимизации Globalizer для
решения задач оптимального управления\footnote{Исследование выполнено при финансовой
поддержке РФФИ в рамках научного проекта № 16-31-00244 мол\_а <<Параллельные методы
решения вычислительно трудоемких задач глобальной оптимизации на гибридных кластерных системах>>}}

\authors{И.Г.~Лебедев, В.В.~Соврасов}
\organizations{Нижегородский государственный университет им. Н.И. Лобачевского}

Задача многомерной многоэкстремальной оптимизации может быть определена как проблема
поиска наименьшего значения действительной функции \(\varphi(y)\) в некоторой области \(D\),
задаваемой функциональными ограничениями:
\begin{displaymath}
  \label{task}
    \varphi(y^*)=\min\{\varphi(y):y\in D\}
\end{displaymath}
\begin{displaymath}
  D=\{x\in \mathbf{R}^n: g_j(x) \le 0, j=\overline{1,m}\}
\end{displaymath}

В ННГУ им. Н.И. Лобачевского под руководством проф. Р.Г. Стронгина разработан
эффективный подход к решению задач глобальной оптимизации \cite{strGergrParOptBook}.
В рамках данного подхода решение многомерных задач сводится к решению серии вложенных задач меньшей размерности.
Главным из рассматриваемых способов редукции размерности является использование кривой Пеано,
однозначно отображающей отрезок вещественной оси \([0,1]\) на \(n\)-мерный куб.
Для организации параллельных вычислений используется параллельный алгоритм глобального поиска,
эффективность которого была показана ранее в \cite{parallelMethod}.

Задача глобальной оптимизации возникает при синтезе оптимальных с точки зрения некоторых критериев управлений в линейных системах ОДУ.
Если управление является линейной обратной связи по состоянию, то система с управлением имеет вид:
\begin{displaymath}
    \dot x = (A+B_u\Theta)x + B_v v, x(0)=0,
\end{displaymath}
где  \(v(t)\in L_2\) --- некоторое возмущение.
Выходы системы описываются формулами \(z_k=(C_k+B_u\Theta),k=\overline{1,N}\).
Вляиние возмущения на \(k\)-й выход системы описывается критерием:
\begin{displaymath}
  J_k(\Theta)=\sup_{v\in L_2} \frac{\max_{1\le i \le n_k} \sup_{t\ge 0}|z_k^{(i)}(t)|}{||v||_2}
\end{displaymath}

Задача оптимизации состоит в том, чтобы найти компоненты вектора \(\Theta\), минимизирующие
один из критериев при заданных ограничениях на другие: \(J_1(\Theta^*)=\min\{J_1(\Theta):J_k(\Theta)\le S_k,k=\overline{2,N}\})\).

В \cite{optControl} указан способ вычисления критериев, сводящийся к решении нескольких задач линейной алгебры,
а также даны решения задачи поиска оптимального управления в некоторых частных случаях. На данных момент результаты,
полученные в \cite{optControl} повторены с помощью системы Globalizer, ведётся
подготовка к решению более сложных задач из рассматриваемого класса, в которых вычисление критериев --- довольно
трудоемкий процесс, требующий привлечения ресурсов вычислительного кластера.

\begin{biblio}

\bibitem{strGergrParOptBook}
Стронгин Р.Г. Гергель В.П. Гришагин
  В.А.~Баркалов К.А.
\newblock {\em Параллельные вычисления в задачах
  глобальной оптимизации}.
\newblock М.: Издательство Московского
  университета, 2013, 280с.

 \bibitem{parallelMethod}
 И.Г. Лебедев, В.В. Соврасов

\bibitem{optControl}
 Д.В. Баландин М.М. Коган
 \newblock {\em Оптимальное по Парето обобщенное
 \(H_2\)-управление и задачи виброзащиты}.

\end{biblio}
\end{document}
