\documentclass{aip-cp}

\usepackage[numbers]{natbib}
\usepackage{rotating}
\usepackage{graphicx}
\pdfmapfile{+txfonts.map}

% Document starts
\begin{document}

% Title portion
\title{The Title Goes Here with Each Initial Letter Capitalized}

\author{Victor Gergel}
\author{Konstantin Barkalov\corref{cor1}}
%\eaddress[url]{http://www.aip.org}
\author{Ilya Lebedev}
%\eaddress{anotherauthor@thisaddress.yyy}

\affil{Lobachevsky State University of Nizhni Novgorod, Nizhni Novgorod, Russia.}
\corresp[cor1]{Corresponding author: konstantin.barkalov@itmm.unn.ru}

\maketitle


\begin{abstract}
Abstract.
\end{abstract}

% Head 1
\section{INTRODUCTION}

In the present paper, the methods for generating the global optimization test problems with non-convex constraints
\begin{eqnarray}
&\varphi(y^\ast)=\min{\left\{\varphi(y):y\in D, \; g_i(y)\leq 0, \; 1 \leq i \leq m\right\}}, \label{i_problem} \\
&D=\left\{y\in R^N: a_i\leq y_i \leq b_i, 1\leq i \leq N\right\} \label{D}
\end{eqnarray}
are considered. The objective function $\varphi(y)$ (henceforth denoted by $g_{m+1}(y)$) and the left-hand sides $g_i(y), \; 1\leq i \leq m,$ of the constraints are supposed to satisfy the Lipschitz condition
\[ \left|g_i(y')-g_i (y'')\right| \leq L_i \left\|y'-y'' \right\|, \; y',y''\in D, \; 1\leq i \leq m+1. \]
with the Lipschitz constants unknown a priori. The analytical formulae of the problem functions may be unknown, i.e. these ones may be defined by an algorithm for computing the function values in the search domain (so called ''black-box''-functions). It is supposed that even a single computing of a problem function value may be a time-consuming operation since it is related to the necessity of numerical modeling in the applied problems (see, for example, \cite{Famularo1999,Menniti2008}).


The evaluation of efficiency of the developed methods is one of the key problems in the optimization theory and applications. Unfortunately, it is difficult to obtain any theoretical estimates in many cases. As a result, the comparison of the methods is performed by carrying out the computational experiments on solving some test optimization problems in most cases. In order to obtain a reliable evaluation of the efficiency of the methods, the sets of test problems should be diverse and representative enough. The problem of choice of the test problems has been considered in a lot of works (see, for example, \cite{Floudas1999}). Unfortunately, in many cases, the proposed sets contain a small number of test problems, and it is difficult to obtain the problems with desired properties. The most important drawback consists of the fact that the constraints are absent in the proposed test problems as a rule (or the constraints are relatively simple: linear, convex, etc.).



Do not abbreviate Figure , Equation, etc.; display items are always singular, i.e., Figure 1 and 2. Equations are always singular, i.e., Equation 1 and 2, and should be inserted using the Equation Editor, not as graphics, in the main text.  Display items and captions should be inserted after the reference section. Please do not use footnotes in the text, additional information can be added to the reference list.


% Acknowledgement
\section{ACKNOWLEDGMENTS}
This research was supported by the Russian Science Foundation, project No 16-11-10150.

% References

\nocite{*}
\bibliographystyle{aipnum-cp}%
\bibliography{LeGo}%


\end{document}
