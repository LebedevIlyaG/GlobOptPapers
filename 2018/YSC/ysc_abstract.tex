% easychair.tex,v 3.1 2011/12/30
%
% Select appropriate paper format in your document class as
% instructed by your conference organizers. Only withtimes
% and notimes can be used in proceedings created by EasyChair
%
% The available formats are 'letterpaper' and 'a4paper' with
% the former being the default if omitted as in the example
% below.
%
\documentclass[procedia]{easychair}
%\documentclass[debug]{easychair}
%\documentclass[verbose]{easychair}
%\documentclass[notimes]{easychair}
%\documentclass[withtimes]{easychair}
%\documentclass[a4paper]{easychair}
%\documentclass[letterpaper]{easychair}

% This provides the \BibTeX macro
\usepackage{doc}
\usepackage{makeidx}

% In order to save space or manage large tables or figures in a
% landcape-like text, you can use the rotating and pdflscape
% packages. Uncomment the desired from the below.
%
% \usepackage{rotating}
% \usepackage{pdflscape}

% If you plan on including some algorithm specification, we recommend
% the below package. Read more details on the custom options of the
% package documentation.
%
% \usepackage{algorithm2e}

\def\procediaConference{99th Conference on Topics of
  Superb Significance (COOL 2014)}

%\makeindex

%% Front Matter
%%
% Regular title as in the article class.
%
\title{Comparison of dimensionality reduction schemes for parallel global optimization
algorithms}

% \titlerunning{} has to be set to either the main title or its shorter
% version for the running heads. When processed by
% EasyChair, this command is mandatory: a document without \titlerunning
% will be rejected by EasyChair

\titlerunning{Dimension reduction schemes in parallel GO}

% Authors are joined by \and. Their affiliations are given by \inst, which indexes into the list
% defined using \institute
%
\author {
Vladislav Sovrasov \and
Ilya Lebedev
}

% Institutes for affiliations are also joined by \and,
\institute  {
  Lobachevsky State University of Nizhni Novgorod, Nizhni Novgorod, Russia
  \email{sovrasov.vlad@gmail.com}\\
  \email{ilya.lebedev@itmm.unn.ru}\\
}
%  \authorrunning{} has to be set for the shorter version of the authors' names;
% otherwise a warning will be rendered in the running heads. When processed by
% EasyChair, this command is mandatory: a document without \authorrunning
% will be rejected by EasyChair

\authorrunning{Sovrasov V., Lebedev I.}

\begin{document}

\maketitle

\keywords{Global optimization, Dimension reduction, Parallel algorithms,
Multidimensional multiextremal optimization, Global search algorithms}

\begin{abstract}
This work considers a parallel algorithms for solving multi-extremal optimization problems.
Algorithms are developed within the framework of the information-statistical approach and
implemented in a parallel solver Globalizer. The optimization problem is solved by reducing
the multidimensional problem to a set of joint one-dimensional problems that are solved in
parallel. Five types of Peano-type space-filling curves are employed to reduce dimension. The
results of computational experiments carried out on several hundred test problems are discussed.
\end{abstract}


%\setcounter{tocdepth}{2}
%{\small
%\tableofcontents}

%\section{To mention}
%
%Processing in EasyChair - number of pages.
%
%Examples of how EasyChair processes papers. Caveats (replacement of EC
%class, errors).


%------------------------------------------------------------------------------
\section{Introduction}
\label{sec:intro}

In the present paper, the parallel algorithms for solving the multiextremal optimization problems
are considered. In the multiextremal problems, the opportunity of reliable estimate of the global
optimum is based principally on the availability of some information on the function known
{\textit a priori} allowing relating the probable values of the optimized function to the known
values at the points of performed trials. Very often, such an information on the problem being
solved is represented in the form of suggestion that the objective function $\varphi(y)$ satisfies
Lipschitz condition with the constant $L$ not known a priori (see, for example,
\cite{Jones,Gablonsky,Evtushenko}). At that, the objective function could be defined by a
program code i. e. could represent a ``black-box''-function. Such problems are presented in the
applications widely (problems of optimal design of objects and technological processes in
various fields of technology, problems of model fitting according to observed data in scientific
research, etc.).

Many methods destined to solving the problems of the class specified above reduce the solving
of a multidimensional problem to solving the one-dimensional subproblems implicitly (see, for
example, the methods of diagonal partitions \cite{Sergeyev2006,SergeyevKvasov2015} or
simplicial partitions \cite{Zilinskas2008,Zilinskas2014}). In the present work, we will use the
approach developed in Lobachevsky State University of Nizhni Novgorod based on the idea of
the dimensionality reduction with the use of Peano space-filling curves $y(x)$ mapping the
interval $[0,1]$ of the real axis onto an $n$-dimensional cube continuously and unambiguously.

Several methods of constructing the evolvents approximating the theoretical Peano curve have
been proposed in \cite{strongin1978,Strongin1992,Goryachih2017,Gergel2009}. These
methods were implemented in the Globalizer software system \cite{globalizerSystem}. The goal
of the present study was comparing the properties of the evolvents and the selecting the most
suitable ones for the use in the parallel global optimization algorithms.

%------------------------------------------------------------------------------
% Index
%\printindex
\begin{thebibliography}{107}

\bibitem{strongin1978}% (MR509033)
\newblock R. G. Strongin,
\newblock \emph{Numerical Methods in Multi-Extremal Problems (Information-Statistical Algorithms)},
\newblock Moscow: Nauka (1978) (In Russian)

\bibitem{Strongin1992}% (MR1263606) [10.1007/BF00122428]
\newblock R. G. Strongin,
\newblock \emph{\emph{Algorithms for multi-extremal mathematical programming problems
employing a set of joint space-filling curves}},
\newblock \emph{J. Glob. Optim.}, \textbf{2}, 357--378 (1992)

\bibitem{Goryachih2017}
\newblock Goryachih, A.
\newblock \emph{A class of smooth modification of space-filling curves for global optimization
problems}
\newblock Springer Proceedings in Mathematics and Statistics, \textbf{197}, pp. 57--65 (2017)

\bibitem{Gergel2009}
Strongin, R.G., Gergel, V.P., Barkalov, K.A.: \emph{Parallel methods for global optimization problem
solving.} Journal of instrument engineering. \textbf{52}, 25--33 (2009) (In Russian)

\bibitem{globalizerSystem}
Gergel V.P., Barkalov K.A., and Sysoyev A.V: Globalizer: \emph{A novel supercomputer software
system for solving time-consuming global optimization problems.} Numerical Algebra, Control
\& Optimization \textbf{8(1)}, 47--62 (2018)

\bibitem{Sergeyev2006}
Sergeyev, Ya.D., Kvasov, D.E.: \emph{Global search based on efficient diagonal partitions and a set
of Lipschitz constants}. SIAM J. Optim \textbf{16(3)}, 910--937 (2006)

\bibitem{SergeyevKvasov2015}
Sergeyev, Y.D., Kvasov, D.E.: \emph{A deterministic global optimization using smooth diagonal
auxiliary functions.} Communications in Nonlinear Science and Numerical Simulation. \textbf{21(1-3)},
99--111 (2015)

\bibitem{Zilinskas2008}
\v Zilinskas, J.: \emph{Branch and bound with simplicial partitions for global optimization.} Math.
Model. Anal. \textbf{13(1)}, 145--159 (2008)

\bibitem{Zilinskas2014}
Paulavi\v cius, R., \v Zilinskas, J.: \emph{Simplicial Lipschitz optimization without the Lipschitz
constant.} J. Glob. Optim. \textbf{59(1)}, 23-40 (2014)

\bibitem{Jones}
Jones, D.R.: The direct global optimization algorithm. In: Floudas, C.A., Pardalos, P.M.
(eds.) \emph{The Encyclopedia of Optimization}, 2nd edn., pp. 725--735. Springer, Heidelberg
(2009)

\bibitem{Gablonsky}
Gablonsky, J.M., Kelley, C.T.: \emph{A locally-biased form of the DIRECT algorithm.} J. Glob.
Optim., \textbf{21(1)}, 27--37 (2001)

\bibitem{Evtushenko}
Evtushenko, Y., Posypkin, M.: \emph{A deterministic approach to global box-constrained
optimization.} Optim. Lett. \textbf{7(4)}, 819--829 (2013)

\end{thebibliography}

%------------------------------------------------------------------------------
\end{document}

% EOF
