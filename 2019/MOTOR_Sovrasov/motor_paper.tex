% This is samplepaper.tex, a sample chapter demonstrating the
% LLNCS macro package for Springer Computer Science proceedings;
% Version 2.20 of 2017/10/04
%
\documentclass[runningheads]{llncs}
%
\usepackage{graphicx}
\usepackage{amssymb}
\usepackage{subfig}
\usepackage{cite}

% Used for displaying a sample figure. If possible, figure files should
% be included in EPS format.
%
% If you use the hyperref package, please uncomment the following line
% to display URLs in blue roman font according to Springer's eBook style:
% \renewcommand\UrlFont{\color{blue}\rmfamily}

\begin{document}
%
\title{Comparison of several derivative-free global optimization algorithms\thanks{The study was supported by the Russian Science Foundation, project No 16-11-10150.}}
%
%\titlerunning{Abbreviated paper title}
% If the paper title is too long for the running head, you can set
% an abbreviated paper title here
%
\author{Vladislav Sovrasov}
%
\authorrunning{V. Sovrasov}
%
\institute{Lobachevsky State University of Nizhni Novgorod, Russia \\
  \email{sovrasov.vlad@gmail.com}
}%
\maketitle              % typeset the header of the contribution
%

\begin{abstract}
This work considers several stochastic and deterministic deri-vative-free global optimization
algorithms. In the first part of the paper popular sequential open-source solvers are compared against
Globalizer solver, which is developed at the Lobachevsky State University. The Globalizer is
designed to solve problems with black-box objectives satisfying the Lipschitz condition and shows
competitive performance with other similar solvers. The comparison is done on several sets of
challenging multi-extremal benchmark functions. The second part of this work is devoted to
comparison between the Globalizer and MIDACO solvers on systems with shared and distributed
memory. MIDACO is a state-of-the-art global solver included to the TOMLAB optimization
environment for MATLAB. Results of the benchmark show advantages of the Globalizer on small-dimensional, but sufficiently multi-extremal benchmark functions.
% We would like to encourage you to list your keywords within
% the abstract section using the \keywords{...} command.
\keywords{deterministic global optimization $\cdot$ stochastic global optimization
  $\cdot$ parallel numerical methods $\cdot$ derivative-free algorithms $\cdot$ black-box
optimization}
\end{abstract}
%
\section{Introduction}

Nonlinear global optimization of non-convex functions is considered to be one of the most
difficult problems of mathematical programming. The finding of the global minimum of
a function of several variables often appears to be more difficult than local optimization in a
thousand-dimensional space. In local optimization applying of the simplest gradient descent
method may appear to be enough, whereas in order to \textit{guarantee} finding the global
optimum, an optimization method has to accumulate information about behavior of the
objective function within the whole search domain
\cite{Jones2009,Paulavicius2011,Evtushenko2013,strSergGO}. Nowadays, various stochastic global
optimization algorithms have attracted much attention, first of all, the evolution ones
\cite{Storn1997, SCHLUTER2009, KennedyEberhart1995}. They have rather simple structure
and allow solving the problems of large dimensionality. However, the evolution algorithms ensure
the global convergence in the probabilistic sense only.

In the present paper, the open-source implementations of nine different global optimization methods
were considered. All the algorithms were tested using a set of 900 essentially multiextremal functions,
which has been generated using generators of test problems \cite{Gaviano2003, grishaginClass}.
Besides the comparison of the sequential algorithms, a comparison of the MIDACO solver
\cite{Schlueter2012} with the Globalizer software system \cite{globalizerSystem,Strongin2018} on a subset of 200 test problems has been
performed in the Lobachevsky supercomputer environment.


\section{Related Work}

Earlier, the comparison of the stochastic global optimization algorithms \cite{Ali2005, JSSv060i06}
as well as of the deterministic ones \cite{posik2012, KVASOV2018245, Liberti2005} between each
other has been considered in the literature. In these works, most of modern methods have
been studied in details, but the emphasis was made mainly on the sequential algorithms while the
reliability and the convergence speed were the main criteria of efficiency of the optimization
methods. In the majority of works, the sets of well-known test problems (for example, the Rastrigin
function, Ackley function, etc.) were taken as the sets of test functions. The sizes of such sets don't
exceed 100 different functions usually, some of which can be the single-extremal ones (such as the
Rosenbrock function).

In Ref. \cite{Beiranvand2017}, some general principles were formulated, which, in the author's
opinion, should be obeyed when comparing the optimization methods. In particular, the authors say
about the advantages of the problem generators allowing generating the large sets of problems thus
minimizing the random effects when comparing the methods. At the same time, the use of a single
generator can appear to be not enough for a comprehensive comparison of the methods. In order to
overcome this problem in part, the authors of Ref. \cite{Beiranvand2017} advise to use several
generators of various nature and to create the sets of problems of various complexity.

Taking into account the experience of the preceding works in the field of comparison of the
optimization methods, two generators of the test problems of different nature will be used in the
present work. Using these ones, 9 sets of 100 problems of various complexity with the
dimensionality varying from 2 to 5 were generated. Besides the comparison of the sequential
methods, a comparison of the efficiencies of two parallel algorithms is presented in the work as well.

\section{Statement of Multidimensional Global Optimization Problem}
In this paper, the core class of optimization problems, which can be solved using
global optimization methods, is formulated. This class involves the multidimensional global
optimization problems without constraints, which can be defined in the following way:
\begin{equation}
\label{eq:task}
\begin{array}{cr}\\
  \varphi(y^*)=\min\{\varphi(y):y\in D\}, \\
  D=\{y\in \mathbb{R}^N:a_i\leq y_i\leq{b_i}, 1\leq{i}\leq{N}\}
\end{array}
\end{equation}
with the given boundary vectors  $a$ and  $b$. It is supposed, that the objective function
\(\varphi(y)\) satisfies the Lipschitz condition
\begin{equation}
\label{eq:lip}
|\varphi(y_1)-\varphi(y_2)|\leq L\Vert y_1-y_2\Vert,y_1,y_2\in D,
\end{equation}
where \(L>0\) is the Lipschitz constant, and \(||\cdot||\) denotes the norm in \(\mathbb{R}^N\)
space.
\par
Usually, the objective function \(\varphi(y)\) is defined as a computational procedure,
according to which the value \(\varphi(y)\) can be calculated for any vector \(y\in D\)
(let us further call such a calculation \textit{a trial}). It is supposed that this procedure
is time-consuming.

\section{Review of Considered Optimization Methods}

\subsection{Parallel Algorithm of Global Search}
\subsubsection{Dimension Reduction with Space-Filling Curves}
Within the framework of the information-statistical global optimization theory,
the Peano space-filling curves (or \textit{evolvents}) \(y(x)\) mapping the interval \([0,1]\)
onto an \(N\)-dimensional hypercube \(D\) unambiguously are used for the dimensionality
reduction \cite{sergeyevStronginLera2013, strongin1978, strSergGO}.
\par
As a result of the reduction, the initial multidimensional global optimization
problem (\ref{eq:task}) is reduced to the following one-dimensional problem:
\begin{equation}
\label{eq:oneDimTask}
\varphi(y(x^*))=\min\{\varphi(y(x)):x\in [0,1]\}.
\end{equation}
\par
It is important to note that this dimensionality reduction scheme transforms the % minimized
Lipschitzian function from (\ref{eq:task}) to the corresponding one-dimensional
function \(\varphi(y(x))\), which satisfies the uniform H{\"o}lder condition, i. e.
\begin{equation}
\label{eq:holder}
|\varphi(y(x_1))-\varphi(y(x_2))|\leq H{|x_1-x_2|}^{\frac{1}{N}}, x_1,x_2\in[0,1],
\end{equation}
where the constant $H$ is defined by the relation \(H=2L\sqrt{N+3}\), \(L\) is the Lipschitz
constant from (\ref{eq:lip}), and \(N\) is the dimensionality of the optimization problem
(\ref{eq:task}).
\par
The algorithms for the numerical construction of the Peano curve approximations are
given in \cite{strSergGO}.

\par
The computational scheme obtained as a result of the dimensionality reduction consists of the
following:
\begin{itemize}
  \item The optimization algorithm performs the minimization of the reduced one-dimensional
  function \(\varphi(y(x))\) from (\ref{eq:oneDimTask}),
  \item After determining the next trial point \(x\), a multidimensional image \(y\) is calculated by
using the mapping \(y(x)\),
  \item The value of the initial multidimensional function \(\varphi(y)\) is calculated at the point
\(y\in D\),
  \item The calculated value \(z=\varphi(y)\) is used further as the value of the reduced one-
dimensional function \(\varphi(y(x))\) at the point \(x\).
\end{itemize}

\subsubsection{Algorithm of Global Search on Shared Memory}
\label{sub:ags}
Parallel optimization methods applied in Globalizer to solve the reduced problem
(\ref{eq:oneDimTask}) are based on the MAGS method, which can be presented as follows ---
see \cite{strongin1978}, \cite{strSergGO}.
\par
The initial iteration of the algorithm is performed at an arbitrary point \mbox{\(x^1\in(0,1)\)}.
Then, let us suppose that \(k\), \(k\ge 1\), optimization iterations have been completed already.
The selection of the trial point \(x^{k+1}\) for the next iteration is performed according to the
following rules.

\textit{Rule 1}. Renumber the points of the preceding trials by the lower indices in order of
increasing value of coordinates
$0=x_0<x_1<...<x_{k+1}=1$.

\textit{Rule 2}. Compute the characteristics \(R(i)\) for each interval \((x_{i-1},x_i),1\leq i\leq
k+1\).

\textit{Rule 3}. Determine the \(p\) intervals with the maximum characteristics $R(t_j)=\max_{1\leq i
\leq k+1}R(i),\: j=\overline{1,p}$.

\textit{Rule 4}. Execute new trials at points \(x^{k+j},\: j=\overline{1,p}\) located within intervals
with
maximum characteristics from the previous step
  $x^{k+j}=d(x_{t_j}),\: j=\overline{1,p}$.

The stopping condition, which terminated the trials, is defined by the inequality
$\rho_{t_j}<\varepsilon,\: j=\overline{1,p}$
for the intervals with maximum characteristics from Step 3 and \(\varepsilon >0\) is the
predefined accuracy of the optimization problem solution. If the stopping condition is not satisfied,
the index \(k\) is incremented by \(p\), and the new global optimization iteration is executed.

This method is employed in Globalizer to organize parallel computations on shared memory: each of
\(p\)
trials can be carried out on one of \(p\) local computation units.

The convergence conditions and exact formulas for decision rules $R(i)$ and $d(x)$ of the
described algorithm are given, for example, in \cite{strSergGO}.

\subsection{Sequential Methods}
\begin{itemize}
  \item \textbf{Algorithm of Global Search}. Sequential version of the method described in Section
\ref{sub:ags}.

  \item \textbf{Locally-biased Algorithm of Global Search (AGS\(l\))} \cite{indexMethod}. It's a
modification of
  the original AGS which make it more locally oriented by alternately using of two types of
characteristics in the Rule 2 from the Section \ref{sub:ags}.

  \item \textbf{Multi Level Single Linkage} \cite{Kan1987StochasticGO}. MLSL is an improved
multistart algorithm.
  It samples low-discrepancy starting points and does local optimizations from them. In contrast to
the dummy multistart schemes
  MLSL uses some clustering heuristics to avoid multiple local descents to already explored local
minima.

  \item \textbf{DIRECT} \cite{Jones2009}. The algorithm is deterministic and recursively divides
the search space and forms a tree of hyper-rectangles (boxes). DIRECT uses the objective function
values and the Lipschitz condition (\ref{eq:lip}) to estimate promising boxes.

  \item \textbf{Locally-biased DIRECT (DIRECT$l$)} \cite{Gablonsky2001}. It's a variation of
DIRECT which pays less attention to non-promising boxes and therefore
  has less exploration power: it can converge faster on problems with few local minima, but lost the
global one in complicated cases.

  \item \textbf{Dual Simulated Annealing} \cite{XIANG1997216}. This stochastic method is a
combination of the Classical Simulated Annealing and the Fast Simulated Annealing coupled to a
strategy for applying a local search on accepted locations. It converges much faster than both parent
algorithms, CSA and FSA.

  \item \textbf{Differential Evolution} \cite{Storn1997}. DE is an adaptation of the original genetic
algorithm to
  the continuous search domain.

  \item \textbf{Controlled Random Search} \cite{Price1983}. The CRS starts with a set of random
points and then defines
  the next trial point in relation to a simplex chosen randomly from a stored configuration of points.
CRS in not an
  evolutional algorithm, although stores something like population and performs transformation
resembling a mutation.

  \item \textbf{StoGO} \cite{Madsen1998}. StoGO is dividing the search space into smaller hyper-
rectangles via a branch-and-bound approach,
  and searching them by a local-search algorithm, optionally including some randomness.

\end{itemize}

All the mentioned algorithms are available in source codes as parts of wide-spread optimization packages (except of AGS and AGS\(l\), which are implemented in the Globalizer system).
DIRECT, DIRECT$l$, CRS, MLSL and StoGO are part of the NLOpt library \cite{nlopt}.
Differential Evolution and DSA can be found in
the latest version of the SciPy \cite{scipy} package for Python.

\section{Tools for Comparison of Global Optimization Algorithms}


The use of the sets of test problems with known solutions generated by some random mechanisms is
one of commonly accepted approaches to comparing the optimization algorithms
\cite{Beiranvand2017}. In the present work, we will use two generators of test problems generating
the problems of different nature \cite{grishaginClass, Gaviano2003}.

Let us denote the problem set obtained with the use of the first generator from \cite{grishaginClass}
as \(F_{GR}\). The mechanism of generation of the problems \(F_{GR}\) doesn't provide the
control of the problem complexity and of the number of local optima. However, the generated
functions are known to be the multiextremal ones essentially. Besides, the problems generated by
\(F_{GR}\) are the two-dimensional ones. In the present work, we will use 100 functions from the
class \(F_{GR}\) generated randomly.

The GKLS generator \cite{Gaviano2003} allows obtaining the problems of given dimensionality
with given number of extrema. Moreover, GKLS allows adjusting the complexity of the problems by
decreasing or increasing the size of the global minimum attractor. In
\cite{SergeyevKvasov2006} the parameters of the generator allowing generating the sets of 100
problems each of two levels of complexity (Simple and Hard) of the dimensionality equal to 2, 3, 4,
and 5 are given. Following the authors of the GKLS generator, we will use the parameters proposed
by them and, this way, add 800 more problems of various dimensionalities and complexity into the
test problem set.

Let us suppose a test problem to be solved if the optimization method executes the scheduled trial
\(y^k\) in a \(\delta\)-vicinity of the global minimum \(y^*\), i.e. $\left\|y^k-y^*\right\|\leq \delta
= 0.01\left\|b-a\right\|$, where \(a\) and \(b\) are the left and the right boundaries of the hypercube
from (\ref{eq:task}). If this relation is not fulfilled before the expiration of the limit of the number of
trials, the problem was considered to be unsolved. The limit of the number of trials was set
for each problem class according to the dimensionality and complexity (see Table \ref{tab:limits}).

\begin{table}
\begin{center}
\caption{Trials limits for the test problem classes}
  \begin{tabular}{|l|{c}|}
    \hline
  Problems class & Trials limit\\
  \hline
  \(F_{GR}\) & 5000 \\
  \hline
  GKLS 2d Simple & 8000 \\
  \hline
  GKLS 2d Hard & 9000 \\
  \hline
  GKLS 3d Simple & 15000 \\
  \hline
  GKLS 3d Hard & 25000 \\
  \hline
  GKLS 4d Simple & 150000 \\
  \hline
  GKLS 4d Hard & 250000 \\
  \hline
  GKLS 5d Simple & 350000 \\
  \hline
  GKLS 5d Hard & 600000 \\
  \hline
  \end{tabular}
  \label{tab:limits}
\end{center}
\end{table}

Let us consider the averaged number of trials executed to solve a single problem and the number of
solved problems as the characteristics of the optimization method on each class. The less the number
of trials, the faster the method converges to a solution, hence the less times it turns to a potentially
computation-costly procedure of computing the objective function. The number of solved problems
evidences the reliability of the method at given parameters on the class of test problems being
solved. In order to make independent the quantities featuring the reliability and the speed of convergence,
averaged number of trials always was calculated taking into account solved problems only.


\section{Results of Numerical Experiments}

The results of various algorithms on different problem classes depend on the adjustments of
algorithms directly. In most cases, the authors of software implementations are oriented onto the
problems of medium difficulty. In order to obtain a satisfactory result when solving the essentially
multiextremal problems, a correction of some parameters is required. When conducting the
comparison, the following parameters for the methods were employed:
\begin{itemize}
  \item in the AGS\(l\) method, the parameter of alternation the
global and local iterations was set to be equal to 5:1;
  \item in the DIRECT and DIRECT\(l\) methods, the parameter \(\epsilon=10^{-4}\);
  \item in the SDA method, the parameter \(visit=2.72\).
\end{itemize}

The rest parameters were varied subject to the problem class (see Table \ref{tab:params}).

\begin{table}
\begin{center}
\caption{Class-specific parameters of the optimization algorithms}
  \begin{tabular}{|l|{c}|{c}|{c}|}
    \hline
    & AGS, AGS\(l\) & CRS & DE\\
  \hline
  \(F_{GR}\) & \(r=3\) & popsize=150 & mutation=(1.1,1.9), popsize=60 \\
  \hline
  GKLS 2d Simple & \(r=4.6\) & popsize=200 & mutation=(1.1,1.9), popsize=60 \\
  \hline
  GKLS 2d Hard & \(r=6.5\) & popsize=400 & mutation=(1.1,1.9), popsize=60 \\
  \hline
  GKLS 3d Simple & \(r=3.7\) & popsize=1000 & mutation=(1.1,1.9), popsize=70 \\
  \hline
  GKLS 3d Hard & \(r=4.4\) & popsize=2000 & mutation=(1.1,1.9), popsize=80 \\
  \hline
  GKLS 4d Simple & \(r=4.7\) & popsize=8000 & mutation=(1.1,1.9), popsize=90 \\
  \hline
  GKLS 4d Hard & \(r=4.9\) & popsize=16000 & mutation=(1.1,1.9), popsize=100 \\
  \hline
  GKLS 5d Simple & \(r=4\) & popsize=25000 & mutation=(1.1,1.9), popsize=120 \\
  \hline
  GKLS 5d Hard & \(r=4\) & popsize=30000 & mutation=(1.1,1.9), popsize=140 \\
  \hline
\end{tabular}
  \label{tab:params}
\end{center}
\end{table}

The results of running the optimization methods on the considered problem classes are presented in
Tables \ref{tab:trials}, \ref{tab:solved}. The DIRECT and AGS\(l\) methods have demonstrated the
best convergence speed on all classes, at that AGS\(l\) inferior to DIRECT on the problems from the
Simple classes and has an advantage on the problems of the Hard classes. As one can see from Table
\ref{tab:solved}, the deterministic methods (AGS, AGS\(l\), DIRECT, and DIRECT\(l\)) were the
most reliable. Among the stochastic methods, MLSL and SDA have demonstrated the highest
reliability.

\begin{table}
\begin{center}
\caption{Averaged number of trials executed by sequential methods for solving the test
optimization problems}
\resizebox{\textwidth}{!}{%
  \begin{tabular}{|l|{c}|{c}|{c}|{c}|{c}|{c}|{c}|{c}|{c}|{c}|}
    \hline
    & AGS & AGS\(l\) & CRS & DIRECT & DIRECT\(l\) & MLSL & SDA & DE & StoGO \\
  \hline
  \(F_{GR}\)     & 193.1 &  \textbf{158.3} & 400.3 & 182.3 & 214.9 & 947.2 & 691.2 & 1257.3 &
1336.8 \\
  \hline
  GKLS 2d Simple &  254.9 & 217.6 & 510.6 & \textbf{189.0} & 255.2 & 556.8 & 356.3 & 952.2
& 1251.5 \\
  \hline
  GKLS 2d Hard   &  728.7 & \textbf{488.0} & 844.7 & 985.4 & 1126.7 & 1042.5 & 1637.9 &
1041.1 & 2532.2 \\
  \hline
  GKLS 3d Simple &  1372.1 & 1195.3 & 4145.8 & \textbf{973.6} & 1477.8 & 4609.2 & 2706.5 &
5956.94 & 3856.1 \\
  \hline
  GKLS 3d Hard   &  3636.1 & \textbf{1930.5} & 6787.0 & 2298.7 & 3553.3 & 5640.1 & 4708.4 &
6914.3 & 7843.2 \\
  \hline
  GKLS 4d Simple &  26654.1 & 11095.7 & 37436.8 & \textbf{7824.3} & 15994.1 & 41514.3 &
21417.9 & 19157.7 & 59895.4 \\
  \hline
  GKLS 4d Hard   &  54536.8 &  \textbf{23167.8} &  73779.3 &  23204.4 &  54489.9 &  80247.2
&  68815.5 &  27466.1 &  109328.1  \\
  \hline
  GKLS 5d Simple &  29810.0 & 11529.0 & 143575.0 & \textbf{7166.5} & 13970.5 & 52647.6 &
34255.3 & 73074.5 & 91580.4 \\
  \hline
  GKLS 5d Hard   &  113129.1 & 67652.7 & 165192.8 & \textbf{66327.4} & 164390.6 & 138766.2
& 116973.1 & 105496.9 & 155123.8 \\
  \hline
\end{tabular}}
  \label{tab:trials}
\end{center}
\end{table}

\begin{table}
\begin{center}
\caption{Number of test optimization problems solved by sequential methods}
  \begin{tabular}{|l|{c}|{c}|{c}|{c}|{c}|{c}|{c}|{c}|{c}|{c}|}
    \hline
    & AGS & AGS\(l\) & CRS & DIRECT & DIRECT\(l\) & MLSL & SDA & DE & StoGO \\
  \hline
  \(F_{GR}\)     &  100 & 100 & 76  & 100 & 100 & 97  & 96  & 96  & 67\\
  \hline
  GKLS 2d Simple &  100 & 100 & 85  & 100 & 100 & 100 & 100 & 98  & 90\\
  \hline
  GKLS 2d Hard   &  100 & 100 & 74  & 100 & 100 & 100 & 93  & 85  & 77 \\
  \hline
  GKLS 3d Simple &  100 & 97  & 75  & 100 & 100 & 100 & 89  & 86  & 44 \\
  \hline
  GKLS 3d Hard   &  100  & 99   & 72   & 100  & 99   & 100  & 88   & 77   & 43 \\
  \hline
  GKLS 4d Simple &  100 & 100 & 46  & 100 & 100 & 94  & 78  & 59  & 16 \\
  \hline
  GKLS 4d Hard   &  100 & 100 & 47  & 99  & 97  & 94  & 72  & 32  & 10  \\
  \hline
  GKLS 5d Simple &  100 & 100 & 68  & 100 & 100 & 98  & 100 & 77  & 9  \\
  \hline
  GKLS 5d Hard   &  97  & 99  & 42  & 100 & 90  & 79  & 84  & 48  & 8 \\
  \hline
  \end{tabular}
  \label{tab:solved}
\end{center}
\end{table}

\section{Conclusions}

In the present paper, several sequential and parallel global optimization algorithms were considered.
A comparison of efficiencies of these ones has been done on a set of test problems. The results allow
making the following conclusions:
\begin{itemize}
  \item AGS\(l\) method implemented in the Globalizer system has demonstrated the convergence
speed and reliability at the level of DIRECT and exceeds many other algorithms, the open-source
implementations of which are available;
  \item the stochastic optimization methods inferior to the deterministic ones in the convergence
speed and in reliability. It is manifested especially strongly on more complex multiextremal
problems;
  \item the parallel version of the Globalizer system demonstrates good speedup
when running on several nodes, on each of which a single objective function value per iterations is
computed. When solving the problems with fast computable objective functions and using multiple
threads on the nodes, the speedup for the Globalizer system degrade with increasing
number of nodes;
  \item the MIDACO system is the most suitable for simple global optimization problems with the
fast-computable objective functions. In this case, MIDACO is reliable enough and provides a linear
speedup with increasing number of nodes and threads executed on these ones in parallel.
\end{itemize}

% ---- Bibliography ----
%
\bibliographystyle{splncs04}
\bibliography{bibliography}{}
%

\end{document}
