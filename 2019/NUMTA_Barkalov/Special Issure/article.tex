\documentclass[review]{elsarticle}

\usepackage{lineno,hyperref}
%\usepackage[T1]{fontenc}
\modulolinenumbers[5]

\journal{Journal of \LaTeX\ Templates}

%%%%%%%%%%%%%%%%%%%%%%%
%% Elsevier bibliography styles
%%%%%%%%%%%%%%%%%%%%%%%
%% To change the style, put a % in front of the second line of the current style and
%% remove the % from the second line of the style you would like to use.
%%%%%%%%%%%%%%%%%%%%%%%

%% Numbered
%\bibliographystyle{model1-num-names}

%% Numbered without titles
%\bibliographystyle{model1a-num-names}

%% Harvard
%\bibliographystyle{model2-names.bst}\biboptions{authoryear}

%% Vancouver numbered
%\usepackage{numcompress}\bibliographystyle{model3-num-names}

%% Vancouver name/year
%\usepackage{numcompress}\bibliographystyle{model4-names}\biboptions{authoryear}

%% APA style
%\bibliographystyle{model5-names}\biboptions{authoryear}

%% AMA style
%\usepackage{numcompress}\bibliographystyle{model6-num-names}

%% `Elsevier LaTeX' style
\bibliographystyle{elsarticle-num}
%%%%%%%%%%%%%%%%%%%%%%%

\begin{document}

\begin{frontmatter}

\title{Global optimization method with dual Lipschitz constant estimates for problems with non-convex constraints}

%% Group authors per affiliation:
\author{Roman Strongin}
\author{Konstantin Barkalov}
\author{Semen Bevzuk}
%\address[mymainaddress]{Lobachevsky State University of Nizhni Novgorod, Nizhni Novgorod, Russia}
%\author[mysecondaryaddress]{Global Customer Service\corref{mycorrespondingauthor}}
%\cortext[mycorrespondingauthor]{Corresponding author}
%\ead{support@elsevier.com}
\address{Lobachevsky State University of Nizhni Novgorod, Nizhni Novgorod, Russia}

\begin{abstract}
In the present work, the constrained global optimization problems, in which the functions are of the “black box” type and satisfy the Lipschitz condition, are considered. The algorithms for solving the problems of this class require the use of the adequate estimates of the a priori unknown Lipschitz constants for the problem functions. A novel approach presented in this paper is based on a simultaneous use of two estimates of the Lipschitz constant: an overestimated and the underestimated ones. The upper estimate provides the global convergence whereas the lower one reduces the number of trials necessary to find the global optimizer with required precision. At that, the considered algorithm of solving the constrained problems doesn’t use the ideas of the penalty function method; each constraint of the problem is accounted for separately. The convergence conditions of the proposed algorithm are formulated according to corresponding theorem. The results of the numerical experiments on the series of the multiextremal problems with the non-convex constraints demonstrating the efficiency of the proposed scheme of dual Lipschitz constant estimates are presented.
\end{abstract}

\begin{keyword}
Global optimization, Multiextremal problems, Non-convex constraints, Lipschitz constant estimates 
\end{keyword}

\end{frontmatter}

\linenumbers


\section{Introduction}


\section{Index Method of Accounting for the Constraints and Dimensionality Reduction}

\begin{equation}\label{problem}
	\min{\left\{\varphi(y):y \in D, \; g_j(y)\leq 0, \; 1 \leq j \leq m\right\}},
\end{equation}

\begin{equation}\label{D}
	D=\left\{y\in R^N: a_i\leq y_i \leq b_i, 1\leq i \leq N\right\}.
\end{equation}

\begin{equation}\label{lipschitz_condition}
	\left|g_j(y')-g_j (y'')\right| \leq L_j \|y'-y''\|, \; y', y''\in D, \; 1\leq j \leq m+1.
\end{equation}

\begin{equation}\label{epsilon_reserved_solution}
	\varphi(y_{\epsilon})=\min{\left\{\varphi(y):y \in D, \; g_j(y)\leq -{\epsilon}_j, \; 1 \leq j \leq m\right\}},
\end{equation}

\begin{equation}\label{Y_epsilon}
	Y_{\epsilon}=\left\{ y \in D, \; g_j(y) \leq 0, \; 1 \leq j \leq m, \; \varphi(y) \leq \varphi(y_{\epsilon}) \right\}.
\end{equation}

\begin{equation}\label{D_sets}
	D_1 = D, \; D_{j+1} = \left\{ y \in D_j: \; g_j(y) \leq 0 \right\}, \; 1 \leq j \leq m.
\end{equation}

\begin{equation}\label{one_dimensional_problem}
	\varphi(y(x^\ast))=\min \left\{\varphi(y(x)): x \in [0,1], \; g_j(y(x))\leq 0, \; 1 \leq j \leq m\right\},
\end{equation}

\[
	\left|g_j(y(x'))-g_j(y(x''))\right| \leq K_j \left|x'-x'' \right|^{1/N}, \; x', x''\in [0,1], \; 1\leq j \leq m+1,
\]

\begin{equation}\label{Q_intervals}
	Q_1=[0,1], \; Q_{j+1}=\left\{x \in Q_j : g_j(y(x)) \leq 0 \right\}, \; 1 \leq j \leq m.
\end{equation}

\[
	g_j(y(x)) \leq 0, \; 1 \leq j \leq \nu-1, \; g_{\nu}(y(x))>0,
\]

\begin{equation}\label{trial_result}
	x^k, \; z^k = g_{\nu}\left( y(x^k) \right), \; \nu = \nu(x^k),
\end{equation}

\begin{equation}\label{reduction_problem}
	\psi(x^*)=\min \left\{\psi(x): x \in [0,1] \right\},
\end{equation}

\begin{eqnarray}
	\psi(x)=\frac{g_{\nu}(y(x))}{K_{\nu}} - 
	\left\{
   \begin{array}{lr}
     0, & \nu \leq m,\\
     \frac{\varphi^*}{K_{m+1}}, & \nu = m + 1.
   \end{array}
	\right.
\end{eqnarray}

\[
	\psi(x)=\frac{\varphi(y(x))-\varphi^*}{K_{m+1}}
\]

\section{Index Algorithm of Global Search}

\[
	0=x_0<x_1<\dots <x_k<x_{k+1}=1,
\]

\[
	I_\nu =\left\{i:1 \leq i \leq k, \; \nu=\nu(x_i) \right\}, \; 1 \leq \nu \leq m+1,
\]

\[
	M=\max\left\{\nu=\nu(x_i), \; 1 \leq i \leq k \right \}.
\]

\begin{equation}\label{current_lower_bounds}
	\mu_{\nu} = \max\left\{ \frac{\left|z_i-z_j\right|}{ (x_i - x_j)^{1/N} }, \; i,j \in I_\nu, \; i>j \right\},
\end{equation}

\begin{equation}\label{z_estimates}
	z_\nu^\ast = \left\{
   \begin{array}{lr}
     -\epsilon_\nu, & \nu < M,\\
     \min\{ z_i: i\in I_\nu \}, & \nu = M,
   \end{array}
	\right.
\end{equation}

\begin{equation}\label{R_1}
	R(i)=\Delta_i+\frac{(z_i-z_{i-1})^2}{r_\nu^2 \mu_\nu^2\Delta_i}-2\frac{z_i+z_{i-1}-2z_\nu^\ast}{r_\nu \mu_\nu}, \;  \nu=\nu(x_i)=\nu(x_{i-1}),
\end{equation}
\begin{equation}\label{R_2}
	R(i)=2\Delta_i-4\frac{z_i-z_\nu^\ast}{r_\nu \mu_\nu}, \; \nu=\nu(x_i)>\nu(x_{i-1}),
\end{equation}
\begin{equation}\label{R_3}
R(i)=2\Delta_i-4\frac{z_{i-1}-z_\nu^\ast}{r_\nu \mu_\nu}, \; \nu=\nu(x_{i-1})>\nu(x_i),
\end{equation}

\begin{equation}\label{MaxR}
	R(t)=\max{\left\{R(i): 1 \leq i \leq k+1\right\}}.
\end{equation}

\[
	x^{k+1} = \frac{x_t + x_{t-1}}{2}, \; \nu(x_{t-1}) \neq \nu(x_t).
\]

\[
	x^{k+1} = \frac{x_t+x_{t-1}}{2} + \frac{\mathrm{sign}(z_t-z_{t-1})}{2r_\nu}\left[\frac{\left|z_t-z_{t-1}\right|}{\mu_\nu}\right]^N, \; \nu=\nu(x_{t-1})=\nu(x_t).
\]

\[
	g_j \left( y(x) \right) = G_j \left( y(x) \right), \; x \in Q_j, \; 1 \leq j \leq m+1,
\]

\begin{equation}\label{theorem_inequalities}
	r_{\nu}\mu_{\nu} > 2^{3-\frac{1}{N}}L_{\nu}\sqrt{N+3}, \; 1 \leq \nu \leq m+1, 
\end{equation}

\[
	\varphi(\bar y) = \inf\left\{\varphi(y^q):g_j(y^q) \leq 0, 1 \leq j \leq m, q = 1,2,\ldots \right\} \leq \varphi(y_{\epsilon}).
\]

\begin{equation}\label{epsilon_nu}
	\epsilon_{\nu} = \mu_{\nu}\delta, \; 1 \leq \nu \leq m, 
\end{equation}


\section{Index Algorithm with Dual Lipschitz Constant Estimates}

\begin{equation}\label{estimates_Lipschitz_constants}
	\frac{r_{\nu}\mu_{\nu}}{2^{3-\frac{1}{N}}L_{\nu}\sqrt{N+3}}, \; 1 \leq \nu \leq m+1, 
\end{equation}

\begin{equation}\label{R_glob_1}
	R(i)=R_{glob}(i)=R_{loc}(i)=2\Delta_i>0. 
\end{equation}

\[
	z_i+z_{i-1}-2z_M^* = |z_i-z_{i-1}|\leq \mu_M\Delta_i
\]

\begin{eqnarray}
	R(i) & = & \Delta_i + \frac{(z_i-z_{i-1})^2}{r_M^2\mu_M^2\Delta_i} - 2\frac{z_i+z_{i-1}-2z_M^*}{r_M\mu_M} \geq \Delta_i + \frac{(z_i-z_{i-1})^2}{r_M^2\mu_M^2\Delta_i} - 2\frac{\mu_M\Delta_i}{r_M\mu_M} \nonumber \\
	& \geq & \Delta_i + \frac{(z_i-z_{i-1})^2}{r_M^2\mu_M^2\Delta_i} - 2\frac{\Delta_i}{r_M} = \Delta_i + \Delta_i\frac{(z_i-z_{i-1})^2/\Delta_i^2}{r_M^2\mu_M^2\Delta_i^2} - 2\frac{\Delta_i}{r_M}
\nonumber \\
	& \geq & \Delta_i + \frac{\Delta_i}{r_M^2} - 2\frac{\Delta_i}{r_M} = \Delta_i \left( 1-\frac{1}{r_M} \right)^2>0  
	\nonumber
\end{eqnarray}

\begin{equation}\label{R_glob_greater_R_loc}
	R_{glob}(i)=\Delta_i\left(1-\frac{1}{r_M^{glob}} \right)^2 > \Delta_i\left(1-\frac{1}{r_M^{loc}} \right)^2 = R_{loc}(i)
\end{equation}

\begin{equation}\label{R_glob_equal_R_loc}
	R_{glob}(i) = \rho R_{loc}(i)
\end{equation}

\begin{equation}\label{R_max}
	R(i) = \max{\left\{ R_{glob}(i), \rho R_{loc}(i) \right\}}
\end{equation}

\begin{equation}\label{R_max_t}
	R(t) = \max{\left\{ R(i): 1 \leq i \leq k+1\right\}}
\end{equation}

\[
	x^{k+1} = \frac{x_t + x_{t-1}}{2}, \; \nu(x_{t-1}) \neq \nu(x_t).
\]

\[
	x^{k+1} = \frac{x_t+x_{t-1}}{2} + \frac{\mathrm{sign}(z_t-z_{t-1})}{2r_\nu}\left[\frac{\left|z_t-z_{t-1}\right|}{\mu_\nu}\right]^N, \; \nu=\nu(x_{t-1})=\nu(x_t).
\]

\begin{eqnarray}
	\varphi(y_1, y_2)=-1.5y_1^2\exp{\left\{1-y_1^2-20.25(y_1-y_2)^2\right\}}- \nonumber \\
	-\left(0.5(y_1-1)(y_2-1)\right)^4\exp{\left\{2-\left(0.5(y_1-1)\right)^4-(y_2-1)^4\right\}}
	\nonumber
\end{eqnarray}

\begin{eqnarray}
	g_1(y_1, y_2) &=& 0.01 \left( (y_1-2.2)^2+(y_2-1.2)^2-2.25 \right) \leq 0, \nonumber \\
	g_2(y_1, y_2) &=& 100 \left(1-(y_1-2)^2/1.44+(0.5y_2)^2 \right) \leq 0, \nonumber \\
	g_3(y_1, y_2) &=& 10 \left( y_2 - 1.5 - 1.5 \sin{\left( 6.283(y_1-1.75) \right)}\right) \leq 0
	\nonumber
\end{eqnarray}

\begin{equation}\label{phi_bar_y}
	\varphi(\bar y) = \inf\left\{\varphi(y^k):g_j(y^k) \leq 0, 1 \leq j \leq m, k = 1,2,\ldots \right\} \leq \varphi(y_{\epsilon}).
\end{equation}

\[
	R_{glob}(t) < \rho R_{loc}(t)
\]

\[
	z_j=g_{\nu}\left( y(x_j) \right) \leq g_{\nu}\left( y(\bar x) \right) + 2L_{\nu}\sqrt{N+3}(x_j-\bar x)^{1/N}, \; \nu=\nu(x_j)
\]

\[
	z_{j-1}=g_{\nu}\left( y(x_{j-1}) \right) \leq g_{\nu}\left( y(\bar x) \right) + 2L_{\nu}\sqrt{N+3}(\bar x - x_{j-1})^{1/N}, \; \nu=\nu(x_{j-1})
\]

\[
	g_{\nu}\left( y(\bar x) \right) \leq -\epsilon_{\nu}, \; 1\leq\nu\leq m.
\]

\[
	g_{m+1}\left( y(\bar x) \right) \leq z_{m+1}^*
\]

\begin{eqnarray}
	R(j) &=& \Delta_i + \frac{(z_j-z_{j-1})^2}{r_{\nu}^2\mu_{\nu}^2\Delta_i} - 2\frac{z_j+z_{j-1}-2z_{\nu}^*}{r_{\nu}\mu_{\nu}}  \nonumber \\
	&\geq& \Delta_i + 4\frac{z_{\nu}^*-\left( g_{\nu}\left( y(\bar x) \right)+L_{\nu}\sqrt{N+3}\left( (x_j-\bar x)^{\frac{1}{N}}+(\bar x - x_{j-1})^{\frac{1}{N}} \right)\right)}{r_{\nu}\mu_{\nu}} \nonumber \\
	&=& \Delta_i-4\frac{L_{\nu}\sqrt{N+3}\left( (x_j-\bar x)^{\frac{1}{N}}+(\bar x - x_{j-1})^{\frac{1}{N}} \right)}{r_{\nu}\mu_{\nu}}+4\frac{z_{\nu}^*-g_{\nu}\left( y(\bar x) \right)}{r_{\nu}\mu_{\nu}} \nonumber \\
	&=& \Delta_i-4\frac{L_{\nu}\sqrt{N+3}\left( \alpha^{\frac{1}{N}}+(1-\alpha)^{\frac{1}{N}} \right)}{r_{\nu}\mu_{\nu}}+4\frac{z_{\nu}^*-g_{\nu}\left( y(\bar x) \right)}{r_{\nu}\mu_{\nu}} \nonumber \\
	&\geq& \Delta_i\left(1-4\frac{L_{\nu}\sqrt{N+3}\max_{0\leq\alpha\leq1} {\left( \alpha^{\frac{1}{N}}+(1-\alpha)^{\frac{1}{N}} \right)}}{r_{\nu}\mu_{\nu}} \right)+4\frac{z_{\nu}^*-g_{\nu}\left( y(\bar x) \right)}{r_{\nu}\mu_{\nu}} \nonumber \\
	&=& \Delta_i\left(1-4\frac{2^{3-1/N}L_{\nu}\sqrt{N+3}}{r_{\nu}\mu_{\nu}} \right)+4\frac{z_{\nu}^*-g_{\nu}\left( y(\bar x) \right)}{r_{\nu}\mu_{\nu}} \geq 0
	\nonumber
\end{eqnarray}

\begin{eqnarray}
	R(j) &=& 2\Delta_i - 4\frac{z_j-z_{\nu}^*}{r_{\nu}\mu_{\nu}} \geq 2\Delta_i+4\frac{z_{\nu}^*-\left( g_{\nu}\left(y(\bar x)\right)+2L_{\nu}\sqrt{N+3}(x_j-\bar x)^{\frac{1}{N}} \right)}{r_{\nu}\mu_{\nu}} \nonumber \\
	&=& 2\Delta_i - 8\frac{L_{\nu}\sqrt{N+3}(x_j-\bar x)^{\frac{1}{N}}}{r_{\nu}\mu_{\nu}}+4\frac{z_{\nu}^*-g_{\nu}\left( y(\bar x) \right)}{r_{\nu}\mu_{\nu}} \nonumber \\
	&\geq& 2\Delta_i\left( 1-4\frac{L_{\nu}\sqrt{N+3}}{r_{\nu}\mu_{\nu}} \right) + 4\frac{z_{\nu}^*-g_{\nu}\left( y(\bar x) \right)}{r_{\nu}\mu_{\nu}} > 0
	\nonumber
\end{eqnarray}

\begin{equation}
	R_{glob}(j) > \rho R_{loc}(t)
\end{equation}

\section{TEST}

There are various bibliography styles available. You can select the style of your choice in the preamble of this document. These styles are Elsevier styles based on standard styles like Harvard and Vancouver. Please use Bib\TeX\ to generate your bibliography and include DOIs whenever available.

Here are two sample references: 

\cite{Evtushenko1971, Piyavskii1972, Shubert1972, Evtushenko2009, Evtushenko2013, Strongin2000, Sergeyev2013, Pinter1996, Jones2009, Wood1991, Meewella1988, Mladineo1986, Vaz2009, Stripinis2019, Paulavicius2016, Pillo2012, Pillo2016, Barkalov2017_1, Barkalov2017_2, Sergeyev2006, Zilinskas2008, Sovrasov2019, Kvasov2003, Sergeyev2010,Sergeyev2016, Horst1996, Gablonsky2001, Jones1993, Gaviano2003, Barkalov2018, Paulavicius2014, Sergeyev2015, Strongin2018,Gergel2017_2,Gergel2019}.


\bibliography{bibliography}

\end{document}