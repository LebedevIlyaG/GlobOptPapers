% This is samplepaper.tex, a sample chapter demonstrating the
% LLNCS macro package for Springer Computer Science proceedings;
% Version 2.20 of 2017/10/04
%
\documentclass[runningheads]{llncs}
%
\usepackage{graphicx}

% If you use the hyperref package, please uncomment the following line
% to display URLs in blue roman font according to Springer's eBook style:
% \renewcommand\UrlFont{\color{blue}\rmfamily}

\begin{document}
%
\title{ Generalized scheme for dimensionality reduction based on nested 
optimization and Peano curve for global optimization problems
\thanks{This study was supported by the Russian Science Foundation, project 
No.\,16-11-10150.}}
%
%\titlerunning{Abbreviated paper title}
% If the paper title is too long for the running head, you can set 
% an abbreviated paper title here
%
\author{Konstantin Barkalov \orcidID{0000-0001-5273-2471} \and
Ilia Lebedev\orcidID{0000-0002-8736-0652}}
%
\authorrunning{K. Barkalov, I. Lebedev}
% First names are abbreviated in the running head.
% If there are more than two authors, 'et al.' is used.
%
\institute{Lobachevsky State University of Nizhny Novgorod, Nizhny Novgorod, 
Russia \email{konstantin.barkalov@itmm.unn.ru}}
%
\maketitle              % typeset the header of the contribution
%
\begin{abstract}
The abstract should briefly summarize the contents of the paper in
150--250 words.

\keywords{Dimensionality reduction \and Global optimization \and 
Multiextremal functions.}
\end{abstract}
%
%
%
\section{Introduction}

\section{The core global search algorithm}

According to the algorithm, the first two trials are executed at the ends of 
the interval  $[a,b]$, i.e., $x^0=a,\;x^1=b$. The function values $z^0=\varphi
(x^0),\;z^1=\varphi(x^1)$  are computed and the number $k$ is set to 1. In 
order to select the point of a new trial $x^{k+1}, k\geq 1,$  it is necessary 
to perform the following steps.

\textbf{Step 1.} Renumber by subscripts (beginning from zero) the points $x^i,
\:0\leq i\leq k$, of the previous trials in increasing order, i.e.,
\begin{displaymath}
a=x_0<x_1<\ldots <x_{k}=b.
\end{displaymath} 
Juxtapose to the points $x_i,\:1\leq i\leq k$,  the function values $z_i=
\varphi(x_i),\:0\leq i\leq k$.

\textbf{Step 2.} Compute the maximum absolute value of the first divided 
differences 
\begin{displaymath}
\mu=\max_{1\leq i\leq k}\frac{\left|z_i-z_{i-1}\right|}{\Delta_i}
\end{displaymath}
where $\Delta_i = x_i-x_{i-1}$. If the above formula yields a zero value, 
assume that $\mu = 1$.


\textbf{Step 3.} For each interval $(x_{i-1},x_i),1\leq i\leq k$,  calculate 
the characteristic
\begin{displaymath}
R(i)=r\mu(x_i-x_{i-1})+\frac{(z_i-z_{i-1})^2}{r\mu(x_i-x_{i-1})}-2(z_i+z_{i-1}
),
\end{displaymath} 
where $r>1$ is a predefined parameter of the method. 

\textbf{Step 4.} Find the interval $(x_{t-1},x_t)$ with the maximum 
characteristic
\begin{equation}\label{MaxR}
R(t)=\max_{1\leq i\leq {k}}R(i).
\end{equation}  

\textbf{Step 5.} Execute the new trial at the point 
\begin{displaymath}
x^{k+1}=\frac{1}{2}(x_{t-1}+x_t) - \frac{z_t-z_{t-1}}{2r\mu}.
\end{displaymath}

The algorithm terminates if the condition $\Delta_t<\epsilon$ is satisfied; 
here $t$ is from (\ref{MaxR}) and $\epsilon>0$ is the preset accuracy. For 
estimation of the global solution, values
\[
z_k^\ast=\min_{0\leq i \leq k}\varphi(x^i), \ x_k^\ast=\arg \min_{0\leq i \leq
 k}\varphi(x^i).
\]
are selected.

\section{Dimensionality reduction}
\subsection{Dimensionality reduction using Peano-type space-filling curves}


\subsection{Nested optimization schemes}

The nested optimization scheme of dimensionality reduction is based on the 
well-known relation (see, e.g., [])
\begin{equation}\label{nested}
\min_{y \in D}\varphi(y) = \min_{a_1\leq y_1 \leq b_1}\min_{a_2\leq y_2 \leq 
b_2}...\min_{a_N\leq y_N \leq b_N}\varphi(y),
\end{equation}
which allows replacing the solving of multidimensional problem (\ref{multi_
problem}) by solving a family of one-dimensional subproblems related to each 
other recursively.

In order to describe the scheme let us introduce a set of reduced functions 
as follows:
\begin{equation}\label{nested_N}
\varphi^N(y_1,...,y_N) = \varphi(y_1,...,y_N),
\end{equation}
\begin{equation}\label{nested_i}
\varphi^i(y_1,...,y_i) = \min_{a_{i+1}\leq y_{i+1}\leq b_{i+1}} \varphi^{i+1}(
y_1,...,y_i,y_{i+1}), 1\leq i\leq N-1.
\end{equation}

Then, according to relation (\ref{nested}), solving of multidimensional 
problem (\ref{multi_problem}) is reduced to solving a one-dimensional problem 
\begin{equation}\label{nested_1}
%\varphi^1(y_1^\ast) = 
\min_{a_1\leq y_1\leq b_1}\varphi^1(y_1).
\end{equation}
But in order to evaluate the function $\varphi^1$ at a fixed point $y_1$ it 
is necessary to solve the one-dimensional problem of the second level
\begin{equation}
\varphi^1(y_1) = \min_{a_2\leq y_2\leq b_2}\varphi^2(y_1,y_2),
\end{equation}
and so on up to the univariate minimization of the function $\varphi^N(y_1
,...,y_N)$ with fixed coordinates $y_1,...,y_{N-1}$ at the $N$-th level of 
recursion.


%
% ---- Bibliography ----
%
% BibTeX users should specify bibliography style 'splncs04'.
% References will then be sorted and formatted in the correct style.
%
 \bibliographystyle{splncs04}
 \bibliography{bibliography}


\end{document}
