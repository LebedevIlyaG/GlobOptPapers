\documentclass[11pt, oneside, a4paper]{article}
%\usepackage[cp1251]{inputenc} % кодировка
\usepackage[utf8]{inputenc} % кодировка
\usepackage[english, russian]{babel} % Русские и английские переносы
\usepackage{graphicx}          % для включения графических изображений
\usepackage{cite}              % для корректного оформления литературы
\usepackage{enumitem}
\usepackage{pavt-ru}

\usepackage{amssymb}
\usepackage{amsmath}
\usepackage{cite}
%\usepackage{subfig}
%\usepackage{caption}
%\captionsetup[table]{skip=9pt}
%\captionsetup{labelfont=bf}

\DeclareMathOperator*{\argmax}{arg\,max}
\DeclareMathOperator*{\argmin}{arg\,min}
\DeclareMathOperator{\sign}{sign}
\newtheorem{theorem}{Теорема}


\begin{document}

% \title - название статьи
% \authors - список авторов

\title{Параллельный алгоритм для получения равномерного приближения решений множества задач глобальной оптимизации с нелинейными ограничениями
\footnote{Исследование выполнено при поддержке РНФ, проект №\,16-11-10150.}}

\authors{В.В.~Соврасов}
\organizations{Нижегородский государственный университет им. Н.И. Лобачевского}

% Аннотация заключается в окружение abstract
\begin{abstract}
В данной работе рассматривается построение параллельной версии алгоритма глобальной оптимизации, решающего одновременно множество задач с нелинейными ограничениями и получающего при этом равномерные оценки решений на этом множестве. Последнее свойство позволяет наиболее оптимально распределять вычислительные ресурсы, т.к. в процессе работы алгоритма погрешности численного решения во всех задачах убывают примерно с одинаковой скоростью. Подобные серии задач возникают, если задача глобальной оптимизации имеет дискретный параметр или при решении задачи многокритериальной оптимизации методом свётртки критериев. Рассматриваемый алгоритм использует отображения типа кривой Пеано для редукции многомерных задач оптимизации к одномерным. Эффективность реализованного алгоритма протестирована на наборах искусственно сгенерированных задач глобальной оптимизации.
\end{abstract}

\keywords{Computational mathematics / Вычислительная математика, Parallel and distributed computing technologies / Технологии параллельных и распределенных вычислений}

% \section{название} - заголовок раздела первого уровня
% \subsection{название} - заголовок раздела второго уровня
% \subsubsection{название} - заголовок раздела третьего уровня
% Не используйте уровень вложенности заголовков больше трех!
% Каждый абзац текста в статье начинается командой \par или пустой
% строкой.

\section{Введение}

Нелинейная глобальная оптимизация невыпуклых функций традиционно считается одной из самых трудных
задач математического программирования. Отыскание глобального минимума функции от нескольких переменных
зачастую оказывается сложнее, чем локальная оптимизация в тысячемерном пространстве. Для последней может оказаться достаточно
применения простейшего метода градиентного спуска, в то время как чтобы \textit{гаранитрованно} отыскать глобальный оптимум методам
оптимизации приходится накапливать информацию о поведении целевой функции во всей области поиска \cite{Jones2009,Paulavicius2011,Evtushenko2013,Strongin2000}. Решение серии таких задач при ограниченных вычислительных
ресурсах является ещё более сложной проблемой: помимо поиска глобального экстремума необходимо
распределять вычислительные ресурсы так, чтобы сразу во всех решаемых задачах положение глобального
экстремума было оценено примерно с одинаковым качеством. Обычно серию из \(q\) задач решают либо последовательно, либо
параллельно порциями по \(p,\:p<<q\) задач, где \(p\) --- количество параллельных вычислительных усройств.
Такой подход ведёт к тому, что в каждый момент времени до окончания вычислений
остаются задачи, в которых оценка глобального оптимума не получена вообще, в то время, как в задачах из начала
списка оптимум может быть оценён даже с избыточной точностью.

В данной работе рассматривается обобщение ранее разработанного в ННГУ им. Н. И. Лобачевского
параллельного метода глобальной оптимизации для одновременного решения множества задач \cite{BarkalovStrongin2018} на
случай задач с нелинейными ограничениями. Для учёта ограничений используется индексная схема \cite{Strongin2000},
позволяющая работать с частично вычислимыми целевым функциями и обладающая экономичностью,
сравнимой с другими подходами \cite{BarkalovLebedev2017}. Эффективность реализованного
алгоритма показана на примере решения множеств задач, сгенерированных специализированным
механизмом, порождающим наборы задач заданной размерности с заданным количеством нелинейных ограничений \cite{GergelBarkalov2019}.
Кроме искуственно сгенерированных задач, рассматриваемый метод протестирован также
на множестве задач, возникающем при решении задачи многокритериальной оптимизации
с нелинейными ограничениями методом свёртки критериев \cite{Ehrgott2005}.


\section{Постановка задачи глобальной оптимизации}
В рамках данной работы будем рассмативать следующую постановку задачи глобальной
оптимизации: найти глобальный минимум \(N\)-мерной функции \(\varphi(y)\) в гиперинтервале
\(D=\{y\in \mathbf{R}^N:a_i\leqslant x_i\leqslant{b_i}, 1\leqslant{i}\leqslant{N}\}\).
Для построения оценки глобального минимума по конечному количеству вычислений
значения функции требуется, чтобы скорость изменения \(\varphi(y)\) в \(D\) была ограничена.
В качестве такого ограничения как правило принимается условие Липшица.
\begin{displaymath}
\label{task}
\varphi(y^*)=\min\{\varphi(y):y\in D\}
\end{displaymath}
\begin{displaymath}
\label{lip}
|\varphi(y_1)-\varphi(y_2)|\leqslant L\Vert y_1-y_2\Vert,y_1,y_2\in D,0<L<\infty
\end{displaymath}

Существуют различные методы, решающие рассмотренную многомерную задачу напрямую \cite{SergeyevKvasov2017, Jones2009},
а также эффективные методы решения одномерных задач \cite{Norkin1992, Strongin2000}. В данной работе рассматривается одномерный метод,
который применяется совместно со схемой редукции размерности.
Классической схемой редукции размерности исходной задачи для алгоритмов глобальной оптимизации является
использование разверток --- кривых, заполняющих пространство \cite{Sergeyev2013}.
\begin{equation}
\label{cube}
\lbrace y\in \mathbf{R}^N:-2^{-1}\leqslant y_i\leqslant 2^{-1},1\leqslant i\leqslant N\rbrace=\{y(x):0\leqslant x\leqslant 1\}
\end{equation}

Отображение вида (\ref{cube}) позволяет свести задачу в многомерном пространстве к решению
одномерной ценой ухудшения её свойств. В частности, одномерная функция \(\varphi(y(x))\)
является не Липшицевой, а Гёльдеровой:
\begin{displaymath}
\label{holder}
|\varphi(y(x_1))-\varphi(y(x_2))|\leqslant H{|x_1-x_2|}^{\frac{1}{N}},x_1,x_2\in[0;1],
\end{displaymath}
где константа Гельдера \(H\) связана с константой Липшица \(L\) соотношением
\begin{displaymath}
H=4Ld\sqrt{N},d=\max\{b_i-a_i:1\leqslant i\leqslant N\}.
\end{displaymath}

Область \(D\) также может быть задана с помощью функциональных ограничений, что
значительно усложняет задачу.
Постановка задачи глобальной оптимизации в этом случае будет иметь следующий вид:
\begin{equation}
  \label{eq:constrained_problem}
  \varphi(y^*)=\min\{\varphi(y):g_j(y)\leqslant 0, 1\leqslant j\leqslant m\}
\end{equation}
Обозначим \(g_{m+1}(y)=\varphi(y)\). Далее будем предполагать, что все функции \(g_k(y),1\leqslant k \leqslant m+1\)
удовлетворяют условию Липшица в некотором гиперинтервале, включающем \(D\).

Далее будем интересоваться решением серии из \(q\) задач вида (\ref{eq:constrained_problem}):
\begin{equation}
  \label{eq:many_problems}
  \min\left\{\varphi_1(y), y\in D_1 \right\}, \min\left\{\varphi_2(y), y\in D_2\right\},..., \min\left\{\varphi_q(y), y\in D_q\right\}.
\end{equation}

\section{Описание метода глобальной оптимизации}

Принимая во внимание схему редукции размерности (\ref{cube}), будем при описании метода считать, что
требуется найти глобальный минимум функции \(\varphi(x), x\in[0;1]\),
удовлетворяющей условию Гёльдера, при ограничениях \(g_j(x)\), также
удовлетворяющих этому условию на интервале \([0;1]\).

Рассматриваемый индексный алгоритм глобального поиска (ИАГП) для решения
одномерной задачи (\ref{eq:constrained_problem}) предполагает построение последовательности
точек \(x_k\), в которых вычисляются значения минимизируемой функции или ограничений \(z_k = g_s(x_k)\).
Для учёта последних используется индексная схема \cite{Strongin2000}. Пусть \(Q_0=[0;1]\). Ограничение, имеющее номер
 \(j\), выполняется во всех точках области
\begin{displaymath}
  Q_j=\left\{x\in [0;1]:g_j(x)\leq 0\right\},
\end{displaymath}
которая называется допустимой для этого ограничения. При этом допустимая область \(D\)
исходной задачи определяется равенством: \(D=\cap _{j=0}^{m}Q_{j}\).
Испытание в точке \(x\in [0;1]\) состоит в последовательном вычислении значений
величин \(g_{1}(x),...,g_{\nu }(x)\), где значение индекса \(\nu\) определяется условиями:
\(x\in Q_{j},0\leqslant j<\nu ,x\notin Q_{\nu }\). Выявление первого нарушенного ограничения
прерывает испытание в точке \(x\). В случае, когда точка \(x\)  допустима, т. е.
\(x\in D\) испытание включает в себя вычисление всех функций задачи. При этом значение
индекса принимается равным величине \(\nu =m+1\). Пара \(\nu =\nu (x),z=g_{\nu }(x)\),
где индекс \(\nu\) лежит в границах \(1\leqslant \nu \leqslant m+1\), называется результатом
испытания в точке \(x\).

Такой подход к проведению испытаний позволяет свести исходную задачу с функциональными
ограничениями к безусловной задаче минимизации разрывной функции:

\begin{displaymath}
  \begin{array}{lr}
    \psi (x^{*})=\min_{x\in [0;1]}\psi (x), \\
    \psi (x)={\begin{cases}g_{\nu }(x)/H_{\nu }&\nu <M\\(g_{M}(x)-g_{M}^{*})/H_{M}&\nu =M\end{cases}}
  \end{array}
\end{displaymath}

Здесь \(M=\max_{}^{}\left\{\nu (x):x\in [0;1]\right\}\), а \(g_{M}^{*}=\min _{}^{}\left\{g_{M}(x):x\in \cap _{i=0}^{M-1}Q_{i}\right\}\).
В силу определения числа \(M\), задача отыскания \(g_{M}^{*}\)
всегда имеет решение, а если \(M=m+1\), то \(g_{M}^{*}=\varphi(x^{*})\).
Дуги функции \(\psi (x)\) гельдеровы на множествах \(\cap _{i=0}^{j}Q_{i},0\leq j\leq M-1\)
с константой 1, а сама \(\psi (x)\) может иметь разрывы первого рода на границах этих множеств.
Несмотря на то, что значения констант Гёльдера \(H_k\) и величина \(g_{M}^{*}\) заранее неизвестны,
они могут быть оценены в процессе решения задачи.

Множество троек \(\{(x_k,\nu_k,z_k)\}, 1\leqslant k\leqslant n\) составляет поисковую информацию,
накопленную методом после проведения \(n\) шагов.

На первой итерации метода испытание проводится в произвольной внутренней точке \(x_1\)
интервала \([0;1]\). Индексы точек 0 и 1 считаются нулевыми, значения \(z\) в
них не определены. Пусть выполнено \(k\geqslant 1\) итераций метода,
в процессе которых были проведены испытания в \(k\) точках \(x_i, 1\leqslant i\leqslant k\).
Тогда точкa \(x^{k+1}\) поисковых испытаний следующей \((k+1)\)-ой
итерации определяются в соответствии с правилами:

Шаг 1. Перенумеровать точки множества \(X_k=\{x^1,\dotsc,x^k\}\cup\{0\}\cup\{1\}\),
которое включает в себя граничные точки интервала \([0;1]\), а также точки предшествующих
испытаний, нижними индексами в порядке увеличения значений координаты, т.е.
\begin{displaymath}
0=x_0<x_1<\dotsc<x_{k+1}=1
\end{displaymath}
и сопоставить им значения \(z_{i}=g_{\nu }(x_{i}),\nu =\nu (x_{i}),i={\overline {1,k}}\).

Шаг 2. Для каждого целого числа \(\nu ,1\leqslant \nu \leqslant m+1\) определить соответствующее
ему множество \(I_{\nu }\) нижних индексов точек, в которых вычислялись значения
функций \(g_{\nu }(x)\):
\begin{displaymath}
  I_{\nu }=\{i:\nu (x_{i})=\nu ,1\leqslant i\leqslant k\},1\leq \nu \leqslant m+1,
\end{displaymath}
определить максимальное значение индекса \(M=\max\{\nu (x_{i}),1\leq i\leq k\}\).

Шаг 3. Вычислить текущие оценки для неизвестных констант Гёльдера:
\begin{equation}
  \label{step2}
  \mu _{\nu }=\max\{\frac{|g_{\nu }(x_{i})-g_{\nu }(x_{j})|}{(x_{i}-x_{j})^{\frac{1}{N}}}:i,j\in I_{\nu },i>j\}.
\end{equation}
Если множество \(I_{\nu }\) содержит менее двух элементов или если значение \(\mu _{\nu }\)
оказывается равным нулю, то принять \(\mu _{\nu }=1\).

Шаг 4. Для всех непустых множеств \(I_{\nu },\nu ={\overline {1,M}}\) вычислить оценки
\begin{displaymath}
  z_{\nu }^{*}={\begin{cases}\min\{g_{\nu }(x_{i}):x_{i}\in I_{\nu }\}&\nu =M\\-\varepsilon _{\nu }&\nu <M\end{cases}},
\end{displaymath}
где вектор с неотрицательными координатами \(\varepsilon _{R}=(\varepsilon _{1},..,\varepsilon _{m})\) называется вектором резервов.

Шаг 5. Для каждого интервала \((x_{i-1};x_{i}),1\leqslant i\leqslant k\) вычислить характеристику
\begin{equation}
  \label{step3_1}
  R(i)={\begin{cases}\Delta _{i}+{\frac {(z_{i}-z_{i-1})^{2}}{(r_{\nu }\mu _{\nu })^{2}\Delta _{i}}}-2{\frac {z_{i}+z_{i-1}-2z_{\nu }^{*}}{r_{\nu }\mu _{\nu }}}&\nu =\nu (x_{i})=\nu (x_{i-1})\\2\Delta _{i}-4{\frac {z_{i-1}-z_{\nu }^{*}}{r_{\nu }\mu _{\nu }}}&\nu =\nu (x_{i-1})>\nu (x_{i})\\2\Delta _{i}-4{\frac {z_{i}-z_{\nu }^{*}}{r_{\nu }\mu _{\nu }}}&\nu =\nu (x_{i})>\nu (x_{i-1})\end{cases}}
\end{equation}
где \(\Delta _{i}=(x_{i}-x_{i-1})^{\frac{1}{N}}\). Величины \(r_{\nu }>1,\nu ={\overline {1,m}}\)
являются параметрами алгоритма. От них зависят произведения \(r_{\nu }\mu _{\nu }\),
используемые при вычислении характеристик в качестве оценок неизвестных констант Гёльдера.

Шаг 6. Выбрать наибольшую характеристику:
\begin{equation}
\label{step4}
t=\argmax_{1\leqslant i \leqslant k+1}R(i)
\end{equation}

Шаг 7. Провести очередное испытание в середине интервала \((x_{t-1};x_{t})\),
если индексы его концевых точек не совпадают: \(x^{k+1}={\frac {1}{2}}(x_{t}+x_{t-1})\).
В противном случае провести испытание в точке
\begin{displaymath}
  x^{k+1}={\frac {1}{2}}(x_{t}+x_{t-1})-\operatorname {sgn}(z_{t}-z_{t-1}){\frac {|z_{t}-z_{t-1}|^{n}}{2r_{\nu }\mu _{\nu }^{n}}},\nu =\nu (x_{t})=\nu (x_{t-1}),
\end{displaymath}
а затем увеличить \(k\) на 1.

Алгоритм прекращает работу, если выполняется условие \(\Delta_{t}\leqslant \varepsilon\),
где \(\varepsilon>0\) есть заданная точность. В качестве оценки глобально-оптимального решения выбираются значения
\begin{equation}
\varphi_k^*=\min_{1\leqslant i \leqslant k}\varphi(x_i), x_k^*=\argmin_{1\leqslant i \leqslant k}\varphi(x_i)
\end{equation}

Далее следуя подходу, описанному в \cite{BarkalovStrongin2018}, для решения серии задач (\ref{eq:many_problems}) будем
использовать \(q\) синхронно работающих копий ИАГП с тем лишь отличием, что на шаге 6 при выборе
интервала с наилучшей характеристикой, выбор будет осуществляться из всех интервалов, которые
породили на данный момент \(q\) копий ИАГП. Если наибольшая характеристика соответствует
задаче \(i\), то выполняется шаг 7 в копии метода с номером \(i\), а остальные копии метода простаивают.
Таким образом, на каждой итерации испытание проводится в задаче, наиболее перспективной с точки зрения
характеристик (\ref{step3_1}), что позволяет динамически распределять ресурсы метода между задачами.

В \cite{BarkalovStrongin2018} приведена теория сходимости такого подхода на случай решения задач без ограничений.
При наличии ограничений характеристики интервалов, на концах которых нарушено разное количество ограничений,
вычисляются в соответствии с нижними строчками из (\ref{step3_1}). Нетрудно заметить, что и в этом случае
величины характеристик нормированы, а в случае точных оценок \(z_{\nu }^{*}\) и \(\mu _{\nu }\) их
масштаб не зависит от целевой функции и ограничений конкретной задачи. Таким образом, рассуждения
из \cite{BarkalovStrongin2018} можно провести и в случае использования индексной схемы.

Параллельная модификация метода не отличается от рассматриаемой в \cite{BarkalovStrongin2018}
и заключается в выборе \(p\) интервалов на шаге 6 и выполнения \(p\) испытаний параллельно
на следующем шаге. При этом все ресурсы метода в рамках итерации могут быть направлены как на одну, так и
на \(l\leqslant p\) задач одновременно (в зависимости от того, какой из задач принадлежат выбранные методам интервалы).

\section{Результаты численных экспериментов}

Использование сгенерированных некоторыми случайными механизмами
наборов тестовых задач с известными решениями является одним из общепринятых подходов
к сравнению алгоритмов оптимизации \cite{Beiranvand2017}. В данной работе
будем использовать два генератора тестовых задач, порождающих задачи различной природы \cite{grishaginClass, Gaviano2003}.
Эти генераторы порождают задачи без нелинейных ограничений, поэтому в дополнение к ним использована
система GCGen \footnote{Исходный код системы доступен по ссылке https://github.com/UNN-ITMM-Software/GCGen} \cite{GergelBarkalov2019}, позволяющая генерировать задачи с ограничениями на основе произвольных нелинейных
функций.

Вместе с системой GCGen распростаняются примеры её использования и построения
наборов задач, каждая из которых состоит из целевой функции и двух ограничений,
порождённых генератором \(F_{GR}\) \cite{grishaginClass} или GKLS \cite{Gaviano2003}.

Обозначим набор из 100 задач, полученных с помощью GCGen и первого генератора из \cite{grishaginClass} как \(F_{GR}^C\). Механизм построения функций \(F_{GR}\) не предусматривает контроля за размерностью (она равна 2) и количеством локальных оптимумов, однако известно, что порождаемые функции
являются существенно многоэкстремальными.

Генератор GKLS \cite{Gaviano2003} позволяет получать функции заданной размерности и с заданным количеством экстремумов.
В сочетании с GCGen было порождено три класса по 100 задач размерностей 2, 3, 4. Каждая из задач имеет два ограничения.

Будем считать, что тестовая задача решена, если метод оптимизации провёл очередное испытание \(y^k\) в
\(\delta\)-окрестности глобального минимума \(y^*\), т.е. $\left\|y^k-
y^*\right\|\leqslant \delta = 0.01\left\|b-a\right\|$, где \(a\) и \(b\) --- левая и правая границы гиперкуба из (\ref{eq:task}).
Если указанное соотношение не выполнено до истечения лимита на количество испытаний, то задача считается нерешённой.
Максимальный лимит на количество испытаний установлен для каждого класса задач, в соответствии с размерностью и сложностью (см. таблицу \ref{tab:limits}).

В качестве характеристик метода оптимизации на каждом из классов будем рассматривать среднее число
испытаний, затраченное для решения одной задачи, и количество решённых задач. Чем меньше число испытаний, тем быстрее метод сходится
к решению, а значит и меньше обращается к потенциально трудоёмкой процедуре вычислений целевой функции.
Количество решённых задач говорит о надёжности метода с заданными параметрами на решаемом классе тестовых задач.
Чтобы сделать величины, характеризующие надёжность и скорость сходимости, при подсчёте среднего количества испытаний учитывались только решённые задачи.

Реализация параллельного метода была выполнена на языке C++ с использованием технологии OpenMP
для распареллеливания процесса проведения испытаний на общей памяти. Все вычислительные
эксперименты проведены на машине со следующей конфигурацией: Intel Core i7, 32GB RAM, ОС Unubtu 16.04.

\subsection{Результаты решения сгенерированных задач}



\subsection{Пример решения многокритериальной задачи}

Для демонстрации эффективности подхода к балансировке нагрузки рассмотрим пример,
в котором множество задач вида (\ref{eq:many_problems}) порождено в результате скаляризации
многокритериальной задачи оптимизации с ограничениями.

Рассмотрим тестовую задачу, предложенную в \cite{BinhKorn1999}:
\begin{equation}
  \label{eq:mco_probem}
  \begin{array}{l}
      Minimize \left \{
      \begin{array}{l}
        f_1(y) = 4 y_1^2 + 4 y_2^2 \\
        f_2(y) = (y_1-5)^2 + (y_2-5)^2 \\
      \end{array}
      \right .
      y_1\in [-1;2],y_2\in [-2;1]
      \\s.t.
      \\
      \left \{
      \begin{array}{l}
        g_1(y) = (y_1 - 5)^2 + y_2^2 - 25 \leqslant 0 \\
        g_2(y) = -(y_1 - 8)^2 + -(y_2 + 3)^2 + 7.7 \leqslant 0\\
      \end{array}
      \right .
  \end{array}
\end{equation}

Будем использовать свёртку Гермейера для скаляризации задачи (\ref{eq:mco_probem}).
После свёртки скалярная целевая функция имеет вид:
\begin{equation}
  \varphi(y,\lambda_1,\lambda_2)=\max\{\lambda_1 f_1(y), \lambda_2 f_2(y)\},
\end{equation}
где \(\lambda_1,\lambda_2\in[0,1],\: \lambda_1+\lambda_2=1\). Перебирая все возможные
коэффициенты свёртки, можно найти всё множество парето-оптимальных решений в
задаче (\ref{eq:mco_probem}). Для численного построения множества Парето выберем
100 наборов коэффициентов \((\lambda_1,\lambda_2)\) таких, что
\(\lambda_1^i=i h,\: \lambda_2^i=1-\lambda_1^i,\: h=10^{-2},i=\overline{1, 100}\).



\section{Заключение}


\bibliographystyle{spmpsci}
\bibliography{bibliography}{}
%\begin{biblio}
%\bibitem{eremin}
%Ерёмин~И.И. Фейеровские методы для задач выпуклой и линейной оптимизации.
%Челябинск: Изд-во ЮУрГУ, 2009. 200~с.
%\bibitem{levin}
%Левин~В.К. Отечественные суперкомпьютеры семейства МВС. URL:~\href{http://parallel.ru/mvs/levin.htm}{http://parallel.ru/mvs/levin.htm} (дата обращения: 27.05.2012).
%\end{biblio}

\end{document}
