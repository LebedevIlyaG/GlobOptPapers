%%%%%%%%%%%%%%%%%%%% author.tex 
%%%%%%%%%%%%%%%%%%%%%%%%%%%%%%%%%%%
%
% sample root file for your "contribution" to a proceedings volume
%
% Use this file as a template for your own input.
%
%%%%%%%%%%%%%%%% Springer 
%%%%%%%%%%%%%%%%%%%%%%%%%%%%%%%%%%


\documentclass{svproc}
%
% RECOMMENDED 
%%%%%%%%%%%%%%%%%%%%%%%%%%%%%%%%%%%%%%%%%%%%%%%%
%%%
%
\usepackage{graphicx}
\usepackage{marvosym}

% to typeset URLs, URIs, and DOIs
\usepackage{url}
\usepackage{hyperref}
\def\UrlFont{\rmfamily}

\def\orcidID#1{\unskip$^{[#1]}$}
\def\letter{$^{\textrm{(\Letter)}}$}

\begin{document}
\mainmatter              % start of a contribution
%
\title{Adaptive global optimization using graphics accelerators\thanks{This study was supported by the Russian Science Foundation, project No.\,16-11-10150.}
}
%
\titlerunning{Adaptive global optimization}  % abbreviated title (for running head)
%                                     also used for the TOC unless
%                                     \toctitle is used
%
\author{Konstantin Barkalov\letter\orcidID{0000-0001-5273-2471} \and \\  Ilya Lebedev\orcidID{0000-0002-8736-0652}}
%
\authorrunning{Konstantin Barkalov \and Ilya Lebedev} % abbreviated author list (for running head)
%
%%%% list of authors for the TOC (use if author list has to be modified)
\tocauthor{Konstantin Barkalov and Ilya Lebedev}
%
\institute{Lobachevsky State University of Nizhni Novgorod, Russia  \\
	\email{konstantin.barkalov@itmm.unn.ru},
	\email{ilya.lebedev@itmm.unn.ru}
}
	
\maketitle              % typeset the title of the contribution

\begin{abstract}

Problems of multidimensional multiextremal optimization and numerical methods for their solution are considered. The general assumption is made about the function being optimized: it satisfies the Lipschitz condition with an a priori unknown constant. Many approaches to solving problems of this class are based on reducing the dimension of the problem; i.e., addressing a multidimensional problem by solving a family of problems with lower dimension. In this work, an adaptive dimensionality reduction scheme is investigated, and its implementation using graphic accelerators is proposed. Numerical experiments on several hundred test problems were carried out, and they confirmed acceleration in the developed GPU version of the algorithm.


\keywords{Global optimization $\cdot$ Multiextremal functions $\cdot$ Reduction of dimensionality $\cdot$ Peano space-filling curves $\cdot$ Recursive optimization $\cdot$ Graphics accelerators }
\end{abstract}

\section{Introduction}

A promising direction in the field of parallel global optimization (which, indeed, is true in many areas related to the software implementation of time-consuming algorithms) is the use of graphics processing units (GPUs). In the past decade, graphics accelerators have rapidly increased performance to meet the ever-growing demands of graphics application developers. Additionally, in the past few years some principles for developing graphics hardware have changed, and as a result it has become more programmable. Today, a graphics accelerator is a flexibly programmable, massive parallel processor with high performance, which is in demand for solving a range of computationally time-consuming problems \cite{Hwu2011}.

However, the potential for graphics accelerators to solve global optimization problems has not yet been fully realized. Using the GPUs, they basically parallelize nature-inspired optimization algorithms, which are somehow based on the idea of random search (see, for example, \cite{Ferreiro2013,Garcia2014,Langdon2011}). By virtue of their stochastic nature, algorithms of this type guarantee convergence to the global minimum only in the sense of probability, which differentiates them unfavorably from deterministic methods.

With regard to many deterministic algorithms of Lipschitzian global optimization with guaranteed convergence, parallel variants have been proposed \cite{Evtushenko2009,He2008,Paulavicius2011}. However, these versions of algorithms are parallelized on the CPU using shared and/or distributed memory; presently, no GPU implementations have been made. For example, \cite{Paulavicius2011} describes parallelization of an algorithm based on the ideas of the branch and boundary method using MPI and OpenMP.

Within the framework of this research, we consider the problems of capturing the optimum, which are characterized by a lengthy period for calculating the values of objective function in comparison with the time needed for processing them. For example, objective function can be specified using systems of linear algebraic equations, systems of ordinary differential equations, etc. Currently, graphics accelerators can be used to solve problems of this type. Moreover, an accelerator can solve several such problems at once \cite{Kindratenko2014}; i.e., using the GPU, one can calculate multiple function values simultaneously.

Thus, calculating the optimization criterion can be implemented on the GPU, and the role of the optimization algorithm (running on the CPU) consists in the effective selection of points for conducting parallel tests. This scheme of working with the accelerator is fully consistent with the work of the parallel global search algorithm developed at the Lobachevsky State University of Nizhni Novgorod and presented in a series of publications [references].


\section{Multidimensional parallel global search algorithm}


\section{Parallel global search algorithm}


\section{Dimensionality reduction schemes in global optimization problems}

\subsection{Dimensionality reduction using multiple mappings}

\subsection{Adaptive dimensionality reduction scheme}


\section{GPU implementation}

\subsection{General scheme}

\subsection{Organization of parallel computing}


\section{Numerical experiments}


\section{Conclusion}

In summary, we observe that the use of graphics processors to solve global optimization problems shows noteworthy promise, because high performance in modern supercomputers is achieved (mainly) through the use of accelerators.

In this paper, we consider a parallel algorithm for solving multidimensional multiextremal optimization problems and its implementation on the GPU. In order to experimentally confirm the theoretical properties of the parallel algorithm under consideration, computational experiments were carried out on a series of several hundred test problems of different dimensions. The parallel algorithm demonstrates good acceleration, both in the GPU and CPU versions.


%
% ---- Bibliography ----
%
\bibliographystyle{spmpsci}
\bibliography{bibliography}{}

\end{document}
