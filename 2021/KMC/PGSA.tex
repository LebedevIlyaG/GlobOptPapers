\documentclass[11pt, oneside, a4paper]{article}
\usepackage[utf8]{inputenc}
%\usepackage[cp1251]{inputenc} % кодировка
\usepackage[english, russian]{babel} % Русские и английские переносы
\usepackage{graphicx}          % для включения графических изображений
\usepackage{cite}              % для корректного оформления литературы
\usepackage{pavt-ru}  

\usepackage{graphicx}
\usepackage{marvosym}
\usepackage{amsmath}
\usepackage{amssymb}


\usepackage{url}
\usepackage{hyperref}
\def\UrlFont{\rmfamily}

\def\orcidID#1{\unskip$^{[#1]}$}
\def\letter{$^{\textrm{(\Letter)}}$}                              

\begin{document}

% \title - название статьи
% \authors - список авторов

\title{Идентификация кинетических параметров химической реакции с помощью параллельного алгоритма глобального поиска }

\authors{К.А.~Баркалов, И.Г.~Лебедев}
\organizations{Нижегородский государственный университет им. Н.И. Лобачевского}

Основным способом математического моделирования химических реакций является построение кинетической модели. При разработке кинетической модели многостадийной химической реакции решается обратная задача химической кинетики, которая представляет собой задачу глобальной оптимизации с целевой функцией вида ``черный ящик''. В данной работе рассматривается решение этой задачи параллельным алгоритмом глобальной оптимизации.  Разработанный алгоритм основан на редукции исходной многомерной задачи к эквивалентной ей одномерной задаче с последующим ее решением эффективными методами оптимизации функций одной переменной. Вопросы распараллеливания алгоритма на разных архитектурах рассмотрены в \cite{Strongin13, Barkalov2016}. 

Исследуется одна из химических реакций, экспериментальные данные и ее описание рассмотрены в работе \cite {Uskov2020}. Математическая модель задач химической кинетики представляет собой систему дифференциальных уравнений, которая описывает изменения концентраций веществ во времени в соответствии со скоростью стадий реакции. Согласно этой моделе можно сформировать оптимизационную задачу, в которой целевая функция определяется как сумма абсолютных отклонений расчетных и экспериментальных концентраций:

\begin{displaymath}\label{func}
F = \sum\limits_{i=1}^M \sum\limits_{j=1}^N \left| x_{ij}^{calc} - x_{ij}^{exp} \right| \rightarrow \min,
\end{displaymath}
где $ x_ {ij} ^ {calc} $ и $ x_ {ij} ^ {exp} $ -- расчетные и экспериментальные значения концентраций компонентов; $ M $ -- количество точек измерения; $ N $ -- количество веществ, участвующих в реакции.

В результате решения поставленной задачи химической кинетики параллельным методом глобального поиска были рассчитаны кинетические параметры процесса предриформинга пропана на Ni катализаторе.

Вычислительные эксперименты проводились на кластере «Лобачевский», Узел кластера состоит из 2-х Intel Sandy Bridge E5-2660 2.2 GHz, 64 Gb RAM. Алгоритм глобального поиска запускался в последовательном и паралелльном режимах, максимальное число задействованных процессоров состовляло 40, и ускорение составило соответственно 15.63.




\begin{biblio}


\bibitem{Strongin13}
Стронгин~Р.Г., Гергель~В.П., Гришагин~В.A., Баркалов~К.А. Параллельные вычисления в задачах глобальной оптимизации. М.: Издательство Московского университета, 2013. 280~с.

\bibitem{Barkalov2016} Barkalov K., Lebedev I. Solving multidimensional global optimization problems using graphics accelerators // Communications in Computer and Information Science. 2016. vol. 687, pp. 224-235. 

\bibitem{Uskov2020} Uskov S.I., Potemkin D.I., Enikeeva L.V., Snytnikov P.V., Gubaydullin I.M., Sobyanin V.A. Propane pre-reforming into methane-rich gas over Ni catalyst: experiment and kinetics elucidation via genetic algorithm. Energies. 2020. vol. 13, art.no 3393.


\end{biblio}
\end{document}