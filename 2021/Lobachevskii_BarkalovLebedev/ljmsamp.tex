%%
%% ****** ljmsamp.tex 13.06.2018 ******
%%
\documentclass[
11pt,%
tightenlines,%
twoside,%
onecolumn,%
nofloats,%
nobibnotes,%
nofootinbib,%
superscriptaddress,%
noshowpacs,%
centertags]%
{revtex4}
\usepackage{ljm}

\begin{document}

\titlerunning{Short form of the title} % for running heads
%\authorrunning{First-Author at al.} % for running heads
\authorrunning{Barkalov, Lebedev} % for running heads

\title{Title of the Article,\\
Broken into Lines}
% Splitting into lines is performed by the command \\
% The title is written in accordance with the rules of capitalization.

\author{\firstname{K.~A.}~\surname{Barkalov}}
\email[E-mail: ]{konstantin.barkalov@itmm.unn.ru}
\affiliation{Lobachevsky State University of Nizhni Novgorod, Gagarin ave. 23, Nizhni Novgorod, 603950 Russia}


\author{\firstname{I.~G.}~\surname{Lebedev}}
\email[E-mail: ]{ilya.lebedev@itmm.unn.ru}
\affiliation{Lobachevsky State University of Nizhni Novgorod, Gagarin ave. 23, Nizhni Novgorod, 603950 Russia}
%\noaffiliation % If the author does not specify a place of work.

\firstcollaboration{(Submitted by A.~A.~Editor-name)} % Add if you know submitter.
%\lastcollaboration{ }

\received{February 1, 2021} % The date of receipt to the editor, i.e. December 06, 2017


\begin{abstract} % You shouldn't use formulas and citations in the abstract.
In this example, the article contains some required author
information and examples of how to gain an article in the REV\TeX~4
for \ljm. You shouldn't use formulas and citations in the abstract.
\end{abstract}

\subclass{90C26, 90C30} % Enter 2010 Mathematics Subject Classification.

\keywords{Global optimization, Non-convex constraints, Mixed-integer problems, Local tuning, Parallel algorithms} % Include keywords separeted by comma.

\maketitle

% Text of article starts here.

\section{Introduction}

To prepare your manuscript, use REV\TeX{} package. The latest version can be downloaded at the project homepage \cite{RTeXHome}.
The process of processing an article using REV\TeX{} is described in detail in the package manuals \cite{RTeX}. Great help in resolving technical
issues on \TeX{} can be found in books \cite{texbook,L,GG,KD}.
Use the ``House Style Guide: Version 2.0'' \cite{hsg} for more help on how to format article.

\section{First level Heading\protect\\
broken into lines}

There are three levels of Headings that are set by the commands
\verb|\section|, \verb|\subsection|, and \verb|\subsubsection|
(Chapter, subchapter, subsection). Split header into lines
is done with the command \verb+\protect\\+.
Second level heading as the title of article is written in accordance with the rules of capitalization \cite{cap}.

\subsection{Second Level Heading}

References in the text are by using the commands
\verb+\cite{#1}+ or \verb+\onlinecite{#1}+. Label \verb+#1+
can have a name consisting of both letters and numbers. In
the bibliography section this link also has a label \verb+#1+ and starts the command \verb+\bibitem{#1}+.

With the command \verb+\cite{texbook}+ get a reference \cite{texbook},
if you need to refer to several sources:
\cite{texbook,L}, the curly brackets indicate references
separated by commas: \verb+\cite{texbook,L}+. When
use the command \verb|\onlinecite{#1}| link will not be
in brackets: see~\onlinecite{texbook}, p.~8.
If there is a link to the sources with serial numbers,
e.g. [1,2,3,6,7,8], then print automatically following links will take
[1--3,6--8].

\section{Standalone formulae}

\subsection{Another second-level heading}

In \LaTeX, there are many ways to embed standalone formulas
on the page and align them. By default, formulas always
centered.

\subsubsection{One-line formulas}

Below are examples of one-line equations:
\begin{eqnarray}
\chi_+(p)\alt{\bf [}2|{\bf p}|(|{\bf p}|+p_z){\bf ]}^{-1/2}
\left(
\begin{array}{c}
|{\bf p}|+p_z\\
px+ip_y
\end{array}\right)\;,
\\
\left\{%
\openone234567890abc123\alpha\beta\gamma\delta1234556\alpha\beta
\frac{1\sum^{a}_{b}}{A^2}%
\right\}%
\label{one}.
\end{eqnarray}
The second formula has the number~(\ref{one}), which is set by command
\verb|\label{one}|. The first formula is assigned a number~(1), but
it cannot be applied with the help of automatic links,
as it has no label.

If the formula number is not necessary, use the environment
\verb+\[+, \verb+\]+, (or \$...\$) which obtained the following formula:
\[g^+g^+ \rightarrow g^+g^+g^+g^+ \dots ~,~~q^+q^+\rightarrow
q^+g^+g^+ \dots ~. \] % You also can use the dollar sign ($).

\subsubsection{Multiline formulae}

Multiline formulas are typed using the environment
eqnarray:
\begin{eqnarray}
{\cal M}=&&ig_Z^2(4E_1E_2)^{1/2}(l_i^2)^{-1}
\delta_{\sigma_1,-\sigma_2}
(g_{\sigma_2}^e)^2\chi_{-\sigma_2}(p_2)\nonumber\\
&&\times
[\epsilon_jl_i\epsilon_i]_{\sigma_1}\chi_{\sigma_1}(p_1),
\end{eqnarray}
\begin{eqnarray}
\sum \vert M^{\rm viol}_g \vert ^2&=&g^{2n-4}_S(Q^2)~N^{n-2}
(N^2-1)\nonumber \\
 & &\times \left( \sum_{i<j}\right)
\sum_{\rm perm}
\frac{1}{S_{12}}
\frac{1}{S_{12}}
\sum_\tau c^f_\tau~.
\end{eqnarray}
If the formula number is not necessary, then at the end of the row in front of the sign
\verb|\\| you need to put the command \verb|\nonumber|. Never
use one line command \verb|\nonumber| and
\verb|\label{#1}|, as this may cause error in automatic
the numbering of references.

If you want to gain a few formulas without number,
use the eqnarray environment* (the asterisk means the abolition
numbering):
\begin{eqnarray*}
\sum \vert M^{\rm viol}_g \vert ^2&=&g^{2n-4}_S(Q^2)~N^{n-2}
(N^2-1)\\
& &\times \left( \sum_{i<j}\right)
\left(
\sum_{\rm perm}\frac{1}{S_{12}S_{23}S_{n1}}
\right)
\frac{1}{S_{12}}~.
\end{eqnarray*}
To add numbers to the formula manually, use the command
\verb+\tag{#1}+, where \verb+#1+~--- the desired equation number.
Here how is the formula with the number of~(\ref{eq:mynum}):
\begin{equation}
g^+g^+ \rightarrow g^+g^+g^+g^+ \dots ~,~~q^+q^+\rightarrow
q^+g^+g^+ \dots ~. \tag{2.6$'$}\label{eq:mynum}
\end{equation}

When you enable single-line and multi-line formulas are surrounded by
subequations, each formula `numbered" with a letter,
as shown in equations~(\ref{mlett:1}) and (\ref{mlett:2}):
\begin{subequations}
\label{generallabel}
\begin{equation}
\left\{
abc123456abcdef\alpha\beta\gamma\delta1234556\alpha\beta
\frac{1\sum^{a}_{b}}{A^2}
\right\},\label{mlett:1}
\end{equation}
\begin{eqnarray}
{\cal M}=&&ig_Z^2(4E_1E_2)^{1/2}(l_i^2)^{-1}
(g_{\sigma_2}^e)^2\chi_{-\sigma_2}(p_2)\nonumber\\
&&\times
[\epsilon_i]_{\sigma_1}\chi_{\sigma_1}(p_1).\label{mlett:2}
\end{eqnarray}
\end{subequations}
If you put the label right after the \verb+\begin{subequations}+,
it can be used further as a reference to all the equations in this
environment. For example, you can refer to
equation~(\ref{generallabel}) this example.

To set multi-line formulas you can use the environment
multline, gather and align. The multline environment is good to use
for a set of standalone long formulas that do not fit on
one line:
\begin{multline}
\int_{a_1}^{a_2} f(x)\,dx+\int_{a_2}^{a_3} f(x)\,dx
+\dots+\int_{a_{n-1}}^{a_n} f(x)\,dx\\
+\int_{a_1}^{a_2} g(x)\,dx+\int_{a_2}^{a_3} g(x)\,dx
+\dots+\int_{a_{n-1}}^{a_n} g(x)\,dx\\
+\int_{a_1}^{a_2} h(x)\,dx+\int_{a_2}^{a_3} h(x)\,dx
+\dots+\int_{a_{n-1}}^{a_n} h(x)\,dx\\
=\int_{a_1}^{a_n} f(x)+g(x)+h(x)\,dx.
\end{multline}
This formula is automatically numbered, if the formula number is not
need, you have to use the environment multline*.

Environment gather centers included in the formula:
\begin{gather}
\int_{a_1}^{a_2} f(x)\,dx+\int_{a_2}^{a_3} f(x)\,dx
+\dots+\int_{a_{n-1}}^{a_n} f(x)\,dx\\
+\int_{a_1}^{a_2} g(x)\,dx+\int_{a_2}^{a_3} g(x)\,dx
+\dots+\int_{a_{n-1}}^{a_n} g(x)\,dx\notag\\
+\int_{a_1}^{a_2} h(x)\,dx+\int_{a_2}^{a_3} h(x)\,dx
+\dots+\int_{a_{n-1}}^{a_n} h(x)\,dx\\
=\int_{a_1}^{a_n} f(x)+g(x)+h(x)\,dx.
\end{gather}
Each line is automatically numbered, if the line number
it is not necessary, before \verb+\\+ in this line, you need to put the command
\verb+\notag+. When you use the environment gather* formula
will not be numbered.

The align environment allows you to align formulas on your
discretion:
\begin{align}
\int_{a_1}^{a_2} f(x)\,dx &+\int_{a_2}^{a_3} f(x)\,dx
+\dots+\int_{a_{n-1}}^{a_n} f(x)\,dx\notag\\
&+\int_{a_1}^{a_2} g(x)\,dx +\int_{a_2}^{a_3} g(x)\,dx
+\dots+\int_{a_{n-1}}^{a_n} g(x)\,dx\\
&+\int_{a_1}^{a_2} h(x)\,dx+\int_{a_2}^{a_3} h(x)\,dx
+\dots+\int_{a_{n-1}}^{a_n} h(x)\,dx\notag\\
&=\int_{a_1}^{a_n} f(x)+g(x)+h(x)\,dx.
\end{align}
Read more about working with these environments can be found in the book
\cite{GG}.

\section{Variables}
For the writing of definitions, theorems, lemmas and their proofs use the following variables. If necessary, you can add your variables in the sample.

\begin{definition}\label{D:1}
Define ...
\end{definition}

\begin{lemma}\label{L:1}
If ...
\end{lemma}

\begin{theorem}\label{Th:1}
Let ...
\end{theorem}
\begin{proof}
Consider...
\end{proof}

\section{Figures and tables}
REV\TeX~4 will automatically number sections, equations, tables, and
drawings. Figure captions are made after the image and the signature
tables~--- before a table, as shown in the examples at the end. In
tables footnotes \footnote{Footnotes in tables, you can try
to do this manually.} not working.

Graphics and diagrams should be monochrome in black-and-white color and saved in EPS format.
%You can include a picture in the article using the figure environment:
%\begin{figure}[h]
%\setcaptionmargin{5mm}
%%\onelinecaptionsfalse % if the caption is multiline
%\onelinecaptionstrue  % if the caption is one-line
%\includegraphics[width=0.85\textwidth]{deform.eps}
%\captionstyle{normal}\caption{Please write your figure caption here.}\label{fig:1}
%\end{figure}

\begin{table}[!h]
\setcaptionmargin{0mm}
\onelinecaptionsfalse
\captionstyle{flushleft}
\caption{ For the insertion of tables, the table environment is used,
 the signatures to the tables are made in the same way as the captions
 for the figures. This is an example of a table whose multi-line name
 is decorated with the \textbf{caption2} package.}
\bigskip
\begin{tabular}{|c|c|c|c|c|c|c|c|}
\hline
 &$r_c$ (\AA)&$r_0$ (\AA)&$\kappa r_0$&
 &$r_c$ (\AA) &$r_0$ (\AA)&$\kappa r_0$\\
\hline
Cu& 0.800 & 14.10 & 2.550 &Sn
& 0.680 & 1.870 & 3.700 \\
Ag& 0.990 & 15.90 & 2.710 &Pb
& 0.450 & 1.930 & 3.760 \\
Au& 1.150 & 15.90 & 2.710 &Ca
& 0.750 & 2.170 & 3.560 \\
Mg& 0.490 & 17.60 & 3.200 &Sr
& 0.900 & 2.370 & 3.720 \\
Zn& 0.300 & 15.20 & 2.970 &Li
& 0.380 & 1.730 & 2.830 \\
Cd& 0.530 & 17.10 & 3.160 &Na
& 0.760 & 2.110 & 3.120 \\
Hg& 0.550 & 17.80 & 3.220 &K &  1.120 & 2.620 & 3.480 \\
Al& 0.230 & 15.80 & 3.240 &Rb & 1.330 & 2.800 & 3.590 \\
Ga& 0.310 & 16.70 & 3.330 &Cs & 1.420 & 3.030 & 3.740 \\
In& 0.460 & 18.40 & 3.500 &Ba & 0.960 & 2.460 & 3.780 \\
Tl& 0.480 & 18.90 & 3.550 & & & & \\[1mm]
\hline
\end{tabular}
\end{table}

\begin{table}[!htb]
\setcaptionmargin{0mm}
\onelinecaptionstrue
\captionstyle{flushleft}
\caption{The name of this table is --- one-line.}
\bigskip
\begin{tabular}{|c|c|c|c|c|c|c|}
  \hline
    & 1 & 2 & 3 & 4 & 5 & 6\\
  \hline
  1 & 1 & 2 & 3 & 4 & 5 & 6\\
  2 & 2 & 4 & 6 & 8 & 10 & 12\\
  3 & 3 & 6 & 9 & 12 & 15 & 18\\[1mm]
  \hline
\end{tabular}
\end{table}

\section{Bibliography}
The following rules apply for references to books:

\begin{itemize}
  \item Authors initials go before the surname with a space between the initials. Use and between the last two authors. All authors listed in the original reference should be cited.
  \item The title of the book is written in italics. If the book is originally published in Russian, only the English translation of the title is cited and after citation the original language may be indicated in square brackets, e.g.: [in Russian].
  \item Examples of bibliographic references can be found at the end of this page % (i.e. article \cite{ex}, PhD thesis \cite{maguire-76}).
\end{itemize}


\begin{acknowledgments}
%We are grateful to C.~C.~Surname3 and the reviewers for careful reading of the manuscript and helpful remarks.
This work was supported by the Ministry of Science and Higher Education of the Russian Federation, project no. 0729-2020-0055, and by the Research and Education Mathematical Center, project no. 075-02-2020-1483/1.
\end{acknowledgments}


%
% The Bibliography
%

\begin{thebibliography}{99}

%\bibitem{Floudas}
%C.~A.~Floudas and M.~P.~Pardalos, \textit{Recent advances in global optimization} (Princeton University Press, 2016).
%
%\bibitem{Locatelli}
%M.~Locatelli and F.~Schoen, \textit{Global optimization: theory, algorithms and applications} (SIAM, 2013).
%
%\bibitem{Strongin1}
%R.~G.~Strongin and Y.~D.~Sergeyev, \textit{Global optimization with non-convex constraints. Sequential and parallel algorithms} (Kluwer Academic Publishers, Dordrecht, 2000, 2nd ed. 2013, 3rd ed. 2014).
%
%\bibitem{Pardalos}
%P.~M.~Pardalos, A.~A.~Zhigljavsky and J.~\v{Z}ilinskas \textit{Advances in stochastic and deterministic global optimization} (Springer, 2016).
%
%\bibitem{Sergeyev1}
%Y.~D.~Sergeyev and D.~E.~Kvasov, \textit{Deterministic Global Optimization. An Introduction to the Diagonal Approach}  (Springer Briefs in Optimization, Springer, 2017).
%
%\bibitem{Paulavicius}
%R.~Paulavi\v{c}ius, J.~\v{Z}ilinskas, \textit{Simplicial Global Optimization} (Springer Briefs in Optimization. Springer, 2014).
%
%\bibitem{Ciegis}
%R.~\v{C}iegis, D.~Henty, B.~K\r{a}gstr\"om and J.~\v{Z}ilinskas, \textit{Parallel scientific computing and optimization: advances and applications}  (Springer, 2009). 
%
%\bibitem{Luque}
%G.~Luque and E.~Alba, \textit{Parallel genetic algorithms. Theory and real world applications} (Springer-Verlag, Berlin, 2011).
%
%\bibitem{Strongin2}
%R.~G.~Strongin, V.~P.~Gergel, V.~A.~Grishagin and K.~A.~Barkalov, \textit{Parallel computations for global optimization problems} (Moscow State University, Moscow, 2013) [In Russian].
%
%\bibitem{Sergeyev2}
%Ya.~D.~Sergeyev, R.~G.~Strongin, D.~Lera, \textit{Introduction to Global Optimization Exploiting Space-Filling Curves} (Springer Briefs in Optimization, Springer, 2013).
%
%\bibitem{Strongin3}
%R.~G.~Strongin, \textit{Numerical Methods in Multiextremal Problems (Information-Statistical Algorithms)} (Nauka, Moscow, 1978) [In Russian].
%
%\bibitem{Strongin4}
%R.~G.~Strongin and Y.~D.~Sergeyev, \textit{Global multidimensional optimization on parallel computer}, Parallel Computing \textbf{18} (11), 1259--1273 (1992).
%
%\bibitem{Sergeyev4}
%Y.~D.~Sergeyev and V.~A.~Grishagin, \textit{Parallel asynchronous global search and the nested optimization scheme}, J. Comput. Anal. Appl., \textbf{3} (2), 123--145 (2001).
%
%\bibitem{Gergel1}
%V.~P.~Gergel and S.~V.~Sidorov, \textit{A Two-Level Parallel Global Search Algorithm for Solution of Computationally Intensive Multiextremal Optimization Problems}, Lecture Notes in Computer Science \textbf{9251}, 505--515 (2015).
%
%\bibitem{Gergel2}
%V.~Gergel, \textit{An Unified Approach to Use of Coprocessors of Various Types for Solving Global Optimization Problems}, in Proceedings of the Second International Conference on Mathematics and Computers in Sciences and in Industry (MCSI), Sliema, 13--18 (2015).
%
%\bibitem{Barkalov}
%K.~Barkalov, V.~Gergel and I.~Lebedev, \textit{Solving global optimization problems on GPU cluster}, AIP Conference Proceedings \textbf{1738}, 400006 (2016).
%
%\bibitem{Gergel3}
%V.~Gergel and E.~Kozinov, \textit{Efficient methods of multicriterial optimization based on the intensive use of search information}, Springer Proceedings in Mathematics and Statistics \textbf{197}, 27--45 (2017). 
%
%\bibitem{Gergel4}
%V.~Gergel and E.~Kozinov, \textit{Parallel computing for time-consuming multicriterial optimization problems}, Lecture Notes in Computer Science \textbf{10421}, 446--458 (2017).
%
%\bibitem{Gergel5}
%V.~Gergel, V.~Grishagin and A.~Gergel, \textit{Adaptive nested optimization scheme for multidimensional global search}, J. Glob. Optim.  \textbf{66} (1), 35--51 (2016).
%
%\bibitem{Lera}
%D.~Lera, Y.~D.~Sergeyev, \textit{Lipschitz and Holder global optimization using space-filling curves},  Appl. Numer. Math. \textbf{60} (1-2), 115--129 (2010).
%
%\bibitem{Grishagin1}
%V~A.~Grishagin. \textit{On convergence conditions for a class of global search algorithms}, in Proceedings of the 3-rd All-Union seminar ``Numerical methods of nonlinear programming'', Kharkov, 82--84 (1979) [In Russian].
%
%\bibitem{Grishagin2}
%V.~A.~Grishagin, Y.~D.~Sergeyev and R.~G.~Strongin, \textit{Parallel characteristic algorithms for solving problems of global optimization}, J. Glob. Optim. \textbf{10} (2), 185--206 (1997).
%
%\bibitem{Strongin5}
%R.~G.~Strongin, \textit{Algorithms for Multi-extremal Mathematical Programming Problems Employing the Set of Joint Space-filling Curves}, J. Glob. Optim. \textbf{2} (4), 357--378 (1992).
%
%\bibitem{Sysoyev}
%A.~Sysoyev, K.~Barkalov, V.~Sovrasov, I.~Lebedev and V.~Gergel, \textit{Globalizer -- A parallel software system for solving global optimization problems}, Lecture Notes in Computer Science \textbf{10421}, 492--499 (2017). 
%
%\bibitem{Sergeyev7}
%Y.~D.~Sergeyev and V.~A.~Grishagin, \textit{Parallel Asynchronous Global Search and the Nested Optimization Scheme}, J. Comput. Anal. Appl. \textbf{3} (2), 123--145 (2001).
%
%\bibitem{Gaviano}
%M.~Gaviano, D.~Lera, D.~E.~Kvasov and Ya.~D.~Sergeyev, \textit{Software for generation of classes of test functions with known local and global minima for global optimization}, ACM Trans. Math. Softw. \textbf{29}, 469--480 (2003).
%
%\bibitem{Modorskii}
%V.~Y.~Modorskii, D.~F.~Gaynutdinova, V.~P.~Gergel and K.~A.~Barkalov, \textit{Optimization in design of scientific products for purposes of cavitation problems}, AIP Conference Proceedings \textbf{1738}, 400013 (2016).
%
%\bibitem{Gergel6}
%V.~P.~Gergel, M.~I.~Kuzmin, N.~A.~Solovyov and V.~A.~Grishagin, \textit{Recognition of surface defects of cold-rolling sheets based on method of localities}, International Review of Automatic Control \textbf{8} (1), 51--55 (2015).


\end{thebibliography}
\end{document}
