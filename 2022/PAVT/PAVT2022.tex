%%%%%%%%%%%%%%%%%%%% author.tex 
%%%%%%%%%%%%%%%%%%%%%%%%%%%%%%%%%%%
%
% sample root file for your "contribution" to a proceedings volume
%
% Use this file as a template for your own input.
%
%%%%%%%%%%%%%%%% Springer 
%%%%%%%%%%%%%%%%%%%%%%%%%%%%%%%%%%


\documentclass{svproc}
%
% RECOMMENDED 
%%%%%%%%%%%%%%%%%%%%%%%%%%%%%%%%%%%%%%%%%%%%%%%%
%%%
%
\usepackage{graphicx}
\usepackage{marvosym}
\usepackage{amsmath}
\usepackage{amssymb}
\usepackage{cite}

\usepackage[russian]{babel}

% to typeset URLs, URIs, and DOIs
\usepackage{url}
\usepackage{hyperref}
\def\UrlFont{\rmfamily}

\def\orcidID#1{\unskip$^{[#1]}$}
\def\letter{$^{\textrm{(\Letter)}}$}

\begin{document}
\mainmatter              % start of a contribution
% Решение обратных задач химической кинетки с помощью асинхронного алгоритма глобального поиска
% Решение обратных задач химической кинетки с помощью смешанного локально-глобального поискового алгоритма 
%\title{Solving the Inverse Problems of Chemical Kinetics Using the Asynchronous Global Optimization Algorithm}
\title{Modeling of Isobutane Alkylation with Mixed C4 Olefins and Sulfuric Acid as Catalyst Using the Asynchronous Global Optimization Algorithm}
\titlerunning{Solving the Inverse Problems of Chemical Kinetics}  % abbreviated title (for running head)
%                                     also used for the TOC unless
%                                     \toctitle is used
%
\author{
Irek Gubaydullin$^{1,2}$\and
Leniza Enikeeva$^{2,3}$\orcidID{0000-0003-4219-4870}
\and
Konstantin Barkalov$^4$ \letter \orcidID{0000-0001-5273-2471}
\and
Ilya Lebedev$^4$\orcidID{0000-0002-8736-0652} 
}

%
\authorrunning{I. Gubaydullin et al.} % abbreviated author list (for running head)
%
%%%% list of authors for the TOC (use if author list has to be modified)
%\tocauthor{Konstantin Barkalov and Ilya Lebedev }
%

\institute{$^1$Institute of Petrochemistry and Catalysis – Subdivision of the Ufa Federal Research Centre of RAS, Ufa, Russia\\$^2$Ufa State Petroleum Technological University, Ufa, Russia \\$^3$Novosibirsk State University, Novosibirsk, Russia\\$^4$Lobachevsky State University of Nizhny Novgorod, Nizhny Novgorod, Russia\\
\email{leniza.enikeeva@yandex.ru},
\email{konstantin.barkalov@itmm.unn.ru},
\email{ilya.lebedev@itmm.unn.ru}
}

	
\maketitle              % typeset the title of the contribution

\begin{abstract}

The paper considers the application of parallel computing technology to the simulation of catalytic chemical reaction, which is widely used in the modern automobile industry to produce gasoline with high octane number. As a chemical reaction, the process of alkylation of isobutane with mixed C4 olefins catalyzed by sulfuric acid is assumed. To simulate a chemical process, it is necessary to develop a kinetic model of the process, that is, to determine the kinetic parameters. To do this, the inverse problem of chemical kinetics is solved, which predicts the values of kinetic parameters based on laboratory data. From a mathematical point of view, the inverse problem of chemical kinetics is a global optimization problem. A parallel information-statistical global search algorithm was used to solve it. The use of the parallel algorithm has significantly reduced the search time to find the optimum. The found optimal parameters of the model made it possible to adequately simulate the process of alkylation of isobutane with mixed C4 olefins catalyzed by sulfuric acid.

%обновить ключевые слова
\keywords{Global optimization $\cdot$ Multiextremal functions $\cdot$ Parallel computing $\cdot$ Chemical kinetics $\cdot$ Inverse problems }
\end{abstract}

\section{Introduction}

%Часть УГНТУ
Currently, there is a tendency to improve the environmental characteristics of automobile fuel while maintaining a high octane number. Sulfuric acid alkylation of isobutane with olefins makes it possible to obtain a high-octane component of gasoline with a minimum content of aromatic hydrocarbons. The alkylate, which is produced by the alkylation of isobutane with C3-C5 olefins in the presence of strong acid, has the advantages of high octane number, low vapor pressure, and zero content of olefins and aromatics that allow it to be a desirable blending component for high-quality gasoline. Alkylates will continue to act as a desirable blending component for high-quality gasoline as the quality of gasoline continues to increase [ссылка на китайскую статью]. Therefore, it is a significant process of the modern refinery [ссылка на catalyst]. To optimize the chemical process in industry, it is necessary to first develop its model, which in this case means building a mathematical model of the chemical process and then its kinetic model, that is, numerically calculate the kinetic constants of the reaction. 

%Стыковочный абзац
As a rule, it is impossible to find out the kinetic constants of the reactions analytically. Therefore, there is a need in the development and application of numerical methods for finding the kinetic constants (see, e.g., []). In this case, the quality criteria of the solution found (objective function) haven't an explicitt analytical description but allows an algorithmic representation and requires considerable computaton resources. Moreover, in the inverse problems of chemical kinetics the objective function can be multiextremal essentially, i.e. can have many local extrema along with the global one. 

%Часть ННГУ
The numerical methods for solving such multiextremal problems (global optimization methods) differ significantly from the local search ones (see, e.g., \cite{Sergeyev2017,PaulaviciusZilinskas2014}). As a rule, the local optimization methods cannot escape a local extremum attractione region and don't find the global optimum. At the same time, the use of the model parameters corresponding to the local solution found may appear to be insufficient since the global solution may provide a considerable advantage as compared to the local ones. 

The diversity of the global optimization problems arising entails various approaches to solving these ones. The methods of solving the global optimization problems can be divided into two classes: the metaheuristic methods and the deterministic ones. The metaheuristic algorithms, as a rule, are based on the simulation of the processes going in the nature. Some examples of the metaheuristic algorithms are simulated annealing, evolution and genetic algorithms, etc. (see e.g. \cite{Battiti2009,Eiben2015}). Because of relative simplicity, the metaheuristic algorithms are more popular among the researchers than the deterministic methods. 
However, a problem solution found by a metaheuristic algorithm is, generally speaking, a local one and may be far from the global solution \cite{Kvasov2018}. 

The possibility to construct deterministic global search methods different from the grid search and from the metaheuristic methods is related to the availability of and taking into account some {\it a priori} assumption on the properties of the problem functions. Such assumptions play a key role in the development of efficient global optimization algorithms and serve as main mathematical tools for estimating the global solutions.

An assumption on limited relative variations of the objective function values is one of natural assumptions of a problem. Such an assumption is related to the ratio of the function increment to respective increment of its argument, which is usually limited by some threshold defined by a limited energy of variations in the simulated system. In this case the functions are known as Lipschitz ones and the problem itself is called the Lipschitz global optimization problem. 

This paper presents the results of application of parallel Lipschitz optimization methods for solving the inverse problems of chemical kinetics. The main part of the paper has the following structure. The description of the mathematical model of the investigated chemical reaction is presented in Section 2. The formal statement of the Lipschitz global optimization problems and the asynchronous parallel algorithm for solving these ones are described in Section 3. The results of the numerical solving the inverse problem of chemical kinetics are discussed in Section 5.

\section{Problem Statement}\label{Sec_math_mod}
%Содержательная постановка задачи

Рассмотрим математическую модель реакции алкилирования изобутана олефинами в присутствии серной кислоты, которая представляет собой систему обыкновенных нелинейлых дифференциальных уравнений (\ref{eq:one}) -- (\ref{eq:twelve}).

\begin{equation}
  \frac{dc_1}{dt} = -k_1c_1 + k_2c_3 - k_3c_1c_3 - k_7c_1c_2c_4 - k_{11}c_1 + k_{14}c_{11}
  \label{eq:one}
\end{equation}

\begin{equation}
  \frac{dc_2}{dt} = -k_4c_2c_4 - k_6c_2c_5 - k_7c_1c_2c_4 - k_{15}c_{11}c_2c_4
  \label{eq:two}
\end{equation}

\begin{equation}
  \frac{dc_3}{dt} = k_1c_1 + k_4c_2c_5 - k_3(c_1 + c_{11})c_3 - k_{5}c_{12}c_3 - k_2c_3 + k_7c_1c_2c_4 + k_{15}c_{11}c_2c_4
  \label{eq:three}
\end{equation}

\begin{equation}
  \frac{dc_4}{dt} = k_3(c_1 + k_{11})c_3 - k_4c_2c_4 - k_{7}c_{1}c_2c_4 - k_{15}c_{11}c_2c_4
  \label{eq:four}
\end{equation}

\begin{equation}
  \frac{dc_5}{dt} = k_5c_{12}c_3 - k_{6}c_2c_5
  \label{eq:five}
\end{equation}

\begin{equation}
  \frac{dc_6}{dt} = k_4c_{2}c_4
  \label{eq:six}
\end{equation}

\begin{equation}
  \frac{dc_7}{dt} = k_6c_{2}c_5 - k_{10}c_7
  \label{eq:seven}
\end{equation}

\begin{equation}
  \frac{dc_8}{dt} = k_7c_{1}c_2c_4 + k_{15}c_{11}c_2c_4 + k_9c_9c_{10} - k_8c_8
  \label{eq:eight}
\end{equation}

\begin{equation}
  \frac{dc_9}{dt} = k_8c_{8} - k_{9}c_{9}c_{10}
  \label{eq:nine}
\end{equation}

\begin{equation}
  \frac{dc_{10}}{dt} = k_8c_{8} - k_{9}c_{9}c_{10}
  \label{eq:ten}
\end{equation}

\begin{equation}
  \frac{dc_{11}}{dt} = -k_3c_{11}c_3 - k_{15}c_{11}c_{2}c_4 + k_{11}c_1 + k_{12}c_{12} - k_{13}c_{11} - k_{14}c_{11}
  \label{eq:eleven}
\end{equation}

\begin{equation}
  \frac{dc_{12}}{dt} = -k_5c_{12}c_3 + k_{13}c_{11} - k_{12}c_{12}
  \label{eq:twelve}
\end{equation}

The initial conditions are $t = 0, c_1 = c_1^0; c_2=c_2^0; c_3 = 0; c_4 = 0; c_5= 0; c_6 = 0; c_7 = 0; c_8 = 0; c_9 = 0; c_{10} = 0; c_{11}=c_{11}^0; c_{12} = c_{12}^0$.
The corresponding species in eqs (\ref{eq:one}) -- (\ref{eq:twelve}) are 1, iC4H8; 2, iC4, 3, iC4+; 4, TMPs+; 5, DMHs+; 6, TMPs; 7, DMHs; 8, HEs; 9, iCx+; 10, iCy=; 11, 2-C4H8; 12, 1-C4H8.

В качестве экспериментальных данных взяты данные со статьи [], которые представляют собой изменение концентраций компонентов реакции во времени при различных температурах; в качестве примера экспериментальных данных для сырья \#1 при температуре 276.2 K представлены данные в таблице \ref{table1}.

\begin{table}
\label{table1}
\caption{Experimental Data}
\begin{center}
\begin{tabular}{ccccccc}
\hline
%\multicolumn{1}{l}{\rule{0pt}{12pt}
%                   Year}&\multicolumn{2}{l}{World population}\\[2pt]
& 1 min & 2 min & 5 min & 10 min & 15 min & 20 min  \\
\hline\rule{0pt}{12pt}
DMH & 0.12 & 0.11 & 0.1	& 0.1 &	0.095 &	0.09  \\
TMP & 0.54 & 0.65 & 0.69 & 0.69 & 0.7 & 0.705 \\[2pt]
\hline
\end{tabular}
\end{center}
\end{table}

Таким образом, решая систему (\ref{eq:one}) -- (\ref{eq:twelve}) с соответсвующими начальными данными, мы получим изменение рассчитанных концентраций компонентов реакции по времени. ...
Необходимо минимизировать следующую целевую фукнцию:

\begin{equation}
  F = \sum ...
  \label{eq:objective_func}
\end{equation}


\section{Parallel Algorithm for Solving Global Optimization Problems }\label{Sec_GSA}

\subsection{Global Optimization Problem}

As it has been already mentioned above, from the formal point of view, we consider the inverse problem of chemical kinetics as Lipschitz global optimization problem. In the general form, a problem of the class specified above can be formulated mathematically as 
follows:
\begin{gather}
 \varphi^* = \varphi(y^\ast)=\min{\left\{\varphi(y):y\in D\right\}}, \label{problemN}\\
 D=\left\{y\in R^N: a_i\leq y_i \leq b_i, \;  1\leq i \leq N\right\} \label{D},
\end{gather}
where $a,b$ are given vectors, $a,b\in R^N$, and the objective function $\varphi(y)$ satisfies the Lipschitz condition
\begin{equation}\label{Lip}
\left|\varphi(y_1)-\varphi(y_2)\right|\leq L\left\|y_1-y_2\right\|,\; y_1,y_2 \in D.
\end{equation}

The function $\varphi(y)$ is assumed to be multiextremal and defined in the form of ``black box'' (i.e. in the form of some computing procedure, the input of which the  vector of parameters is supplied into, and the corresponding function value is taken from the output). Moreover, each \textit{trial} (i.e. the computing of the function value at a point of the search domain) is assumed to be time-consuming operation. 
As it has been noted in Introduction, such problem statement corresponds to the inverse problem of chemical kinetics completely.

The Lipschitz condition (\ref{Lip}) can be utilized to estimate the global minimum of a function within an interval, and knowing the Lipschitz constant allows constructing the global search algorithms and proving the convergence conditions for these ones (see, for example, \cite{Strongin2000}).

The growth of the computation costs with increasing problem dimensionality is one of the main difficulties in solving the multidimensional global optimization problems. The decreasing of the number of trials at preserving the solution accuracy is possible by a complete utilization of some {\it a priori} assumptions on the objective function that leads to adaptive sequential optimization algorithms.

For example, non-uniform space covering method \cite{Evtushenko2013} and simplicial partitions method \cite{Zilinskas2010} are such methods. These approaches were applied successfully for the development of the parallel optimization methods as well \cite{Evtushenko2009,Paulavicius2011}. 
Another adaptive approach to solving a multidimensional problem (\ref{problemN}) is its reduction to a single one-dimensional problem or to several ones followed by application of the one-dimensional algorithms. 
Such a reduction can be made, for example, using nested optimization scheme \cite{Grishagin2018} or Peano-Hilbert curves \cite{Barkalov2018}. 
The latter approach was used in the present work.

Using the continuous unambiguous mapping (Peano-Hilbert curve) $y(x)$ of the interval $[0,1]$ of the real axis on the hypercube $D$ from (\ref{D}), one can reduce a multidimensional problem (\ref{problemN}) to a one-dimensional problem
\[
\varphi(y^\ast)=\varphi(y(x^\ast))=\min{\left\{\varphi(y(x)): x\in[0,1]\right\}},
\]
where the function $\varphi(y(x))$ will satisfy a uniform H{\"o}lder condition
\[
\left|\varphi(y(x_1))-\varphi(y(x_2))\right|\leq H\left|x_1-x_2\right|^{1/N}
\]
with the H{\"o}lder constant $H$ linked to the Lipschitz constant $L$ by the relation $ H=2 L \sqrt{N+3}$. 
The issues of the numerical construction of various approximations of the Peano-Hilbert curve were considered in \cite{Strongin2000,Sergeyev2013}.

So far, a search trial at some point $x'\in[0,1]$ will include first the construction of the image $y'=y(x')$ and then the computing the value of the function $z'=\varphi(y')$.

\subsection{Parallel Asynchronous Global Search Algorithm}

In the approach proposed, the parallelization scheme corresponds to the ``master/worker'' principle.
In the master process the global search algorithm is executed, which accumulates the search information, evaluates the Lipschitz constant for the objective function on its base, determines the new trial points and distributes these ones among the worker processes. 

The worker processes receive the trial points from the master process, perform the new trials at these points, and send the trial results to the master process. 

Let us assume the master process to compute one point of the next trial at every iteration and to send it to a worker process for executing the trial. 
At the same time, the execution of the trial by the worker process is much more computational-costly operation than the choice of a new trial point by the master that excludes idle worker processes. 
In this case (unlike the synchronous parallel algorithms), the total number of trials executed by each worker process will depend on the computation costs of executing a particular trial and cannot be estimated in advance.

In the description of the parallel algorithm, let us assume $p+1$ computing processes to be at our disposal: one master process and $p$ worker ones.
 
At the beginning of the search, the master process (let us assume it to be Process No 0) initiates the parallel execution of $p$ trials at $p$ different points of the search domain. 
Two of these points are the boundary ones while the rest are the internal ones, i.e. at the points $\{y(x^1), y(x^2), ...,y(x^p)\}$ where 
$x^1 = 0$, $x^p = 1$, $x^i\in(0,1), i=2,..., p-1$.

Now let us assume $k\geq 0$ trials (in particular, $k$ may be equal to 0) to be completed, and the worker processes are performing the trials at the points $\{y(x^{k+1}), y(x^{k+2}), ...,y(x^{k+p})\}$. 

Each worker process having completed its trial at some point (without any loss of generality, let us assume this point to be $y(x^{k+1})$ corresponding to process No 1), sends the trial result to the master process. 
In turn, the master process selects a new trial point $x^{k+p+1}$ for the worker process according to the rules described below.
Note that in this case we will have a set of preimages of the trial points
\[
I_k = \left\{ x^{k+1},x^{k+2},...,x^{k+p} \right\},
\]
which the trials have been already started at but haven't been completed yet.

Step 1. Renumber set of preimages of the trial points 
\[
X_k = \left\{x^1, x^2,...,x^{k+p} \right\},
\]
containing all the preimages, which the trials either have been completed or are being performed at in the increasing order (by the lower index), so that
\[
0=x_1<x_2<...<x_{x+p}=1.
\]
Step 2. Compute the values 
\begin{gather*}
M_1=\max \left\{ \frac{ \left|z_i - z_{i-1} \right|}{(x_i-x_{i-1})^{1/N}} : x_{i-1} \notin I_k, x_i \notin I_k, 2\leq i\leq k+p \right\},\\
M_2=\max \left\{ \frac{ \left|z_{i+1} - z_{i-1} \right|}{(x_{i+1}-x_{i-1})^{1/N}} : x_i \in I_k, 2\leq i < k+p \right\},\\
M=\max\{M_1,M_2\},
\end{gather*}
where $z_i=\varphi(y(x_i))$ if $x_i \notin I_k, \; 1\leq i \leq k+p$. The values $z_i$ at the points $x_i \in I_k$ are undefined since the trials at the points $x_i \in I_k$ haven't been completed yet. If the value of $M$ equals 0, then set $M=1$.

Step 3. Juxtapose each interval $(x_{i-1},x_i), \; x_{i-1} \notin I_k, x_i \notin I_k, \; 2\leq i\leq k+p$ to a quantity $R(i)$, which is called a characteristic of the interval and is computed according to the formula
\[
R(i)=rM\Delta_i+\frac{(z_i-z_{i-1})^2}{rM\Delta_i}-2(z_i+z_{i-1}),
\]
where $\Delta_i=\left(x_i-x_{i-1}\right)^{1/N}$ and $r>1$ is the reliability parameter of the method.

Step 4. Select the interval $[x_{t-1},x_t]$, which the maximum characteristic corresponds to, i.e.
\[
R(t) = \max \left\{ R(i): \; x_{i-1} \notin I_k, x_i \notin I_k, \; 2\leq i\leq k+p \right\}.
\]

Step 5. Define the new trial point $y^{k+p+1}=y(x^{k+p+1})$, the preimage of which is $x^{k+p+1} \in (x_{t-1},x_t)$ according to the formula
\[
x^{k+p+1} = \frac{x_{t}+x_{t-1}}{2} - \mathrm{sign}(z_{t}-z_{t-1})\frac{1}{2r}\left[\frac{\left|z_{t}-z_{t-1}\right|}{M}\right]^N.
\]

Upon computing the next trial point, the master process adds it to the set $I_k$ and sends it to the worker process, which initiates the new trial at this point. 

The master process terminates the algorithm if one of two conditions is satisfied: $\Delta_{t}<\epsilon$ or $k+p>K_{max}$.
The real number $\epsilon>0$ and the integer number $K_{max}>0$ are the parameters of the algorithm and correspond to the solution search precision and to the maximum number of trials, respectively.

The parallel asynchronous algorithm described above is based on the sequential information global search algorithm. The theoretical substantiation of the algorithm convergence is given in \cite{Strongin2000}. Also, the synchronous parallelization schemes used earlier in solving a number of applied problems \cite{Kalyulin2017,Modorskii2016} are presented here.
The novelty of the present work consists in a practical implementation and application of the asynchronous parallelization scheme featured by a higher efficiency in solving the problems with different computation costs of performing the trials at different points of the search domain. 
It was confirmed by the results of experiments described in the next section.

\section{Numerical Experiments}\label{Sec_Exp}




\section{Conclusions and Future Work}



\medskip

\textbf{Acknowledgments}. This study was supported by the Russian Science Foundation, project No.\,21-11-00204 and by RFBR, project No.\,19-37-60014.

%
% ---- Bibliography ----
%
\bibliographystyle{spmpsci}
\bibliography{bibliography}{}

\end{document}
