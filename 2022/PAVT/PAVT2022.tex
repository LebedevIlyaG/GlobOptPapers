%%%%%%%%%%%%%%%%%%%% author.tex 
%%%%%%%%%%%%%%%%%%%%%%%%%%%%%%%%%%%
%
% sample root file for your "contribution" to a proceedings volume
%
% Use this file as a template for your own input.
%
%%%%%%%%%%%%%%%% Springer 
%%%%%%%%%%%%%%%%%%%%%%%%%%%%%%%%%%


\documentclass{svproc}
%
% RECOMMENDED 
%%%%%%%%%%%%%%%%%%%%%%%%%%%%%%%%%%%%%%%%%%%%%%%%
%%%
%
\usepackage{graphicx}
\usepackage{marvosym}
\usepackage{amsmath}
\usepackage{amssymb}
\usepackage{cite}

\usepackage[russian]{babel}

% to typeset URLs, URIs, and DOIs
\usepackage{url}
\usepackage{hyperref}
\def\UrlFont{\rmfamily}

\def\orcidID#1{\unskip$^{[#1]}$}
\def\letter{$^{\textrm{(\Letter)}}$}

\begin{document}
\mainmatter              % start of a contribution
% Решение обратных задач химической кинетки с помощью асинхронного алгоритма глобального поиска
\title{Solving the Inverse Problems of Chemical Kinetics Using the Asynchronous Global Optimization Algorithm}
%
\titlerunning{Solving the Inverse Problems of Chemical Kinetics}  % abbreviated title (for running head)
%                                     also used for the TOC unless
%                                     \toctitle is used
%
\author{
Irek Gubaydullin$^{1,2}$\and
Leniza Enikeeva$^{2,3}$\orcidID{0000-0003-4219-4870}
\and
Konstantin Barkalov$^4$ \letter \orcidID{0000-0001-5273-2471}
\and
Ilya Lebedev$^4$\orcidID{0000-0002-8736-0652} 
}

%
\authorrunning{I. Gubaydullin et al.} % abbreviated author list (for running head)
%
%%%% list of authors for the TOC (use if author list has to be modified)
%\tocauthor{Konstantin Barkalov and Ilya Lebedev }
%

\institute{$^1$Institute of Petrochemistry and Catalysis – Subdivision of the Ufa Federal Research Centre of RAS, Ufa, Russia\\$^2$Ufa State Petroleum Technological University, Ufa, Russia \\$^3$Novosibirsk State University, Novosibirsk, Russia\\$^4$Lobachevsky State University of Nizhny Novgorod, Nizhny Novgorod, Russia\\
\email{leniza.enikeeva@yandex.ru},
\email{konstantin.barkalov@itmm.unn.ru},
\email{ilya.lebedev@itmm.unn.ru}
}

	
\maketitle              % typeset the title of the contribution

\begin{abstract}

%обновить ключевые слова
\keywords{Global optimization $\cdot$ Multiextremal functions $\cdot$ Parallel computing $\cdot$ Chemical kinetics $\cdot$ Inverse problems }
\end{abstract}

\section{Introduction}


\section{Mathematical Model}\label{Sec_math_mod}


\section{Parallel Global Search Algorithm}\label{Sec_GSA}

Мастер-рабочие, мастер выполняет алгоритм и распределяет задания между рабочими. 

На начальной фазе поиска процесс-мастер инициирует параллельное проведение $p$ испытаний в $p$ произвольных внутренних точках области поиска $\{y(x^1), y(x^2), ...,y(x^p)\}$.

Пусть выполнено $k$ испытаний (в частности, $k$ может быть равно 0), и проводятся испытания в точках $\{y(x^{k+1}), y(x^{k+2}), ...,y(x^{k+p})\}$

На основной фазе поиска каждый процесс-рабочий, завершивший испытание в некоторой точке (будем считать, что это точка $y(x^{k+1})$), информирует об этом процесс-мастер, который, в свою очередь, немедленно выбирает точку нового испытания $x^{k+p+1}$ в соответствии с правилами, описанными ниже.

В данном случае мы будем иметь множество 
\[
I_k = \{x^{k+1},x^{k+2},...,x^{k+p}\}
\]
точек, в которых испытания уже начались, но еще не завершены.

Step 1. Точки множества 
\[
X_k
\]
перенумеровать в порядке возрастания координаты, т.е.
\[
0=
\]
Step 2. Вычислить значения 
\[
M_1=\max \left\{ \frac{ \left|z_i - z_{i-1} \right|}{(x_i-x_{i-1})^{1/N}} : x_{i-1} \notin I_k, x_i \notin I_k, 2\leq i\leq k+p \right\},
\]
\[
M_2=\max \left\{ \frac{ \left|z_{i+1} - z_{i-1} \right|}{(x_{i+1}-x_{i-1})^{1/N}} : x_i \in I_k, 2\leq i < k+p \right\},
\]
\[
M=\max\{M_1,M_2\},
\]
где $z_i=\varphi(y(x_i))$, if $x_i \notin I_k, \; 1\leq i \leq k+p$. Значения $z_i$ в точках $x_i \in I_k$ являются неопределенными, т.к. испытания в точках $x_i \in I_k$ еще не завершены.

Написать про 0 или неопределеность

Step 3. Каждому интервалу поставить в соответствие число 


\section{Numerical Experiments}\label{Sec_Exp}


\section{Conclusions and Future Work}


\medskip

\textbf{Acknowledgments}. This study was supported by the Russian Science Foundation, project No.\,21-11-00204 and by RFBR, project No.\,19-37-60014.

%
% ---- Bibliography ----
%
\bibliographystyle{spmpsci}
\bibliography{bibliography}{}

\end{document}
