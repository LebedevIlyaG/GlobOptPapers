\documentclass[11pt, oneside, a4paper]{article}
\usepackage[utf8]{inputenc}
%\usepackage[cp1251]{inputenc} % кодировка
\usepackage[english, russian]{babel} % Русские и английские переносы
\usepackage{graphicx}          % для включения графических изображений
\usepackage{cite}              % для корректного оформления литературы
\usepackage{pavt-ru}  
\usepackage{amsmath}
\usepackage{amssymb}                              

\begin{document}

% \title - название статьи
% \authors - список авторов

\title{Решение многомерных задач глобальной оптимизации с использованием Intel OneApi\footnote{Исследование выполнено за счет гранта РНФ № 21-11-00204.}}

\authors{К.А. Баркалов\superscript{1}, И.Г. Лебедев\superscript{1} Я.В. Кольтюшкина\superscript{1}}
\organizations{Нижегородский государственный университет им. Н.И. Лобачевского\superscript{1}}

% !!!!!!!!!!!!!!!!!!!!!!!!!!!!!!!!!!!!!!!!!!!!!!!!!!!!!!!!!!!!!!!!!!!!!!!!!!!!
% !!!!!!!!!!!!!!!!!!!!!!!!!!!!!!!!!!!!!!!!!!!!!!!!!!!!!!!!!!!!!!!!!!!!!!!!!!!!
% !!!!!!!!!!!!!!!!!!!!!!!!!!!!!!!!!!!!!!!!!!!!!!!!!!!!!!!!!!!!!!!!!!!!!!!!!!!!

Задача многомерной многоэкстремальной оптимизации может быть определена как поиск наименьшего значения действительной функции \(\varphi(y)\)  в гиперинтервале \(D=\{y\in R^N:a_i\leqslant x_i\leqslant{b_i}, 1\leqslant{i}\leqslant{N}\}\). 
Следует отметить, что в общем случае задача задана по принципу черного ящика. Для целевой функции предполагаем выполнение условия Липшица с априори неизвестной константой \(L\).

В ННГУ им. Н.И. Лобачевского разработан эффективный подход к решению задач глобальной оптимизации \cite{Strongin2013}. В рамках данного подхода решение многомерной задачи сводится к решению одномерной. Для это используется редукция размерности с помощью кривой Пеано $y(x)$, которая непрерывно и однозначно отображает отрезок вещественной оси \([0,1]\) на  \(N\)-мерный куб \cite{Sergeyev2013}.


В процессе работы алгоритма строится последовательность точек \(x^k\), в которых поводятся \textit{испытания}, т.е. вычисляются значения целевой функции \(\varphi^k=\varphi(y(x^k))\).  Распараллеливание организовано следующим образом: в ходе выполнения одной итерации алгоритма одновременно проводится \(P \geq 1\) испытаний в различных точках области поиска.

Одним из вариантов реализации данного подхода является использование инструментов Intel OneAPI. Он прост, открыт и легко интегрируя в существующий код. Intel OneAPI позволяет написать код один раз и в дальнейшем запускать его на различных устройствах. 


На данный момент получены результаты вычислительных экспериментов на генераторе тестовых задач GKLS \cite{GKLS}. В таблице \ref{table:GKLS_RES} приведены среднее число итераций параллельного алгоритма глобального поиска и ускорение относительно последовательной версии алгоритма. В каждом эксперименте решалось 100 размерности $N=4$ и $N=5$.

\begin{table}[!hbp]
    \centering
    \caption{Число итераций и ускорение при решении серии 4-х и 5-ти мерных задач}
    \begin{tabular}{|c|c|c|c|c|}
    \hline
    P    & Итерации ($N=4$) & Ускорение ($N=4$) &         Итерации ($N=5$) & Ускорение ($N=5$) \\ \hline
	128 & 106.5   & 4.7      &         235.6   & 4.4      \\ \hline
	256 & 41.2    & 8.5      &         90.4    & 9.2      \\ \hline
	\end{tabular}
    
    \label{table:GKLS_RES}
\end{table}


% !!!!!!!!!!!!!!!!!!!!!!!!!!!!!!!!!!!!!!!!!!!!!!!!!!!!!!!!!!!!!!!!!!!!!!!!!!!!
% !!!!!!!!!!!!!!!!!!!!!!!!!!!!!!!!!!!!!!!!!!!!!!!!!!!!!!!!!!!!!!!!!!!!!!!!!!!!
% !!!!!!!!!!!!!!!!!!!!!!!!!!!!!!!!!!!!!!!!!!!!!!!!!!!!!!!!!!!!!!!!!!!!!!!!!!!!

\begin{biblio}

\bibitem{Strongin2013}
Стронгин Р.Г. Гергель В.П. Гришагин В.А. Баркалов К.А. Параллельные вычисления
в задачах глобальной оптимизации.
Москва: Издательство Московского университета,
2013, 280 с.

\bibitem{Sergeyev2013}
Sergeyev Y.D., Strongin R.G., Lera D. Introduction to global optimization exploiting space-filling curves. 
New York: Springer, 2013, 125 p.


\bibitem{GKLS}
Gaviano M., Lera D., Kvasov D.E., Sergeyev Ya.D. Software for generation of classes of test functions with known local and global minima for global optimization // ACM Transactions on Mathematical Software. 2003. vol.~29(4). p.~469--480.



\end{biblio}
\end{document}