% !!!!!!!!!!!!!!!!!!   ВНИМАНИЕ !!!!!!!!!!!!!!!!!!!!!!!!!!!!!!!!!!!!!!!!!!!!!!!!!!!!!!!!!
% Заголовки разделов формируются при помощи команд \section{}, \subsection{}, \subsubsection{}
% Не используйте уровень вложенности заголовков больше трех!
% -----------------------------
% Для оформления теорем, лемм, следствий используйте окружения 
% Def     - Определение
% Teor    - Теорема
% Lem     - Лемма
% Predl   - Предложение
% Ass     - Утверждение
% Cor     - Следствие
% Example - Пример
% -----------------------------
% Доказательство теоремы начинается командой \proof и завершается командой \endproof
% -----------------------------
% Литература помещается в окружение biblio.

\documentclass[11pt, oneside, a4paper]{article}
%\usepackage[cp1251]{inputenc} % кодировка
\usepackage[utf8]{inputenc} % кодировка
\usepackage[english, russian]{babel} % Русские и английские переносы
\usepackage{graphicx}          % для включения графических изображений
\usepackage{cite}              % для корректного оформления литературы
\usepackage{enumitem}
\usepackage{amsmath,amsthm,amssymb}
\usepackage{mathtext}


%стилевой пакет
\usepackage{schoolseminar2022}                                


%\renewcommand{\thefootnote}{*}\footnote{Работа выполнена при частичной финансовой поддержке .....}

\begin{document}
% \udk     - универсальный десятичный классификатор
% \msc     - Индекс предметной классификации (Mathematics Subject Classification)
% \title   - название статьи
% \authors - список авторов
\setcounter{page}{1}


%\udk{КодУДК}

\title{Разделение параметров в задачах глобальной оптимизации с помощью методов машинного обучения\footnote{Работа выполнена при поддержке Министерства науки и высшего образования РФ (проект \textnumero~0729-2020-0055) и научно-образовательного математического центра <<Математика технологий будущего>> (проект \textnumero~075-02-2021-1394).}}


\authors{Баркалов К.А., Усова М.А.}
\organizations{Нижегородский государственный университет им. Н.И. Лобачевского \\ Нижний Новгород, Россия}


% \section{название} - заголовок раздела первого уровня
% \subsection{название} - заголовок раздела второго уровня
% \subsubsection{название} - заголовок раздела третьего уровня

\bigskip

В настоящее время методы глобальной оптимизации используются в различных сферах науки и техники, например, для решения обратных задач химической кинетики. В обратных задачах число параметров задачи может составлять десятки и сотни. Использование детерминированных методов для решения задач такой размерности крайне ограничено из-за чрезвычайно больших вычислительных затрат даже при использовании эффективных алгоритмов (например, \cite{Evtushenko2009,Paulavicius2016}). 
При этом многие обратные задачи характеризуются тем, что зависимость от разных групп параметров носит разный характер. Например, от одной группы параметров зависимость может быть близка к линейной, тогда как по второй группе зависимость может носить сложный многоэкстремальный характер. Такие задачи могут быть решены с помощью рекурсивной оптимизации. Однако заранее указать разделение на группы параметров, как правило, не представляется возможным, т.к. целевая функция в обратных задачах задается как черный ящик. 

В работе рассматриваются задачи оптимизации вида 
\begin{eqnarray}\label{main_problem}
& \varphi(y^\ast)=\min{\left\{\varphi(y): y\in D\right\}}, \nonumber \\
& D=\left\{y\in R^N: a_i\leq y_i \leq b_i, 1\leq i \leq N\right\}. \nonumber
\end{eqnarray}
Будем предполагать, что функция $\varphi(y)$ является многоэкстремальной и удовлетворяет условию Липшица с априори неизвестной константой. Считается, что в данной задаче зависимость от разных групп параметров носит разный характер.

Предложена схема выделения параметров задачи, которые оказывают локальное влияние на целевую функцию, что позволяет решать существенно многомерные задачи с использованием рекурсивной оптимизации \cite{Grishagin2007}. Разделение параметров происходит в 3 этапа:

Этап 1. Зафиксировать опорную точку. 

Этап 2. Провести исследование на локальность по каждой переменной: вычислить в заданном количестве точек значение целевой функции в опорной точке при варьировании исследуемой переменной; на основе полученных данных построить регрессионную модель; вычислить оценку $R^2$ и на её основе провести классификацию переменной.

Этап 3. Если локальные переменные были обнаружены, то продолжить решение задачи с использованием многошаговой схемы с разделением локальных и глобальных переменных по уровням. В противном случае запустить алгоритм глобального поиска \cite{Grishagin2007}, считая все переменные глобальными.

В основе проводимой регрессии лежит метод наименьших квадратов (МНК). Оценка качества построенной модели на шаге 2 производится с помощью коэффициента детерминации $R^2$. Подробная схема алгоритма будет представлена в полной версии публикации.

Предложенный подход показал свою работоспособность при решении нескольких серий тестовых задач, представляющих собой линейные комбинации подзадач с глобальными параметрами (функции Гришагина или GRIS, функции Сергеева или GKLS \cite{Grishagin2001}) и подзадач с локальными параметрами, представляющих собой комбинацию близких к линейным одномерных функций (далее L) или  комбинацию  функций с разным вкладом квадратичной составляющей (далее LQ).

По результатам экспериментов при решении серий задач с существенной многоэкстремальностью (GRIS) было проведено корректное разделение на глобальные и локальные переменные, а затем задачи были успешно решены при помощи многошаговой схемы за приемлемое время и количество итераций. Задачи с функциями  GKLS для алгоритма являются более сложными, в силу схожести части параметров функций с параболоидами, поэтому решить все задачи серии за ограниченное количество итераций метода не удалось.

Вычисления проводились на компьютере с CPU Intel Core™ i7 10750H, 2.6 GHz. Для построения и анализа регрессионной модели использовался пакет scikit-learn из Python.
Численные результаты серий экспериментов приведены в Таблице \ref{tab1}.
В ней представлены данные о числе решенных задач $S$, среднем числе итераций глобального поиска на верхнем уровне рекурсии $G_{av}$, среднем числе испытаний на нижнем уровне $L_{av}$, среднем времени решения задачи $t_{av}$ (в секундах).

\begin{table}[ht]
	\caption{Результаты решения серий тестовых задач}
	\label{tab1}
	\begin{center}
		\begin{tabular}{ l c c c c c c } \hline
		 & $S$ &  $G_{av}$ &  $L_{av}$ & $t_{av}$ \\
    \hline
		GRIS-L & 20/20  & 705 &  378 545 & 1.3 \\
		GRIS-LQ & 20/20 & 705 &  420 397 & 1.4 \\
		GKLS-L & 18/20 & 9531 &  3 899 417 & 16.4 \\
		\hline
		\end{tabular}
	\end{center}
\end{table}

\begin{biblio}

\bibitem{Evtushenko2009} Евтушенко Ю. Г., Малкова В. У., Станевичюс А.-И. А. Параллельный поиск глобального экстремума функций многих переменных // Журн. вычисл. матем. и матем. физ. 2009. Т. 49. № 2. C. 255--269.

\bibitem{Paulavicius2016} Paulavi{\v c}ius R., {\v Z}ilinskas J. Advantages of Simplicial Partitioning for Lipschitz Optimization Problems with Linear Constraints // Optim. Lett. 2016. V. 10. No. 2. P. 237--246.

\bibitem{Grishagin2007} Городецкий С. Ю., Гришагин В. А. Нелинейное программирование и многоэкстремальная оптимизация. Н. Новгород: Изд-во ННГУ, 2007. 489 с.

\bibitem{Grishagin2001} Sergeyev Y., Grishagin V. Parallel Asynchronous Global Search and the Nested Optimization Scheme // J. Comput. Anal. Appl. 2001. V. 3, No. 2. P. 123--145.

%\bibitem{Sergeyev2008} Сергеев Я. Д., Квасов Д. Е. Диагональные методы глобальной оптимизации. М.: Физматлит, 2008. 352 с.

%\bibitem{Strongin13} Стронгин Р.Г., Гергель В.П., Гришагин В.А., Баркалов К.А. Параллельные вычисления в задачах глобальной оптимизации. М.: Изд-во МГУ, 2013. 280 с.

\end{biblio}

\title{Parameters Separation in Global Optimization Problems Using Machine Learning Methods}

\authors{Konstantin Barkalov, Marina Usova}
\organizations{ Lobachevsky State University of Nizhny Novgorod, Nizhny Novgorod, Russia}

\end{document}

