%% 
%% Copyright 2007-2020 Elsevier Ltd
%% 
%% This file is part of the 'Elsarticle Bundle'.
%% ---------------------------------------------
%% 
%% It may be distributed under the conditions of the LaTeX Project Public
%% License, either version 1.2 of this license or (at your option) any
%% later version.  The latest version of this license is in
%%    http://www.latex-project.org/lppl.txt
%% and version 1.2 or later is part of all distributions of LaTeX
%% version 1999/12/01 or later.
%% 
%% The list of all files belonging to the 'Elsarticle Bundle' is
%% given in the file `manifest.txt'.
%% 

%% Template article for Elsevier's document class `elsarticle'
%% with numbered style bibliographic references
%% SP 2008/03/01
%%
%% 
%%
%% $Id: elsarticle-template-num.tex 190 2020-11-23 11:12:32Z rishi $
%%
%%
 \documentclass[preprint,12pt]{elsarticle}

%% Use the option review to obtain double line spacing
%% \documentclass[preprint,review,12pt]{elsarticle}

%% Use the options 1p,twocolumn; 3p; 3p,twocolumn; 5p; or 5p,twocolumn
%% for a journal layout:
%% \documentclass[final,1p,times]{elsarticle}
%% \documentclass[final,1p,times,twocolumn]{elsarticle}
%% \documentclass[final,3p,times]{elsarticle}
%% \documentclass[final,3p,times,twocolumn]{elsarticle}
%% \documentclass[final,5p,times]{elsarticle}
%% \documentclass[final,5p,times,twocolumn]{elsarticle}

%% For including figures, graphicx.sty has been loaded in
%% elsarticle.cls. If you prefer to use the old commands
%% please give \usepackage{epsfig}

%% The amssymb package provides various useful mathematical symbols
\usepackage{amssymb}
%% The amsthm package provides extended theorem environments
%%\usepackage{amsthm}

%% The lineno packages adds line numbers. Start line numbering with
%% \begin{linenumbers}, end it with \end{linenumbers}. Or switch it on
%% for the whole article with \linenumbers.
\usepackage{lineno}

%% Русский язык - закомментировать в финальной версии
\usepackage[russian]{babel}

%% advanced mathematics
\usepackage{amsmath}
\usepackage[ruled]{algorithm2e}

\journal{Knowledge-Based Systems}

\begin{document}

\begin{frontmatter}

%% Title, authors and addresses

%% use the tnoteref command within \title for footnotes;
%% use the tnotetext command for theassociated footnote;
%% use the fnref command within \author or \address for footnotes;
%% use the fntext command for theassociated footnote;
%% use the corref command within \author for corresponding author footnotes;
%% use the cortext command for theassociated footnote;
%% use the ead command for the email address,
%% and the form \ead[url] for the home page:
%% \title{Title\tnoteref{label1}}
%% \tnotetext[label1]{}
%% \author{Name\corref{cor1}\fnref{label2}}
%% \ead{email address}
%% \ead[url]{home page}
%% \fntext[label2]{}
%% \cortext[cor1]{}
%% \affiliation{organization={},
%%             addressline={},
%%             city={},
%%             postcode={},
%%             state={},
%%             country={}}
%% \fntext[label3]{}

\title{Title of the paper}

%% use optional labels to link authors explicitly to addresses:
%% \author[label1,label2]{}
%% \affiliation[label1]{organization={},
%%             addressline={},
%%             city={},
%%             postcode={},
%%             state={},
%%             country={}}
%%
%% \affiliation[label2]{organization={},
%%             addressline={},
%%             city={},
%%             postcode={},
%%             state={},
%%             country={}}

\author[UNN,ITMO]{Konstantin Barkalov}
\author[UNN,ITMO]{Evgeny Kozinov}
\author[UNN,ITMO]{Ilya Lebedev}
\author[UNN,ITMO]{Denis Karchkov}
\author[UNN,ITMO]{Denis Rodionov}

\affiliation[UNN]{organization={Lobachevsky University},%Department and Organization
            addressline={Gagarin Av. 23}, 
            city={Nizhni Novgorod},
            postcode={603022}, 
            %state={},
            country={Russia}}
						
\affiliation[ITMO]{organization={ITMO University},%Department and Organization
            addressline={Lomonosova St. 9}, 
            city={St. Petersburg},
            postcode={191002}, 
            %state={},
            country={Russia}}


\begin{abstract}
%% Text of abstract

\end{abstract}

%%Graphical abstract
%\begin{graphicalabstract}
%\includegraphics{grabs}
%\end{graphicalabstract}

%%Research highlights
\begin{highlights}
\item Research highlight 1
\item Research highlight 2
\end{highlights}

\begin{keyword}
%% keywords here, in the form: keyword \sep keyword

%% PACS codes here, in the form: \PACS code \sep code

%% MSC codes here, in the form: \MSC code \sep code
%% or \MSC[2008] code \sep code (2000 is the default)

\end{keyword}

\end{frontmatter}

\linenumbers

%% main text
\section{Introduction}
\label{sec_intro}
Подбор гиперпараметров - важная и сложная задача.

Много разных фреймворков, у всех есть свои достоинства и недостатки.


\section{Related work}
\label{sec_rel}

Простые задачи - метод перебора

Задачи большой размерности - метаэвристические алгоритмы

Задачи малой размерности - байесовская оптимизация.

Липшицева оптимизация - основные понятия, ссылки на работы. Соотношение с другими подходами. Проблема дискртетных параметров. 



\section{Lipschitz optimization} 
\label{sec_lip}

\subsection{Problem statement} 

Мы будем рассматривать задачу поиска глобального минимума в следующей постановке:
\begin{gather}
	\varphi(y^*) = \min \varphi(y), \; y \in D, \label{f_func} \\
	D = \left\{y \in R^N : a_j \leq y_j \leq b_j , \; 1 \leq j \leq N \right\}. \label{f_D}
\end{gather}

Будем также предполагать, что целевая функция $\varphi(y)$ удовлетворяет в области поиска $D$ условию Липшица, т.е.
\begin{equation} \label{f_lip}
	\left| \varphi(y')-\varphi(y'') \right| \leq L\left\| y' - y''  \right\| , \; y',y'' \in D, \; 0<L<\infty.
\end{equation}
Здесь $ \left\| \cdot \right\|$ обозначает евклидову норму, а константа $L$ является априори не изветной и подлежит оценке в процессе решения. 
%Такие задачи называются задачами липшицевой глобальной оптимизации with box-constraints.

Функция $\varphi(y)$ предполагается многоэкстремальной и заданной в виде ``черного ящика'', т.е. некоторого алгоритма вычисления ее значений. Кроме того, подразумевается, что каждое поисковое испытание (т.е. вычисление значения функции в точке допустимой области) требует значительных вычислительных ресурсов. Такая постановка проблемы полностью соответствует задаче настройки гиперпараметров методов машинного обучения (see ).

Условие Липшица допускает следующую геометрическую интерпретацию. Допустим, что одномерная липшицева функция $f(x)$ (с известной константой Липшица $L$) была вычислена в двух точках $x'$ и $x''$. В соответствии с условием (\ref{f_lip}) выполняются следующие неравенства, характеризующие поведение функции $f(x)$ на интервале $[x', x'']$:
\begin{gather*}
	f(x) \leq f(x') + L(x-x'), \; x \geq x',\\
	f(x) \geq f(x') - L(x-x'), \; x \geq x',\\
	f(x) \leq f(x') - L(x-x''), \; x \leq x'',\\
	f(x) \geq f(x') + L(x-x''), \; x \leq x''.
\end{gather*}

В силу данных неравенств значения функции в точках интервала $[x', x'']$ должны располагаться внутри области, ограниченной прямыми, проходящими через точки $(x', f(x'))$ и $(x'', f(x''))$ с наклоном $L$ и $-L$ (см. рис. ).

Также в соответствии с (\ref{f_lip}) может быть записано неравенство $f(x) \geq F(x)$, где функция $F(x)$ (называемая минорантой) определяется в соответствии с формулой
\[
F(x) = \max\left\{f(x') - L(x-x'),f(x') + L(x-x'')\right\}, \; x\in [x', x''].
\] 

Наименьшее значение миноранты $F(x)$ на $[x', x'']$ совпадает с оценкой наименьшего значения функции $f(x)$ на данном интервале. Указанная оценка достигается в точке 
\[
\overline{x} = \frac{x'+x''}{2}-\frac{f(x'')-f(x')}{2L}
\] 
и равна
\[
F(\overline{x}) = \frac{f(x')+f(x'')}{2} -L \frac{x''-x'}{2}.
\]

Методы липшицевой глобальной оптимизации используют эту идею в своих вычислительных правилах; на ней также основано и доказательство их сходимости (see, e.g., ).

\subsection{Проблема редукции размерности} 

Одной из основных трудностей при решении многомерных задач глобальной оптимизации является рост вычислительных затрат при увеличении размерности задачи. Эта сложность имеет место и при решении задач липшицевой глобальной оптимизации.
Например, решение задачи (\ref{f_func}) c помощью поиска по равномерной сетке с шагом $\epsilon$ потребует проведения 
\[
K \approx \prod_{j=1}^N{\frac{b_j-a_j}{\epsilon}}
\]
испытаний. Очевидно, что $K$ экспоненциально возрастает с ростом $N$.
Уменьшение числа испытаний при тех же требованиях к точности решения возможно за счет адаптивного построения неравномерной сетки в области изменения параметров.

Известным подходом к решению многомерной задачи (\ref{f_func}) является ее редукция к одномерной задаче с последующим применением эффективных одномерных алгоритмов глобальной оптимизации (see []). Здесь хорошо зарекомендовал себя подход, в котором размерность задачи редуцируется при помощи кривой Пеано–Гильберта $y(x), x \in [0, 1]$.
Такая кривая однозначно и непрерывно отображает интервал $[0, 1]$ на гиперинтервал $D$ из (\ref{f_D}), т.е. заполняет всю область $D$.

Для решения задачи глобальной липшицевой оптимизации кривые, заполняющие пространство, могут быть применены следующим образом.
Если $y(x)$ есть кривая Пеано–Гильберта, то из непрерывности целевой функции $\varphi(y)$ следует, что
\[
\min_{y \in D } \varphi(y) = \min_{x \in [0,1] } \varphi(y(x)),
\]
т.е. исходная многомерная задача (\ref{f_func}) редуцируется к одномерной.


При этом известно (see []), что если многомерная функция $\varphi(y), \; y \in D,$  удовлетворяет условию Липшица с константой $L$, то редуцированная одномерная функция $\varphi(y(x)), \; \in [0,1]$ удовлетворяет the H{\"o}lder condition
\[
\left|\varphi(y(x'))-\varphi(y(x''))\right|\leq H\left|x'-x''\right|^{1/N}, \; x',x''\in[0,1].
\]
Здесь $N$ есть размерность исходной задачи, а коэффициент $ H=2 L \sqrt{N+3}$.

Замечание 1. Для редуцированной одномерной функции $\varphi(y(x))$  не будет выполняться условие Липшица. Однако многие одномерные алгоритмы липшицевой оптимизации
могут быть обобщены на случай минимизации гeльдеровых функций (see []). Конкретный пример такого обобщения приведен в subsection \ref{sec_GSA}. 

Замечание 2. Теоретическая кривая Пеано-Гильберта $y(x)$ определяется как предельный объект. В численных алгоритмах применяются развертки $y_m(x)$, аппроксимирующие истинную кривую Пеано–Гильберта c заданным уровнем точности $2^{-m}$, зависящим от требуемой точности поиска. Эффективные схемы вычисления нескольких видов разверток подробно описаны в \cite{Sergeyev2013}.
%пример развертки - рисунок с функцией


\subsection{Проблема дискретных параметров} 

Особый интерес представляют задачи, в которых часть параметров может принимать значения из заранее заданного конечного множества, т.к. для них сложнее построить оценки оптимума по сравнению с задачами с непрерывными параметрами.

Методам решения задач со смешанными параметрами посвящено много публикаций (см., например, обзоры \cite{Burer2012,Boukouvala2016}). 
Известные детерминированные методы решения задач данного класса ориентированы на решение линейных или выпуклых задач.
Многоэкстремальные задачи предлагается решать с помощью метаэвристических алгоритмов.

Нами был предложен и реализован новый детерминированный подход к решению задач со смешанными параметрами; приведем здесь его краткое описание. 

Пусть  целевая функция задачи зависит от двух векторов параметров: вектора $y$, принадлежащего гиперинтервалу $D$, и вектора $u$, имеющий конечный (и при этом не очень большой) набор возможных значений, т.е. 
\begin{gather}\label{problem_i}
\min{\left\{ \varphi(y,u) : y\in D \right\}},\\
D=\left\{a_j \leq y_j \leq b_j, \; 1\leq j \leq N \right\}.\nonumber
\end{gather}

Такие конечные наборы значений могут характеризовать множество возможных сочетаний категориальных гиперпараметров исследуемого алгоритма машинного обучения. 
%Целочисленные параметры = непрерывные параметры с округлением, добавить комментарий об этом.

Занумеруем целыми числами $s, 1\leq s \leq S,$ все различные комбинации категориальных параметров, т.е. сопоставим каждому номеру $s$ вектор $u_s$. 
Тогда рассматриваемая задача может быть записана в виде 
\begin{gather}\label{problem_is}
 \min_{s\in\{1,...,S\}}\left[\min{\left\{ \varphi(y,u_s) : y\in D \right\}}\right].
\end{gather}

Используя схему редукции размерности с помощью кривых Пеано $y(x), x\in [0,1]$ можно сопоставить каждой задаче минимизации по $y$ одномерную задачу минимизации
\[
 \min{\left\{ \varphi(y(x),u_s): x \in [0,1] \right\}}, s \in \{1,...,S\}.
\]

Рассмотрим теперь отображение 
\[
Y(x)=y(x-E(x)), \; x\in[0,S],
\]
переводящее любую точку интервала [0,S] на область $D$ (обозначение $E(x)$ соответствует целой части числа $x$) и определим функцию 
\[
f(x) = \varphi(Y(x),u_{E(x)+1}), x\in[0,S],
\]
имеющую, вообще говоря, разрывы в целочисленных точках $x_i = i, 1\leq i \leq S-1$.
Поэтому значения  $z_i = f(x_i)$ в этих точках будем считать неопределенным, а значения индекса -- равным 0, т.е. $\nu(x_i) = 0$.

Используя введенные обозначения можно переформулировать исходную задачу как
\begin{equation}\label{problem_is1}
\min \left\{f(x): x \in [0,S] \right\}.
\end{equation}

Применяя к решению задачи (\ref{problem_is1}) алгоритм глобального поиска, мы найдем решение задачи (\ref{problem_i}). При этом основная часть испытаний будет проведена в $s$-й подзадаче, решение которой соответствует решению исходной задачи (\ref{problem_i}). В остальных подзадачах будет проведена лишь незначительная часть испытаний, т.к. решения данных подзадач являются локально-оптимальными по отношению к решению $s$-й подзадачи.


Test reference \cite{Strongin2000,Barkalov2021}.


\section{iOpt framework} 
\label{sec_iOpt}

Дать характеристику фреймворка - язык разработки, используемые технологии, ориентация на какие задачи, и т.п. 


\subsection{Global search algorithm}
\label{sec_GSA}

Алгоримт глобального поиска - ядро фреймворка. 

Словесное описание алгоритма.

Формулы, определяющие его работу  (вычисление характеристик, оценка константы Липшица, точка нового испытания).

Описание в виде алгоритма в LaTeX

\begin{algorithm}
\LinesNumbered
 \KwIn{this text}
 \KwOut{how to write algorithm with \LaTeX2e }
 Initialize $k=1$\\
 \While{ $\Delta_t \geq \epsilon$ and  $k \leq K_{max}$}{
  read current\\
  \eIf{understand}{
   go to next section\\
   current section becomes this one
   }{
   go back to the beginning of current section
  }
 $k = k + 1$\\
 }
 \KwRet{$y_{min}$}
 \caption{Global search}\label{alg_GSA}
\end{algorithm}

\subsection{GSA for the problems with categorical parameters}
\label{sec_mGSA}

Здесь описать схему назначения целочисленных значений категориальным параметрам, и схему сведения к задаче  (\ref{problem_is1}).

потом сказать, что задача решается алгоритмом \ref{alg_GSA}.


\section{Numerical experiments}
\label{sec_exp}

\subsection{Experimental setup}

Datasets

Metrics

Algorithms for comparison

\subsection{Experimental results}

First result

Second result 

Third result





\section*{CRediT authorship contribution statement}

Konstantin Barkalov:
Denis Rodionov: 


\section*{Declaration of competing interest}

The authors declare that they have no known competing financial interests or personal relationships that could have appeared to influence the work reported in this paper.

\section*{Acknowledgements}

This research was supported by the research center ``Strong Artificial Intelligence in Industry'' of ITMO University.



%% The Appendices part is started with the command \appendix;
%% appendix sections are then done as normal sections
%% \appendix

%% \section{}
%% \label{}

%% If you have bibdatabase file and want bibtex to generate the
%% bibitems, please use
%%
\bibliographystyle{elsarticle-num} 
\bibliography{bibliography}

%% else use the following coding to input the bibitems directly in the
%% TeX file.

%\begin{thebibliography}{00}

%% \bibitem{label}
%% Text of bibliographic item

%\bibitem{}

%\end{thebibliography}
\end{document}
\endinput
%%
%% End of file `elsarticle-template-num.tex'.
