% Latex-template for one-page abstract submissions.
% NUMTA2023 - Numerical Computations: Theory and Algorithms
% 14-20 June 2023, TUI Magic Life Calabria, Pizzo Calabro (Italy)

\documentclass[oribibl]{llncs}

% Please, do not use your own macros/redefinitions.

\usepackage{fancyhdr}
\renewcommand{\headrulewidth}{0pt}
\renewcommand{\footrulewidth}{0pt}

\begin{document}
\cleardoublepage

\title{Title of Your Presentation}
\author{Name1 Surname1, Name2 Surname2}

%%% Please note that the first author is supposed to be the speaker.

\institute{University1 of Place1, Address1, Country1 \\
           University2 of Place2, Address2, Country2 \\
\email{Name1@university1.edu, Name2@university2.edu}}

\maketitle

\thispagestyle{fancy}

\textbf{Keywords.} Keyword1; keyword2; keyword3.

 \vspace*{0.5cm}

%-----------------------------------------------------------------

The goal of the Conference is to create a multidisciplinary round
table for an open discussion on numerical modeling nature by using
traditional and emerging computational paradigms. The Conference
will discuss all aspects of numerical computations and modeling from
foundations and philosophy to advanced numerical techniques. New
technological challenges and fundamental ideas from theoretical
computer science, linguistic, logic, set theory, and philosophy will
meet requirements and new fresh applications from physics,
chemistry, biology, and economy.

Researchers from both theoretical and applied sciences are very
welcome to submit their papers and to use this excellent possibility
to exchange ideas with leading scientists from different research
fields. Papers discussing new computational paradigms, relations
with foundations of mathematics, and their impact on natural
sciences are particularly solicited. A special attention will be
also dedicated to numerical optimization and different issues
related to theory and practice of the usage of infinities and
infinitesimals in numerical computations.

Papers discussing new computational paradigms and their impact on
natural sciences are particularly solicited.

\textit{Please, avoid figures and tables in the text. Max 4
bibliographic references are allowed.}

\textbf{Acknowledgements.}

This research was supported by the following grants...

 \vspace{0.5cm}

\begin{thebibliography}{4}

\bibitem{Book} Cantor G. (1955) \emph{Contributions to the founding of
the theory of transfinite numbers}. Dover Publications, New York.

\bibitem{Article} Gordon P. (2004) Numerical cognition without words:
{E}vidence from {A}mazonia. \emph{Science}, Vol.~306,
pp.~496--499.

\bibitem{Proceedings} Author I., Author I.I. (2019) An
abstract. In Proceedings of the \emph{International
Conference``XXX''} (ed. by Editor E.), Place (Country), p.~1.

\bibitem{BookChapter} Author X. (2019) A
chapter. In \emph{YYY Book} (ed. by Editor E.), Place (Country),
pp.~10--20.

\end{thebibliography}

\end{document}
