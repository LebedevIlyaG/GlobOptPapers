% Latex-template for one-page abstract submissions.
% NUMTA2023 - Numerical Computations: Theory and Algorithms
% 14-20 June 2023, TUI Magic Life Calabria, Pizzo Calabro (Italy)

\documentclass[oribibl]{llncs}

% Please, do not use your own macros/redefinitions.

\usepackage{fancyhdr}
\renewcommand{\headrulewidth}{0pt}
\renewcommand{\footrulewidth}{0pt}

\begin{document}
\cleardoublepage

\title{Application of Machine Learning to Increase the Efficiency of the Global Search Algorithm for Solving Multicriterial Problems}
\author{Konstantin Barkalov, Vladimir Grishagin, Evgeny Kozinov}

%%% Please note that the first author is supposed to be the speaker.

\institute{Lobachevsky State University of Nizhni Novgorod, Nizhni Novgorod, Russia \\
\email{\{konstantin.barkalov,evgeny.kozinov\}@itmm.unn.ru}, vagris@unn.ru}

\maketitle

\thispagestyle{fancy}

\textbf{Keywords.} Multicriterial problems; global optimization; logistic regression. % machine learning; 

 \vspace*{0.5cm}

%-----------------------------------------------------------------

Decision making models described as problems of multicriterial optimization are very complicated for investigation because of the property of criteria contradictoriness that leads to the notion of the solution as the set non-dominated parameters (Pareto set). The complexity of these problems increases significantly in the case where criteria are multiextremal. In the approaches based on convolution schemes reducing the multicriterial problem to a family of scalar (single criterion) optimization subproblems, these subproblems are multiextremal as well. This circumstance requires the use of global optimization methods to solve those.

In the paper a novel algorithm for multicriterial black-box optimization with multiextremal criteria are considered. This algorithm applies ideas of complexity reduction when the initial multicriterial problem are reduced to a set of scalar subproblems on the base of maximum convolution, then each scalar multidimensional subproblem is transformed to equivalent one-dimensional problem by means of Peano mapping which provides single-valued correspondence between N-dimensional hyperparallelepiped and the unit interval of the real axis. For solving univariate problems a global optimization algorithm with guaranteed convergence to global optimum is used. As a core novel feature, this algorithm includes in its computational scheme a machine learning procedure with combination of accumulating the information of solved subproblems that allows one to significantly accelerate the building the Pareto set.

For verification of workability and estimation of efficiency, the representative computational experiment on test sets of multicriterial problems with multiextremal criteria constructed on the base test class GKLS  has been conducted for different dimensionalities and number of criteria. 

Along with the proposed algorithm, in the experiment some known method belonging to the class of genetic algorithms have been tested and compared where the hypervolume index  HV was used for assessment of algorithms' quality. According to the results of the experiment, the proposed algorithm  has demonstrated its essential advantage over other methods.


\textbf{Acknowledgements.}

This research was supported by the Russian Science Foundation, project No. 21-11-00204.

 \vspace{0.5cm}

\begin{thebibliography}{4}

\bibitem{Book} Cantor G. (1955) \emph{Contributions to the founding of the theory of transfinite numbers}. Dover Publications, New York.

\bibitem{Article} Gordon P. (2004) Numerical cognition without words: {E}vidence from {A}mazonia. \emph{Science}, Vol.~306,
pp.~496--499.

\end{thebibliography}

\end{document}
